
% * Commands for the document
% * -------------------------------------------------------------------
% * User-defined macros (functions) can go here, 
% * these may be overriden for each subfile by using \renewcommand{cmd}[args][default]{def} or simply \renewcommand{cmd}{def}


% * --- Layout & Graphics --- * %
%% Defines a new command for the horizontal lines. Change thickness here
\newcommand{\HRule}{\rule{\linewidth}{0.5mm}} 


%\renewcommand{\arraystretch}{1.25}  % For stretching tables and arrays


% * --- Crossreferencing --- * %
%% Avoids some words when crossreferencing to an enumerated list of equations within a theorem environment
%% Using \cref{thrm:my_theorem} \cref{eq:mt_c} will print "Theorem 2.0.2 (c)" and not "Theorem 2.0.2 Item (c)"
%% Original settings for both \crefname{} and \Crefname collected from cleverref pacakge .sty file
%\Crefname{enumi}{Item}{Items}%
%\Crefname{enumii}{Item}{Items}%
%\Crefname{enumiii}{Item}{Items}%
%\Crefname{enumiv}{Item}{Items}%
%\Crefname{enumv}{Item}{Items}%

\Crefname{enumi}{}{}%
\Crefname{enumii}{}{}%
\Crefname{enumiii}{}{}%
\Crefname{enumiv}{}{}%
\Crefname{enumv}{}{}%

\crefname{enumi}{}{}%
\crefname{enumii}{}{}%
\crefname{enumiii}{}{}%
\crefname{enumiv}{}{}%
\crefname{enumv}{}{}%


% * --- Mathematical symbol commands --- * %
% Symbols
\newcommand{\nsubset}{\not\subset}

% Arabic letters
\newcommand{\N}{\mathbb{N}}  % Already defined
\newcommand{\Z}{\mathbb{Z}}
\newcommand{\bbP}{\mathbb{P}}  % as \P yields some note P
\newcommand{\PP}{\mathbb{PP}}
\newcommand{\Q}{\mathbb{Q}}
\newcommand{\R}{\mathbb{R}}
\newcommand{\C}{\mathbb{C}}  % Already defined

\newcommand{\Fp}{\mathbb{F}_p}
\newcommand{\Fq}{\mathbb{F}_q}
\newcommand{\indic}{\mathbb{1}}

% Greek letters
\newcommand{\Tau}{\mathrm{T}}


% Z with strikethrough
\newcommand{\Zstroke}{\text{\ooalign{\hidewidth\raisebox{0.2ex}{--}\hidewidth\cr$Z$\cr}}}
\newcommand{\zstroke}{\text{\ooalign{\hidewidth -\kern-.3em-\hidewidth\cr$z$\cr}}}


% Degrees and-such
\newcommand{\degc}{$^\circ$C~}
\renewcommand{\deg}{\ensuremath{^{\circ}}}
\newcommand{\edegc}{^\circ \text{C}~}
\newcommand{\minone}{$^{-1}$}
\newcommand{\eminone}{^{-1}}


% Brackets
%% normal-brackets
\newcommand{\brac}[1]{(#1)}
\newcommand{\bracMed}[1]{\left(#1\right)}  % Husk at \left og \right legger på mellomrom. Ex: E (A) vs. E(A) 
%% Square-brackets
\newcommand{\bras}[1]{[#1]}
\newcommand{\brasMed}[1]{\left[#1\right]}
%% Curly-brackets
\newcommand{\braq}[1]{\{#1\}}
\newcommand{\braqMed}[1]{\left\{#1\right\}}
%% Angled-brackets
\newcommand{\braa}[1]{\langle#1\rangle}
\newcommand{\braaMed}[1]{\left \langle #1 \right \rangle}
%% Absolute values
\newcommand{\bral}[1]{|#1|}
\newcommand{\bralMed}[1]{\left|#1\right|}


% Operator names  
% A more high-level approach is to use \DeclareMathOperator{\max}{max} to create a shortcut with brackets and operator name at once, \max -> max(.)
\newcommand{\spn}[1]{\operatorname{span}\brac{#1}}
\newcommand{\spnMed}[1]{\operatorname{span}\bracMed{#1}}
\newcommand{\spnclos}[1]{\overline{\operatorname{span}}\brac{#1}}
\newcommand{\spnclosMed}[1]{\overline{\operatorname{span}}\bracMed{#1}}
\newcommand{\mes}[1]{\operatorname{mes}\brac{#1}}
\newcommand{\mesMed}[1]{\operatorname{mes}\bracMed{#1}}


\newcommand{\Cper}{C_{\text{per}}}
\newcommand{\intnozero}{\Z\setminus\braq{0}} 

% others 
\newcommand{\inpl}[1]{\langle \lambda, #1 \rangle}  % inner product w. lambda
\newcommand{\indicator}[2]{\mathbbm{1}_{#1}(#2)}  % indicator 1_A (t) 
\newcommand{\indicatorNoVar}[1]{\mathbbm{1}_{#1}}  % indicator 1_A 

% Set of exponential functions
\newcommand{\onedexp}{ \braq{e^{2\pi i \lambda t } : \lambda \in \Z} }
\newcommand{\alldexp}{ \braq{e^{2\pi i \inpl{t}} : \lambda \in \Z^d} }

% text replacements
\newcommand{\Ltwonorm}{$\|\cdot\|_{L^2}$-norm}  %! Uten dollartegn, rett i teksten ->
\newcommand{\LPnorm}{$\|\cdot\|_{L^p}$-norm}  %! HUSK, hvis det ikke er siste ord i setningen, bruk \space etter komandoen
\newcommand{\GenNormX}{$\|\cdot\|_{X}$-norm}
\newcommand{\GenNormH}{$\|\cdot\|_{H}$-norm}

\newcommand{\lambfunc}{\mathbf{L}(\lambda_1)}  % will type L(lambda_1)  consider: \mathsf{}, \mathbf{}, or L
\newcommand{\lambfuncNoVar}{\mathbf{L}}  % no variable
\newcommand{\lambfuncGen}[1]{\mathbf{L}(#1)}  % general, must give input
\newcommand{\betafunc}{\bm{\beta}} 


% * --- Other Settings --- * %
% \newcommand{\masl}{m.a.s.l}


\newcommand{\mycomment}[1]{} %! Block comment
\newcommand{\SigridComment}[1]{\textcolor{green}{kommentar: #1}}
\newcommand{\SigridChange}[1]{\textcolor{green}{#1}}

% For use in enumerate, itemize, description write \item \printvalues to get the following values
\newcommand{\printvalues}{topsep=\the\topsep; itemsep=\the\itemsep; parsep=\the\parsep; partopsep=\the\partopsep}