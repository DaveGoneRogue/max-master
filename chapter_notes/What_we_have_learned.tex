



%* Section: Results on vector spaces
On the matter of sequence notation v.s. Element of notation. It seems that Heil's definition on page 159 allows for the option that it is a sequence. And on page 67, they make no mention of the fact that it can be a sequence.
%? Solved
writing y_n without sequence notation can be interpreted as we are looking at "one element" and not a bunch of elements which we have from the sum in the definition of the span


%* Section: Results on vector spaces
Note that any set of nonzero orthogonal vectors can be turned into an orthonormal set by dividing each vector by its own length. 
Thus, results stated for orthonormal sets also hold for orthogonal sets. The converse is not necessarily true.
%? Partly solved
Thought: The answer is no. The Reason being is that the requirement of $1$ is so powerful. However, since everything can be made orthonormal, it will hold for the orthonormal variant of it. 



%* On spectral sets
%? chat open-ai
Your intuition is correct! In many cases, the spectrum associated with a spectral set serves as an orthonormal basis (ONB) for the space of square-integrable functions (L^2 space) defined on the spectral set.

When the set of exponentials corresponding to the spectrum forms an orthogonal basis for the L^2 space of the spectral set, it means that these exponentials can be used as a complete set of functions to represent any square-integrable function defined on the spectral set. Each function in the L^2 space can be expressed as a linear combination of the exponentials associated with the spectrum.

This property is particularly useful in spectral analysis and signal processing because it allows us to decompose and represent signals or functions on the spectral set in terms of their spectral components. The spectrum acts as a basis that captures the essential frequency information, enabling efficient analysis, manipulation, and synthesis of signals or functions in the L^2 space of the spectral set.

%? 2
The existence of an orthogonal basis formed by the exponentials associated with the spectrum Λ suggests that the functions in E(Λ) capture distinct frequency components, which complement each other and form a complete representation of functions in L2(Ω).

%? Max
A spectral pair $\brac{\Omega,\Lambda}$ is a complementary relationship between a set $\Omega$ and a set of frequencies $\Lambda$ used to create a set of exponential functions which can represent any square-integrable function defined on $\Omega$ completely. 