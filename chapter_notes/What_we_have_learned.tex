



%* Section: Results on vector spaces
On the matter of sequence notation v.s. Element of notation. It seems that Heil's definition on page 159 allows for the option that it is a sequence. And on page 67, they make no mention of the fact that it can be a sequence.
%? Solved
writing y_n without sequence notation can be interpreted as we are looking at "one element" and not a bunch of elements which we have from the sum in the definition of the span


%* Section: Results on vector spaces
Note that any set of nonzero orthogonal vectors can be turned into an orthonormal set by dividing each vector by its own length. 
Thus, results stated for orthonormal sets also hold for orthogonal sets. The converse is not necessarily true.
%? Partly solved
Thought: The answer is no. The Reason being is that the requirement of $1$ is so powerful. However, since everything can be made orthonormal, it will hold for the orthonormal variant of it. 




