
\mycomment{
Jo høyere dimensjon vi får jo mer eksotisk kan disse tilingene bli. Står noe referanser til i paperet. 
Keller hadde en teori om at enhver tiling må være av den typen -- (Se for deg enhetskuben i 2D)
enhver tiling nødvendigvis medføre at,
alle tiles har en full felles kant med en annen tile

i 2D gjelder det her med at man har to kanter hvor man har en full felles kant (med den over og den under)


i 1D har man ingen sidekant, det er et punkt
men poenget er at i veldig høye dimensjoner så kan tilingene bli mer eksotiske, og et utrykk for det er at du ikke trenger å ha en full kant med noen som helst annen tile
dimensjon > 7

%!   
-> når det er større en 1 har vi flere muligheter
-> MEN en begrensing legges av Kellers theorem
}


%? ——————————————————————————————————  Info  ——————————————————————————————————
%? Timeline. Keller's theorem and conjecture are both from Keller's paper from 1930. Although, they are never stated as "Keller's theorem"  or "Keller's Conjecture" until after. At least, that is my Thought. I think the first time these names appears is in a paper by his student Perron in 1940.
%? ——————————————————————————————————  Short version [Kellers Theorem]  ——————————————————————————————————
%? Kellers theorem was prooved by Keller \cite{ott-heinrichkellerUberLuckenloseErfullung1930} in 1930. A detailed proof appears in Perron \cite{perronUeberLueckenloseAusfuellung1940} satz 9 in chapter 3 on page 11. Source for this is \cite[p. 8]{iosevichSpectralTilingProperties1998}. Furthermore, Perron writes in the intro of his paper \cite{perronUeberLueckenloseAusfuellung1940} that Keller proved his theorem and will do this in chapter 3 while adding some more info to the problem (blue). Additionally, it is important to mention that in Remark 1.2 in \cite{iosevichSpectralTilingProperties1998}, they present Theorem 1.1, which is essentially my master, and some valuable insight into the importance of this, as well as on the topic of exotic tilings.
%? ——————————————————————————————————  Short version [Kellers Conjecture]  ——————————————————————————————————
%? In their paper \cite[p. 2]{iosevichSpectralTilingProperties1998}, they briefly introduce the problem's history. They state, "Keller, while working on Minkowski's conjecture, made the stronger conjecture that one could omit the lattice assumption in Minkowski's conjecture." Furthermore, Perron, in the intro of his paper \cite{perronUeberLueckenloseAusfuellung1940}, discusses this further. The most interesting part is that Keller later doubted his conjecture. He also states that Keller only sketched the proof and that it was hard to follow for dimensions 5 and 6. Worth mentioning that Perron proves this for $n \leq 6$. Furthermore, The paper on the resolution of Keller's conjecture \cite{brakensiekResolutionKellerConjecture2020} states that the conjecture originates from \cite{kellerUberLuckenloseErfullung1930}. They also write that Perron proved this over two papers (the one above and in a paper I have downloaded but not cited) and that Minkowski's conjecture was proved in 1942 by Hajós (no citation here).
%? ——————————————————————————————————  Sigrid Note  ——————————————————————————————————
%? The existence of exotic tilings is perhaps surprising in light of Keller's theorem. If anything, Sigrid would say that the conjecture hints at the opposite. That such exotic tilings do NOT exist. That is to say, Keller made the thoughts that led to his conjecture after he had proved his theorem (or while doing it). He also later doubted this (see intro in Perron). Sigrid emphasizes that if one can show the integer difference independent of the lattice assumption (and dimension), then it is reasonable to assume that one can strengthen the Minkowski conjecture into Keller's conjecture. By lattice assumption, we mean that the integer difference exists whether it is a lattice or not. It is important to remark that the conjecture does not support that exotic tilings should exist. Rather, the opposite. It indicates some rigidity. However, as we know it to be untrue for $d>7$, we have that these exotic tilings appear. 

%* Hvis jeg kan vise at jeg alltid har denne heltalls-differansen så er det naturlig å anta at jeg kan styrke denne mikovsky conjecturen.
%* Dette er fordi  han viser theoremet uavhengig av om det er en lattice eller ikke

%* Derfor blir følgende er ikke helt riktig:
%* "One indicator of this characteristic follows from Keller's conjecture \cite{ott-heinrichkellerUberLuckenloseErfullung1930}."


%? Placeholder Keller's theorem

%* Transition between Keller theorem and conjecture.
%* start
- As we have yet to present a formal definition of a tiling set, a formal definition will follow after this explanation/comment/note/discussion on exotic tilings. 
- After this informal discussion on exotic tilings, the reader should note that we will formally define what we mean by a \emph{tiling set}. 
- As we have yet to present a formal \namecref{def:tiling} of a tiling set (\cref{def:tiling}), the reader should note that this will follow after this informal discussion on exotic tilings. 
- Note that after this informal discussion on exotic tiling, we will formally define what we mean by a \emph{tiling set}. 

%* slutt
- Additionally, the same paper by Keller also conjectured the following. 
- Following his theorem, Keller also conjectured the following. 
- In the same paper, Keller conjectured the following. 
- Keller conjectured that



%? Placeholder Keller's conjecture
%* Known to be true... d<=7, but in fact false when d>7...
- However, after it was recently proven true in dimension seven, we now know the conjecture to be false for all $d>7$ \cite{brakensiekResolutionKellerConjecture2020}. 

- If the dimension is two, the squares share an edge; if the dimension is three, they share a face. It is important to remark/emphasize that the conjecture indicates a rigidity in the tiling of the space/$\R^d$. It does not indicate the existence of exotic tilings. However, after it was recently proven true in dimension seven, we now know the conjecture to be false for all $d>7$ \cite{brakensiekResolutionKellerConjecture2020}.

%? placeholder conjecture solution theorem
- Gammel OC: 
- It is precisely in the higher dimensions where Keller's conjecture fails that we get the tilings we consider highly exotic and counterintuitive.

- It is these higher dimensional tilings we \SigridChange{consider to be/categorize as /classify as} highly exotic and counterintuitive.
- It is precisely where Keller's conjecture fails that we get the tilings \SigridChange{that are both/we consider to be/categorize/classify as} highly exotic and counterintuitive. %both


- Nye OC:
- It is precisely the tilings that counterexample Keller's conjecture, that is, the ones that do not have shared faces, that we intuitively understand to be exotic tilings. 

- By exotic and counterintuitive tilings, we mean precisely the higher dimensional tilings that counterexample Keller's conjecture. 


- It is precisely in the higher dimensions where Keller's conjecture fails that we get the tilings we intuitively understand 

It is precisely these BLABLABLA that we get tilings that counterexample Keller's conjecture — and this is what we mean by exotic. 

This is not a precise definition

This is the intuitive definition of exotic tiling. 

By exotic tiling, we mean essentially, or morally, the ones that counterexample to Keller's conjecture, I.e., the ones that do not have shared faces. 

This is what we intuitively define/understand as a  


%* Flere setninger på hva vi mener med 
- This conjecture does not support the fact that exotic tilings could exist, as it indicates a rigidity in tiling the space. However, it is now known to be false for all dimensions greater than seven after being proven true in dimension seven.

- It is the higher dimensional tilings in which two cubes do not share an entire $(d-1)$-dimensional face we classify as being exotic and counterintuitive.
- It is these higher dimensional tilings in which the property of two cubes not sharing an entire $(d-1)$-dimensional face that makes so the tilings are considered highly exotic and counterintuitive. 
- These higher dimensional tilings in which the property of two cubes not sharing an entire $(d-1)$-dimensional face make the tilings highly exotic and counterintuitive. 
- as we now know the conjecture to be false for all dimensions greater than seven, we have that these exotic tilings do exist. 




%* Translation from German
derimot så er det for n>3 ikke nødvendigivs slik at en hver kube med en nabo en hel (n-1) dimensjonal side til felles har 
On the other hand, it is for $d\geq 3$ not necessarily true that every one of the cubes shares a whole $(n-1)$-dimensional face with all of its neighbors.

%* Alternative remarks on Keller's conjecture
It is interesting that Keller himself later doubted whether the conjecture he made was actually true. Furthermore, Keller also wrote a paper on his conjecture, claiming to solve it for $d\geq 6$. However, Perron later described Keller's solution for $d=5$ and $d=6$ as "quite incomprehensible to me" and "very sketchy" in the literal sense of the word \cite{perronUeberLueckenloseAusfuellung1940}

- It is interesting that Keller himself later doubted whether his conjecture was actually true.