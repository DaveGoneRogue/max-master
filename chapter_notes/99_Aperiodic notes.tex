
Winston aperiodic notes

Just to clarify: a tiling is periodic if there exists a non-trivial translation $\tau$ such that $\tau(\Lambda)=\Lambda$ - as sets 
(obviously, if I colored a tile and applied $\tau$, it would move to a different place, but the point is that the whole shape shouldn't change). 
If no such translation exists, the tiling is aperiodic. So, to prove a tiling is aperiodic, one need to show that if $\tau(\Lambda)=\Lambda$
then the only possibility for $\tau$ is the identity. 

Thus what we are doing in the proof is taking a translation and comparing the image of $\Lambda$ under this translation with the original set. 
So it makes sense to talk about the intersection of the shifted 2*2*2 cube with the defected parts. 

%* Mail 2:
You take the tiling, translate the whole thing, and compare this image with the original. 
I don't understand the bit about matching the coordinates. You could translate by an irrational shift or anything less than 1, 
and this would give a different set. The point is, to find a periodic tiling you need to demonstrate the existence of some translation 
that gives you the same set. If you can't find any such translation the tiling must be aperiodic. 

Naturally/intuitively, the translation needs to be an integer shift.  
To prove this you start with a 2*2*2 block, colour it, this will move when you translate the whole tiling. 
Now compare the translation with the original tiling. The block can hit the shifted columns, but if so, it can only intersect in at 
most four cubes of these columns. Since we are assuming the translation preserves the tiling, the other four cubes of the block must 
match exactly with cubes in non-shifted columns. Hence the translation must be an integer shift; i.e. from Z^3    
%* ————————–




%* ?????
We note the distinction between non-periodic and aperiodic. 
Every aperiodic tiles non-periodic; however, not all non-periodic tilings can also tile aperiodically. 



Non-periodic tiling means that there is no period parallelogram.  
Worth emphasizing is the fact that we lack a method for determining whether a tile is the prototile of a monohedral tiling. 
Also, one can have tilings with no periods at all. However, it is very different "to ask" for tiles that are aperiodic, i.e., 
for tiles that can tile but only in a way that admits no periods. In fact, the answer to this question is not known if we insist that only 
translations are allowed. (One has allowed translation etc., see more on this below)



It is these three motions that in literature often exhibit interesting behavior \cite{kolountzakisTilingsTranslation2010}. 
This latter is the last one of the \emph{group of rigid motions} of the Euclidean group one usually considers in tilings. %* Fant bare wikipedia kilde her
Last, in addition to translation and rotation, one can have reflections \cite{kolountzakisTilingsTranslation2010}. 
An example of this can be seen in \cref{fig:tiling_four}, where a reflected tile is also used.

%* Important distinction between reflection symmetry and tiling by reflection (AND translation!)
The first can clearly be found in the Penrose figure with reflection LINES passing through the middle. 
Tilings by reflection use a reflected variant of the tile in addition to the "normal" tile AND then use translation to cover the surface. 
Sometimes one also needs to rotate in addition in order to cover the entire surface. 
Think about the examples on page 35/36 in \cite{grunbaumTilingsPatterns1987}


Note that one would be hard-pressed to find a reflection line in the aperiodic tiling of \cref{fig:tiling_two}; however, reflection lines exist for 
individual fragments of the tiling. DETTE ER FEIL! Fant refleksjonslinje på figuren vist.

The full group of rigid motions is a subgroup of the Euclidian group https://en.wikipedia.org/wiki/Euclidean_group
translation, rotation $\subset$ translation, rotation, reflection

%* KAN IKKE FINNE I lizards, jo hvis man reflekterer to ganger en vertikal og en horisontal, og så translerer
https://en.wikipedia.org/wiki/List_of_aperiodic_sets_of_tiles
The tilings obtained from an aperiodic set of tiles are often called aperiodic tilings, though strictly speaking, it is the tiles themselves 
that are aperiodic. (The tiling itself is said to be "nonperiodic.")