% Better comment extension for Vscode colors these comments differently
% Normal comment color
% * Important information
% ! ALERT
% ? Question
% TODO stuff to do
% // This is strikethrough



MY NOTEPAD, this is not imported anywhere


%* Failed attempt to get the alignment right. \phantom{text} to create invisible eq. with alignment did not work
%* Must use the widest equation on both sides of the =, but that did not work://
Consider first $\lambda_1, \lambda_1' \in \Lambda_1$. If $\lambda_1 \neq \lambda_1'$, which results in $\lambda_2\in \lambfunc$ and $\lambda_2'\in \lambfuncGen{\lambda_1'}$, we would have that the inner integral will be zero because $\brac{\Omega_1, \Lambda_1}$ is a spectral pair. Conversely, if $\lambda_1 = \lambda_1'$ the resulting integral will factor as % THE inner integral will be equal to $\mes{\Omega_1}$, and the resulting integral will be
\begin{align*}
    &= \mes{\Omega_1} \int_{\Omega_2} e^{2\pi i  \brac{\lambda_2- \lambda_2'}t_2} dt_2.
\end{align*}
Now, if one assumes $\lambda_2 \neq \lambda_2'$, then as both $\lambda_2, \lambda_2' \in \lambfunc$ the integral will be zero from the fact that $\brac{\Omega_2, \lambfunc}$ is a spectral pair. However, if $\lambda_2 = \lambda_2'$, observe that we have the case where $\lambda = \lambda'$, and
\begin{align*}
    %\braa{e_\lambda, e_\lambda}_{L^2\brac{\Omega_1 \times \Omega_2}}
    %&= \int_{\Omega_1} \int_{\Omega_2} e_{\lambda}(t) \overline{e_{\lambda}(t)} dt_2 dt_1
    %=\int_{\Omega_1} \int_{\Omega_2} |e^{2 \pi i \braa{\lambda, t}}|^2 dt_2 dt_1
    %= \int_{\Omega_1} \int_{\Omega_2} |1|^2 dt_2 dt_1\\
    %= \mes{\Omega_2}\mes{\Omega_1} \neq 0  %? Either commentet, or not commented solution
    &= \mes{\Omega_1}\mes{\Omega_2} \neq 0.
\end{align*}
%* ———————


%* Attempting to prove Keller's theorem using my tiling definition
%* Max tiling attempts:
\SigridComment{Must now argue that all elements in this set still equal one almost everywhere. }\\
maybe some argument on the fact that 
\begin{align*}
    \sum_{\gamma \in \Lambda} \indicator{I^d}{(x+\alpha) - \gamma} = 1 a.e 
\end{align*}
i.e considering all shifted $x$'s, and then defining $y=x+\alpha$, to yield $\sum_{\gamma \in \Lambda} \indicator{I^d}{y - \gamma} = 1$ a.e. $y\in \R^d$ or some $y\in \R^d\setminus \braqMed{something}$. 

Or maybe some argument that lets us consider points $x-\alpha$, which would yield
\begin{align*}
    \sum_{\gamma \in \Lambda} \indicator{I^d}{(x-\alpha) - \gamma + \alpha} = \sum_{\gamma \in \Lambda} \indicator{I^d}{x - \gamma}
\end{align*}
which we know is equal to $1$ almost everywhere since $\Lambda$ is a tiling set. 
%* Interlude
%* Max tiling attempt 1
To show that $\mathbf{S}$ is a tiling set, we must show that
\begin{equation*}
    \sum_{\gamma \in \Lambda} \indicator{I^d}{x-s(\gamma)} = 1, \quad a.e. \text{\space\space} x \in \R^d.
\end{equation*}
Since $\Lambda$ is a tiling set, we already know that
\begin{equation*}
    \sum_{\gamma \in \Lambda} \indicator{I^d}{x-\gamma} = 1, \quad a.e. \text{\space\space} x \in \R^d.
\end{equation*}
For some $\gamma\in \Lambda$ we know that either $\gamma_1-\lambda_1 \in \Z$ or $\gamma_1-\lambda_1 \not\in \Z$. In the latter case, observe that since
$\indicator{I^d}{x-\gamma} = 1$ then this implies that $\indicator{I^d}{x-s(\gamma)} = 1$ since $s(\gamma) = \gamma$. In the first case, the argument is not as straightforward. Observe that we have
\begin{align*}
    \indicator{I^d}{x-s(\gamma)} &= \indicator{I^d}{x- (\gamma-\alpha)}\\
    &= \indicator{I^d}{x- (\gamma_1-\alpha_1,\gamma_2,\dots,\gamma_d)}\\
    &= \indicator{I^d}{x_1-\gamma_1+\alpha_1,x_2- \gamma_2,\dots,x_d-\gamma_d}\\
\end{align*}
%* Interlude
%* Max tiling attempt 2
To show that $\mathbf{S}$ is a tiling set, we must show that
\begin{equation*}
    \sum_{\gamma \in \mathbf{S}} \indicator{I^d}{x-\gamma} = 1, \quad a.e. \text{\space\space} x \in \R^d.
\end{equation*}
Since $\Lambda$ is a tiling set, we already know that
\begin{equation*}
    \sum_{\gamma \in \Lambda} \indicator{I^d}{x-\gamma} = 1, \quad a.e. \text{\space\space} x \in \R^d.
\end{equation*}
Consider two subsets $A, B \subset S$ defined such that 
\begin{align*}
    S\setminus A =& \braqMed{s(\gamma) = \gamma : \gamma \in \Lambda}\\
    S\setminus B =& \braqMed{s(\gamma) = \gamma - \alpha : \gamma \in \Lambda}
\end{align*}
\begin{align*}
    \sum_{\gamma \in \mathbf{S}} \indicator{I^d}{x-\gamma} = \sum_{\gamma \in S\setminus A} \indicator{I^d}{x-\gamma} +\sum_{\gamma \in S\setminus B} \indicator{I^d}{x-\gamma}
\end{align*}
Since $s(\gamma) = \gamma$ for all elements in $S\setminus A$, the first sum equals $1$ almost everywhere. For the second sum, we have
\begin{align*}
    \sum_{\gamma \in S\setminus B} \indicator{I^d}{x-\gamma} = \sum_{\gamma \in \Lambda} \indicator{I^d}{x-\gamma + \alpha}
\end{align*}

% maybe this can come in handy
\begin{equation*}
    x\in I^d + \gamma - \alpha \Longleftrightarrow x-\alpha \in I^d + \gamma,  \quad \forall x\in something
\end{equation*}
%* ———————



%* From the proof of both classifications of spectra 2d.
\SigridComment{Sigrid wanted to consider cutting this because I was saying the same two times. I have now deleted the last statement and changed the wording. Would you say it is better now, or would you rather remove the whole equation as well}

we get that $\brac{\lambda-\lambda'} \notin \Zstroke_{I^2}$ which contradicts \labelcref{eq:inclusion_dim_2}. Thus, $\lambda_1'$ must be an integer for any element $\lambda'\in \Lambda$ as this implies
\begin{align*}  %*, which corresponds to the fact that the "x-axis values" distribution is an integer! we disregard the other values (lambda_2) completely
    \brac{\underbrace{\lambda_1}_{\in \intnozero}-\underbrace{\lambda_1'}_{\in \Z}} \in \intnozero. %* this is the one that is correct
    %\quad \text{and} \quad
    %\brac{\underbrace{\lambda_2}_{\notin \Z}-\underbrace{\lambda_2'}_{\notin \Z}} \notin \intnozero, %* denne er beviselig feil 3.5 - 0.5 = 3 som er helt klart et heltall
\end{align*}
%and that $\brac{\lambda-\lambda'} \in \Zstroke_{I^2}$. 
%* ———————




%* 
Til bruk for plotsene
When representing this in a figure, one should note the similarity to a cross product of spaces such as $\Z \times \Z$ or $\R \times \R$. Most important in the construction of spectra presented \cref{thrm:construction_spectra} is that it opens up the possibility of having a different discrete set at each $\lambda_1$ value. This newfound flexibility will be discussed after the proof of the \namecref{thrm:construction_spectra}.
%? VURDERE. Før "When representing" ha: Note that we would still have that the zero-element is the only element that is orthogonal to both sets.



% * In \cite{jorgensenSpectralPairsCartesian2001}, Steen P. and Palle E. presented a method for the construction of spectral pairs in higher dimensions. It is a recursive technique in what they dub "a cross-product construction" using "factors" in lower dimensions and applies for any two spectral pairs $\brac{\Omega_i,\Lambda_i}$ where $i=1,2$ and in arbitrary dimensions $d_1$ and $d_2$. The technique is presented in the following theorem



%* Second construction of the "cross-product" construction again. 
\begin{example}\label{exmp:second_construction}
    Let $\brac{\Omega_1,\Lambda_1} = \brac{I^2, \Lambda}$, where $\Lambda$ is as given in \labelcref{eq:first_construction}. If we again let 
    $\brac{\Omega_2,\Lambda(\lambda_1)} = \brac{I, \Z}$ for all $\lambda_1 \in \Lambda_1$ as in the previous \namecref{exmp:first_construction}, it then also follows from \cref{thrm:construction_spectra} that $\brac{I^2 \times I, \Lambda'}$ is a spectral pair in $2+1=3$ dimensions where 
    \begin{equation*}
        \Lambda'=\braq{\brac{\lambda_1,\lambda_2}: \lambda_1 \in \Lambda, \lambda_2 \in \Z }.
    \end{equation*}
    Similarly, we can also show that $\brac{I^2 \times I^2, \Lambda''}$ is a spectral pair in $2+2=4$ dimensions with
    \begin{equation*}
        \Lambda''=\braq{\brac{\lambda_1,\lambda_2}: \lambda_1 \in \Lambda, \lambda_2 \in \Lambda} \qedhere
    \end{equation*}
\end{example}



%* Old definition of orthogonality
% We consider orthogonal sets with a focus on countable orthogonal sequences. 
Let $H$ be a hilbert space, and $\left\{ e_{n} \right\}_{n\in \mathbb{N}}$ denote a sequence of orthonormal vectors in $H$. Then the closed span, $\overline{\operatorname{span}}\left( \left\{ e_{n} \right\}_{n\in \mathbb{N}} \right)$, is a closed subspace of $H$. In connection with this, we can give an explicit formula for the orthogonal projection of a vector onto $\overline{\operatorname{span}}\left( \left\{ e_{n} \right\}_{n\in \mathbb{N}} \right)$. This is presented in the following theorem \cite[p.~210]{heilMetricsNormsInner2018}.
%
%
\begin{theorem}\label{thrm:orthog_proj_formula_and_facts}
    If $H$ is a Hilbert space and if $\left\{ e_{n} \right\}_{n\in \mathbb{N}}$ is an orthonormal sequence in $H$,  then the following statements hold:
    \begin{enumerate}[label=(\alph*)]
        \item \label{eq:opfaf_a} Bessel's inequality holds for all $x \in H$. That is, 
        \begin{equation}
            \sum_{n=1}^{\infty} \left| \langle x, e_n \rangle \right|^2 \leq \| x\|^2 
        \end{equation}
        
        \item \label{eq:opfaf:b} Let $c_n$ be scalars. If the series $$ x=\sum_{n=1}^\infty c_n e_n.$$ converges in the norm of $H$, then the scalars $c_n= \langle x, e_n\rangle$ for each $n \in \mathbb{N}$.
        
        \item \label{eq:opfaf_c} The following equivalence: 
        \begin{equation}
            \sum_{n=1}^{\infty} c_n e_n \text{ converges in norm of } H \Longleftrightarrow \sum_{n=1}^{\infty} \left| c_n \right|^2 < \infty.
        \end{equation}
        In this case, the series $\sum_{n=1}^{\infty} c_n e_n$ converges unconditionally, meaning that it converges regardless of the ordering of the index set.
        
        \item \label{eq:opfaf_d} If $x \in H$, then 
        \begin{equation} 
            p= \sum_{n=1}^{\infty} \langle x, e_n \rangle e_n, 
        \end{equation} 
        is the orthogonal projection of $x$ onto  $\overline{\operatorname{span}} \left( \left\{ e_{n} \right\}_{n\in \mathbb{N}} \right) $, with
        \begin{equation}
            \| p\|^2 = \sum_{n=1}^{\infty} \left| \langle x,e_n \rangle \right|^2
        \end{equation}
        
        \item \label{eq:opfaf_e} If $x \in H$, then 
        \begin{equation}
            x\in \overline{\operatorname{span}} \left( \left\{ e_{n} \right\}_{n\in \mathbb{N}} \right) \Longleftrightarrow x=\sum_{n=1}^{\infty} \langle x, e_n \rangle e_n \Longleftrightarrow \| x\|^2 = \sum_{n=1}^{\infty} \left| \langle x,e_n\rangle \right|^2
        \end{equation}
    \end{enumerate}
\end{theorem}
%
%
% ! Max Kommentar
\textcolor{red}{poof}\\
Intro, ONB
%
%
\begin{theorem}\label{thrm:orthonormal_equivalences}
    If $H$ is a Hilbert space and if $\left\{ e_{n} \right\}_{n\in \mathbb{N}}$ is an orthonormal sequence in $H$,  then the following are equivalent:
    \begin{enumerate}[label=(\alph*)]
        \item \label{eq:oe_a} $\left\{ e_{n} \right\}_{n\in \mathbb{N}}$ is complete. That is, $\overline{\operatorname{span}} \left( \left\{ e_{n} \right\}_{n\in \mathbb{N}} \right) = H$
        
        \item \label{eq:oe_b} $\left\{ e_{n} \right\}_{n\in \mathbb{N}}$ is a \emph{Schauder basis} for $H$. That is, for each $x\in H$ there exists a unique sequence  $(c_n)_{n\in\mathbb{N}}$ of scalars such that $x = \sum c_n x_n$.
        
        \item \label{eq:oe_c} If $x\in H$, then the following series converges in the norm of $H$. \begin{equation} \label{eq:orthonormal_equivalences_3} x=\sum_{n=1}^\infty \langle x,e_n\rangle e_n, \end{equation}
        
        \item \label{eq:oe_d} Plancherel's equality holds for all $x \in H$. That is, $$ \|x \|^2 = \sum_{n=1}^\infty \left| \langle x,e_n \rangle \right|^2.$$
        
        \item \label{eq:oe_e} Parseval's equality holds for all $x,y \in H$. That is, $$ \langle x,y \rangle = \sum_{n=1}^\infty \langle x,e_n \rangle \langle e_n,y \rangle $$
    \end{enumerate}
\end{theorem}

Note that a sequence that satisfies any of the equivalent conditions in \cref{thrm:orthonormal_equivalences} is an orthonormal basis. We form the following definition.




\begin{definition}\label{def:ONB}
    A countable infinite orthonormal sequence $\left\{ e_{n} \right\}_{n\in \mathbb{N}}$ that is complete in a Hilbert space $H$ is called an \emph{orthonormal basis} for $H$.
\end{definition}

\begin{definition}\label{def:ONB}
    Let $H$ be a Hilbert space. A countable infinite orthonormal sequence $\left\{ e_{n} \right\}_{n\in \mathbb{N}}$ that is complete in a Hilbert space $H$ is called an \emph{orthonormal basis} for $H$.
\end{definition}




