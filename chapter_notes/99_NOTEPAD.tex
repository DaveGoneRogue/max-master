% Better comment extension for Vscode colors these comments differently
% Normal comment color
% * Important information
% ! ALERT
% ? Question
% TODO stuff to do
% // This is strikethrough


MY NOTEPAD, this is not imported anywhere

%-----------
Heltall er gir flislegging gir veldig mening gitt kelles dimensjonsuavhengige theorem (to ulike lambdaer, heltallsdistanse mellom!)
Samme theorem gir også at det ikke finnes andre flislegginger for enhetskuben enn akkurat heltallene

For spektra har vi lignende resulater, og der følger det egentlig at det er heltallene som gjør at eksponentialfunksjonen er lik én
og når vi da har integrert og har denne brøken, så vil vi da få 1-1 = 0 som beviser ortogonalitet, som er selve KRAVET for å være et spektra
(to ulike lambdaer, heltallsdistanse melleom!). 


linken mellom flislegging og spektra ligger i linken mellom kellers vanlige resultat og kellers spectral resulat

Derfor, ingen andre spektra en heltallene

%-----------
TANKE, språk på om vi ser på ONB, spectra, etc.

%-----------
Also, spectral sets are a rarity, as very few sets satisfy all the conditions. 
Even classifying all spectra for certain sets is not a trivial case. which is why the spectral conjecture of Fuglede is so important. 


%-----------
FIN setning: Showing that an orthonormal basis cannot be a strict subset of another orthonormal basis for the same space. 


%-----------
Tilings gammelt

%* generelt
%* denne introen til tilings kan vurderes å droppes for T(\Lambda) ville da vært en tiling OG IKKE et tiling set
To begin, given a subset $\Omega$ of $\R^d$ and a discrete set $\Lambda$ of $\R^d$, we denote by $T(\Lambda)$ the set of translates 
\begin{equation*}
    T(\Lambda) = \braq{\Omega+\lambda : \lambda\in \Lambda},
\end{equation*}
where $\Omega + \lambda$ is the translate of $\Omega$ by the vector $\lambda$. That is the set
\begin{equation*}
    \Omega + \lambda = \braq{\omega + \lambda : \omega \in \Omega}.
\end{equation*}
Rather than defining tiling in terms of non-overlapping and covering of the space, we will instead define tiling in the Fourier domain, using little of our geometric intuition \cite{kolountzakisTilingsTranslation2010} \cite{kolountzakisStudyTranslationalTiling2003}. 
%* ———————


%* Old tiling definition in two variations
\begin{definition}[Tiling set]
    Let $\Omega \subset \R^d$ be a subset with nonzero measure, and consider a set $\Lambda \subset \R^d$. If $T(\Lambda)$ covers $\R^d$ up to measure zero, and if all intersections 
    \begin{equation*}  %* NON OVERLAPPING
        (\Omega+\lambda) \cap (\Omega+\lambda')
    \end{equation*}
    for $\lambda\neq \lambda'$ in $\Lambda$ have measure zero, then $\Omega$ is called a \emph{tile}, and $\Lambda$ is called a \emph{tiling set} for $\Omega$. We say that $(\Omega, \Lambda)$ is a \emph{tiling pair}. 
\end{definition}
\begin{definition}  %* Sigrid would not want to use the above rather than this one
    Let $\Omega \subset \R^d$ be a subset with nonzero measure, and consider a set $\Lambda \subseteq \R^d$. If the following two conditions are satisfied, then $\Omega$ is called a \emph{tile}, and $\Lambda$ is called a \emph{tiling set} for $\Omega$. We say that $(\Omega, \Lambda)$ is a \emph{tiling pair}. 
    \begin{itemize}
        \item If $T(\Lambda)$ cover $\R^d$ up to measure zero. That is,  %* Cover the whole space
        \begin{equation*}
            \bigcup_{\lambda \in \Lambda} (\Omega + \lambda) = \R^d
        \end{equation*}
        \item If all intersections of $(\Omega+\lambda) \cap (\Omega+\lambda')$ for $\lambda\neq \lambda'$ in $\Lambda$ have measure zero. %* Mutually NON-OVERLAPPING 
    \end{itemize}
\end{definition}
%* ———————


%* Old proof of dimension one case
In dimension one, the unit cube is simply the unit interval $I=\bras{0,1}$.
\begin{theorem}  %* Bruker alle elementene i T for å flytte på I, da dekker vi hele $\R$.  
    Let $\Omega = I$. If $\Lambda=\Z$, then $\Lambda$ is a tiling set for $I$.
\end{theorem}

\begin{proof}
    It is clear that
    \begin{align*}
        %\bigcup_{\lambda\in \Z} (I + \lambda) &= \bigcup_{\lambda\in \Z} \braq{\omega + \lambda : \omega \in I}\\
        \bigcup_{\lambda\in \Z} (I + \lambda) &= \dots (I-1) \cup (I-0) \cup (I+1) \dots\\ 
        &= \dots [-1,0] \cup [0,1] \cup [1,2] \dots\\
        &= \R
    \end{align*}
    
    and that  %* consider deleting this
    \begin{align*}  %* consider deleting this
        \mesMed{\R \setminus \bigcup_{\lambda\in \Z} (I + \lambda) } = \mesMed{\emptyset} = 0.
    \end{align*}
    This shows that the set of translates $T(\Lambda)$ covers $\R$. % up to measure zero. 
    
    Now take $\lambda,\lambda' \in \Z$. %If $\lambda = \lambda'$ we have that
    %\begin{equation*}
    %    \mes{(I+\lambda) \cap (I+\lambda')} = \mes{(I+\lambda)} = (1+\lambda) - (0+\lambda) = 1.
    %\end{equation*}
    %And if $\lambda \neq \lambda'$ we have that 
    If $\lambda \neq \lambda'$ we have that 
    \begin{equation*}
        \mes{(I+\lambda) \cap (I+\lambda')} = \mes{\emptyset} = 0,
    \end{equation*}
    showing that the cubes are non-overlapping for all distinct $\lambda , \lambda' \in \Lambda$. 
\end{proof}
%* ———————


%*  Def tiling from Notion page, which again comes from one of the earliest papers from Sigrid
\begin{definition}[Tiling set]
    Let $\Omega \subset \mathbb{R}^d$ be a subset with nonzero measure, and consider a set $T \subseteq \mathbb{R}^d$. If the set of translates ${}$$\{\Omega+l: l\in T\}$ cover $\mathbb{R^d}$ up to measure zero, and if all intersections $(\Omega+l) \cap (\Omega+l')$  for $l\neq l'$ in $L$ have measure zero, then $\Omega$ is called a \emph{tile}, and $T$ is called a \emph{tiling set} for $\Omega$. We say that $(\Omega, T)$ is a \emph{tiling pair}. 
\end{definition}

In other words, the shifts $\Omega + T$ constitute a \emph{measuredisjoint covering} of $\mathbb{R}^d$, and we can say that $\Omega$ \emph{tiles} $\mathbb{R}^d$ \emph{by translation}, or that $\Omega+T$ is a \emph{tiling} of $\mathbb{R}^d$.
%* ———————


%* Fuglede
Fuglede's spectral set conjecture. Let $\Omega$ be a set in $\mathbb{R}^d$ with positive and finite Lebesgue measure. Then $\Omega$ is a spectral set if and only if $\Omega$ tiles $\mathbb{R}^d$ by translation.

An equivalent restatement of Fuglede was presented in Jorgen/Pedersen. 

Let $\Omega \subset \mathbb{R}^d$ have positive and finite Lebesgue measure. Then $\Omega$ is a spectral set if, and only if $\Omega$ is a tile, i.e., there exists a set $\Lambda$ so that $(\Omega, \Lambda)$ is a spectral pair if and only if there exists a set $\Lambda^{\prime}$ so that $\left(\Omega, \Lambda^{\prime}\right)$ is a tiling pair.


Jorgen/Pedersen added the following dual conjectures 1.3 and 1.4. after Fuglede.


Conjecture 1.3: Let $\Lambda \subset \mathbb{R}^d$. Then $\Lambda$ is a spectrum if, and only if $\Lambda$ is a tiling set, i.e., there exists a set $\Omega$ so that $(\Omega, \Lambda)$ is a spectral pair if and only if there exists a set $\Omega^{\prime}$ so that $\left(\Omega^{\prime}, \Lambda\right)$ is a tiling pair.

Conjecture 1.4: Let $\Lambda \subset \mathbb{R}^d$. Then $\left(I^d, \Lambda\right)$ is a spectral pair if and only if $\left(I^d, \Lambda\right)$ is a tiling pair.

When $\Omega = I^d$, the connection between tiles and spectrum is more direct than for other examples of sets $\Omega$. As shown in sigrid_note:lagarias_reeds_wang and Spectral and tiling properties of the unit cube_ALEX IOSEVICH AND STEEN PEDERSEN, it is possible to classify all spectra by showing that $\Lambda$ is a spectrum for the unit cube $I^d$, if and only if $I^d$ tiles $\mathbb{R}^d$ by $\Lambda$-translates.
%* ———————


%* Generelt bevis for noe
%* Vet ikke hvor jeg vil ha denne, generalisering av ONB for hele $L^2([0,1]^d)$, som jo stemmer.
%* Ikke vist formelt i teksen, men løst

\begin{lemma}
    The set of exponential functions $\alldexp$ is an orthonormal basis for $L^2(I^d)$
\end{lemma}

\begin{proof}
    Let $x\in \R^d$ and $\lambda, \lambda' \in Z^d$. Note that they are all vectors on the form $x=(x_1,\dots, x_n, \dots, x_d)$.
    %Using vector notation: $x=(x_1,\dots x_d)$, $\lambda=(\lambda_1, \dots, \lambda_d)$, and $\kappa=(\kappa_1, \dots, \kappa_d)$.
    Show that we can rewrite 
    \begin{align*}
        \langle e_{\lambda},e_{\lambda'} \rangle_{L^2(I^d)} &= \int_{[0,1]^d} e^{2\pi i \langle\lambda, x\rangle} e^{-2 \pi i \langle \lambda', x\rangle} dx \\
        &= \int_{[0,1]^d} e^{2\pi i  (\lambda_1x_1 + \dots +\lambda_d x_d)} e^{-2\pi i  (\lambda_1' x_1 + \dots +\lambda_d' x_d)} dx\\
        &= \int_{[0,1]^d} e^{2\pi i  (\lambda_1 -\lambda_1')x_1} \cdots e^{2\pi i  (\lambda_d -\lambda_d')x_d} dx\\
        &= \int_0^1 e^{2\pi i  (\lambda_1- \lambda_1')x_1} \int_0^1 e^{2\pi i  (\lambda_2 - \lambda_2')x_2}  \cdots \int_0^1 e^{2\pi i  (\lambda_d - \lambda_d')x_d} dx_1 dx_2 \dots dx_d \\
        &=\prod_{n=1}^d \int_0^1 e^{2\pi i  (\lambda_n- \lambda_n')x_n} d x_n 
    \end{align*}
    Show the following 
    if $\lambda = \lambda'$ we get that its equal to 1
    if $\lambda \neq \lambda'$, then we must show that for some index $n$ we have that $\lambda_n \neq \lambda_n'$ which is equal to 0, forcing everyting to be 0, i.e integral with index n vanish.
    That is orthonormality.

    Show basis either by generalizing the proof over to d dimensions
    
    or by using stone-weierstrass in the complex version to show the system is first dense in $C_{per}(I^d)$
    and then, since $C_{per}(I^d)$ is dense in $L^2(I^d)$, then the system is dense in $L^2(I^d)$. 

\end{proof}