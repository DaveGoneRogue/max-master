\documentclass[../thesis.tex]{subfiles}
% Separate preamble for this subfile. This preamble is loaded last, so one can override various functions before \begin{document}

% Better comment extension for Vscode colors these comments differently
% Normal comment color
% * Important information
% ! ALERT
% ? Question
% TODO stuff to do
% // This is strikethrough


\begin{document}

\section{Equivalence of tiling sets and spectral sets for the n-cube}
%* Sammenligning tiling sets <--> spectral sets

essensielt samme som \cref{lem:zero_set_jp_1_5}. %* Denne er en omskriving av det som står i paperet


\begin{lemma}[Spectral version of Kellers theorem]\label{lem:zero_set_AiSp}
    %* JP: If $\brac{I^d,T}$ is a tiling pair, then $T-T \subset \Zstroke_{I^d}\cup \braq{0}$.\\
    Let $T\subset \R^d$ be a discrete subset. The pair $\brac{\Omega,T}$ is orthogonal if and only if given any pair $\lambda, \lambda' \in T$, such that $\lambda\neq\lambda'$, there exists a $j\in \braq{1,\dots,d}$ so that $\bral{t_j -t_j' } \in \N$
\end{lemma}


The following \namecref{cor:tiling_pair_implies_orthogonal} follows directly from Keller's theorem (\cref{thrm:keller_tiling}), and \cref{lem:zero_set_AiSp}.
\begin{corollary}\label{cor:tiling_pair_implies_orthogonal}
    If $\brac{I^d, \Lambda}$ is a tiling pair, then $\brac{I^d,T}$ is orthogonal. Meaning that 
    \begin{equation*}
        T-T \subset \Zstroke_{I^d}\cup \braq{0}
    \end{equation*}
\end{corollary}





\end{document}