\documentclass[../thesis.tex]{subfiles}
% Separate preamble for this subfile. This preamble is loaded last, so one can override various functions before \begin{document}

% Better comment extension for Vscode colors these comments differently
% Normal comment color
% * Important information
% ! ALERT
% ? Question
% TODO stuff to do
% // This is strikethrough


\begin{document}  
%* Sammenligning tiling sets <--> spectral sets

%! —————— Villee hatt med korolaret jeg har skrevet opp først, og så argumentert for det. 
%* skiller seg ut på et viktig sted, i lemmaet over bruker vi link mellom Ortogonalitet og tingen med null settet, mens i min lemma, så sier jeg bare at null settet til funksjonen er denne tingen. jeg sier ingeinting om at dette er ekvivalent med et ortogonalitetskrav. Selv i remarken etterpå sier vi bare at å være "blabla" er det samme som "lemma 4.7 + kompletthet". Ikke at lemma 4.7 erstatter ortogonalitetskravet. 
%* Kan fremheve denne biten i teksten ved å bruke en ligning 
%*  Forklar i teksten denne likheten mellom keller spectral og lemma 4.7. mitt zero set lemma for I^D

Recall \cref{thrm:keller_tiling}, Keller's \namecref{thrm:keller_tiling}. We have an equivalent version, or more specifically, an equivalent restatement of \cref{thrm:keller_tiling}, which closely relates to the previously mentioned spectral version of Keller's \namecref{thrm:keller_tiling}, \cref{lem:zero_set_AiSp}. The equivalence of \Cref{thrm:keller_tiling} and the following \cref{thrm:keller_alternative} follows by rewriting the subtraction of two distinct elements in \cref{thrm:keller_tiling} into $(\Lambda-\Lambda)\setminus\braq{0}$, and by observing that $\Zstroke_{I^d}$ from \cref{lem:zero_set_jp_1_5} captures the rest of the requirements.

\begin{theorem}[Keller's theorem reformulation]\label{thrm:keller_alternative}
    If $\brac{I^d,\Lambda}$ is a tiling pair, then 
    \begin{equation*}
        \Lambda - \Lambda \subseteq \Zstroke_{I^d} \cup \braq{0}.
    \end{equation*} 
\end{theorem}

When using the alternative version, the connection to the spectral version of Keller's \namecref{thrm:keller_tiling}, \cref{lem:zero_set_AiSp}, becomes apparent in that Keller's requirement for a cube-tiling corresponds to the exponentials being orthogonal for the same discrete set $\Lambda$.

\begin{corollary}\label{cor:tiling_pair_implies_orthogonal}
    %! The following result shows that any tiling set for the cube is orthogonal. It is a critical step in our proof that any tiling set for the cube must be a spectrum for the cube and should be compared with the spectral version of Keller's theorem. 
    %If $\brac{I^d, \Lambda}$ is a tiling pair, then $\brac{I^d,\Lambda}$ is an orthogonal pair.\\
    If $\brac{I^d, \Lambda}$ is a tiling pair, then $\Lambda$ gives an orthogonal set of exponentials $E(\Lambda)$.
\end{corollary}

Until now, we have only seen that we have exactly the same tiling sets as spectra for $I^d$ in dimension one. We will now see that the same is true for dimension two. Here we have an analogous result to \cref{thrm:class_all_shift_2d}, which we state in its entirety for convenience. It is worth emphasizing that it, in this case, is as a consequence of Keller's \namecref{thrm:keller_tiling} that we have that these tiling sets must also look like they do in \cref{thrm:class_all_shift_2d}.

\begin{theorem}\label{thrm:class_all_tiling_2d}  %* If given a tiling set for I^2, then this is our only two possibilities. 
    The only subsets $\Lambda \subset \R^2$ such that $\Lambda$ is a tiling set for $\Omega = I^2$ belong to one of the two classes
    \begin{equation}\label{eq:2d_all_shift_class_1_tiling}
        \Lambda = \braqMed{
            \begin{pmatrix}
            m + \alpha \\
            n + \beta_m
            \end{pmatrix} : m,n \in  \Z
            }
    \end{equation}
    or
    \begin{equation}\label{eq:2d_all_shift_class_2_tiling}
        \Lambda = \braqMed{
            \begin{pmatrix}
            m + \beta_n\\
            n + \alpha
            \end{pmatrix} : m,n \in  \Z
            }
    \end{equation}
    where $\alpha \in \R$ is fixed and $\brac{\beta_m}_{m\in \Z}$ is an infinite sequence with $\beta_m \in [0,1)$.
\end{theorem}

\begin{proof}
    Let $\Lambda \subset \R^d$. Keller's theorem, \cref{thrm:keller_tiling}, immediately shows that there must be an integer distance between two distinct points $\lambda,\lambda' \in \Lambda$ if $\Lambda$ is to be a tiling set for $I^2$. That gives us that at least one of $\brac{\lambda_1-\lambda_1'}$ or $\brac{\lambda_2-\lambda_2'}$ is in $\intnozero$. That said, as in the proof of \cref{thrm:class_all_shift_2d} while using \cref{thrm:keller_alternative}, we can assume that the zero-element $\brac{0,0}$ is in $\Lambda$, which would result in the same two options for $\lambda_1$ and $\lambda_2$ for all $\lambda\in \Lambda$. That is, at least one of them must be a non-zero integer. Likewise, if we further assume $\lambda\notin \Z^2$, and that $\lambda_1 \in\intnozero$, we have a case identical to the first of the two possibilities in the proof of \cref{thrm:class_all_shift_2d}. If we now follow the exact same arguments from that proof, we can first show that 
    \begin{align*}  %* ————  den mengden til høyre er ALLTID større en Lambda ———— 
        \Lambda \subseteq \braqMed{ \bracMed{\lambda_1, \lambda_2} : \lambda_1 \in \Z, \lambda_2 \in \R},
    \end{align*}
    and then that $\Lambda$ is a subset of \labelcref{eq:2d_all_shift_class_1_tiling}, with $\lambda_1 = m$ and $\lambda_2 = n+\beta_m$. That is,
    %Furthermore, by following the same argument as in the proof of \cref{thrm:class_all_shift_2d}, we now have that $\Lambda$ is, in fact, a subset of \labelcref{eq:2d_all_shift_class_1_tiling}. That is, 
    \begin{equation*}  %* ———— DENNE inklusjonen DERIMOT, der er mengden til høyre POTENSIELT større en Lambda, men vi må ha likhet ————
        %\Lambda \subseteq \underbrace{\braqMed{\begin{pmatrix} m \\ n+\beta_m\end{pmatrix} : m,n \in \Z}}_{A}.
        \Lambda \subseteq \braqMed{\begin{pmatrix} m \\ n+\beta_m\end{pmatrix} : m,n \in \Z}.
    \end{equation*}
    If we now assume that $\Lambda$ is incomplete with respect to the set \labelcref{eq:2d_all_shift_class_1_tiling}, then there would be an arbitrary point $\lambda'$ in \labelcref{eq:2d_all_shift_class_1_tiling} that is not in $\Lambda$. However, now $\Lambda$ is not tiling set anymore. Observe that we would have an entire square, specifically the square at
    \begin{equation*}
        %I^2 + \lambda' = \brasMed{0+\lambda_1,1+\lambda_1}\times \bras{0+\lambda_2,1+\lambda_2} = \bras{0+m,1+m}\times \bras{0+n+\beta_m,1+n+\beta_m}, 
        %I^2 + \lambda' = \brasMed{\bras{0+m}},1+m}\times \bras{0+n+\beta_m,1+n+\beta_m}, 
        I^2 + \lambda' = \brasMed{\bracMed{0+m},\bracMed{1+m}}\times \brasMed{\bracMed{0+n+\beta_m},\bracMed{1+n+\beta_m}}, 
    \end{equation*}
    with no value. As $\mes{I^2 + \lambda'}$ is non-negligible, this violates \cref{def:tiling} where we require the value of $1$ for almost every $x\in \R^2$. Hence $\lambda'\in \Lambda$ for $\Lambda$ to be a tiling set. This shows that $\Lambda$ cannot be a proper subset of \labelcref{eq:2d_all_shift_class_1_tiling}, which means that we must have equality.% if the use of A instead of \labelcref: and $\Lambda$ must, in this first case, be of the form \labelcref{eq:2d_all_shift_class_1_tiling}.
    
    If we now go back a few steps and assume the opposite, that is $\lambda_2 \in \intnozero$; we have a case identical to the second of the two possibilities in the proof of \cref{thrm:class_all_shift_2d}. Again, by following the same arguments, we can show that $\Lambda$ is a subset of \labelcref{eq:2d_all_shift_class_2_tiling}. That is,
    \begin{equation*}  %* ———— OG DENNE inklusjonen, der er mengden til høyre POTENSIELT større en Lambda, men vi må ha likhet ————
        %\Lambda \subseteq \underbrace{\braqMed{\begin{pmatrix} m +\beta_n \\ n\end{pmatrix} : m,n \in \Z}}_{B}.
        \Lambda \subseteq \braqMed{\begin{pmatrix} m +\beta_n \\ n\end{pmatrix} : m,n \in \Z}.
    \end{equation*}
    If we again assume that $\Lambda$ is incomplete with respect to the set in \labelcref{eq:2d_all_shift_class_2_tiling}, we can similarly show that $\Lambda$ cannot be a proper subset of \labelcref{eq:2d_all_shift_class_2_tiling}. We must have equality. To conclude, $\brac{I^2, \Lambda}$ is a tiling pair in the case where $\Lambda$ is given by either \labelcref{eq:2d_all_shift_class_1_tiling} or \labelcref{eq:2d_all_shift_class_2_tiling}, and there are no other tiling sets for the unit cube in dimension $d=2$.
\end{proof}

\begin{remark}
    By using \cref{cor:tiling_pair_implies_orthogonal} and a few more results from the proof of \cref{thrm:class_all_shift_2d}, we could have shortened the above proof of \cref{thrm:class_all_tiling_2d}, especially the opening. This follows from the following observations. The \namecref{cor:tiling_pair_implies_orthogonal} implies that the tiling pair is also orthogonal. In the proof of \cref{thrm:class_all_shift_2d}, we argue from the orthogonality of $E(\Lambda)$, then it must be that $\Lambda$ belong to either of the two classes of \labelcref{eq:2d_all_shift_class_1} or \labelcref{eq:2d_all_shift_class_2}, or in this case \labelcref{eq:2d_all_shift_class_1_tiling} or \labelcref{eq:2d_all_shift_class_2_tiling} to be specific even though they are the same. Nevertheless, this would allow us to skip the first part of our proof of \cref{thrm:class_all_tiling_2d}, leaving us with only showing the final equality using the tiling argument from \cref{def:tiling}.
\end{remark}

It should be stressed that when working with a spectrum, it is due to the connection between the zero-set and orthogonality that the spectrum must look like they do. Similarly, when working with a tiling set, it is due to Keller's theorem. That said, it is interesting that Keller's theorem can be reformulated in a manner that aligns with the concept of the zero-set. In fact, in view of this concept, Kolountzakis showed that Keller's \namecref{thrm:keller_alternative} is a special case of a more general result when he proved \cref{thrm:main_result} as a corollary in \cite{kolountzakisPackingTilingOrthogonality2000}, wherein Keller's \namecref{thrm:keller_alternative} also follows as a consequence.

\SigridChangeTwo{
Now for the equivalence of tiling sets and spectra for the unit cube. Jorgen and Pedersen conjectured in their paper \cite{jorgensenSpectralPairsCartesian2001} what is now a \namecref{thrm:main_result}, \cref{thrm:main_result}, after it was proved in full generality in two independent proofs around the same time by Iosevich and Pedersen \cite{iosevichSpectralTilingProperties1998} and Lagarias, Reeds, and Wang in \cite{lagariasOrthonormalBasesExponentials2000}, which we will discuss further below. We remark that Jorgensen and Pedersen in the aforementioned paper \cite{jorgensenSpectralPairsCartesian2001} also proved the conjecture for $d\leq3$ while additionally classifying all spectra and tiling sets in these cases. We have already seen these classifications for $d=1$ and $d=2$. While the case for $d=3$ is similar to the previous two, it is also more intricate due to the greater flexibility one gets when increasing the dimension. Although Jorgen and Pedersen did not prove the conjecture, they were able to prove it in all dimensions provided that $\Lambda$ is \emph{periodic}, which is the content of the following result presented after the \namecref{def:periodic_set}.}

\begin{definition}(Periodic set)\label{def:periodic_set}
    Let $A\subset\R^d$ be a discrete subset. Then $A$ is \emph{periodic} if there exists a finite set $K\subset\R^d$ and an invertible $d\times d$ matrix $M$ consisting of real entries such that 
    \begin{equation*}
        A = K + M\Z^d. \qedhere
    \end{equation*}
    %? fra periodicity of spectrum in dimension one
    %Definition 1.3. A set S ⊆ Rd is called (fully) periodic if there exists a lattice L ⊆ Rd (a discrete subgroup of Rd with d linearly independent generators; the period lattice) such that S + t = S for all t ∈ L. We call a translation tiling periodic if the set of translations is periodic.
    %? Også def på pdf side 2 i Tiling the line with translates of one tile 
\end{definition}

\begin{theorem}\label{thrm:periodic_equivalence}
    Let $\Lambda= K + L$, where $K\subset \R^d$ is a subset with $\bral{K}$ elements, and $L=M\Z^d$ is the lattice in $\R^d$ generated by the columns of $M$, an invertible $d\times d$ matrix with real entries. If $L\cap(K-K)= \braq{0}$, then the following are equivalent:
    \begin{enumerate}[label=(\roman*), topsep=-3pt]
        \item $\brac{I^d,\Lambda}$ is a spectral pair
        \item $\brac{I^d,\Lambda}$ is a tiling pair  \SigridChange{They have a tiny bit different definition here}
        \item $\bral{K} = \bral{\det(M)}$ \textsc{and} $L+\brac{K-K}\subseteq \Zstroke_{I^d} \cup \braq{0}$
    \end{enumerate}
\end{theorem}

\begin{remark}
    The last statement simplifies the zero-set condition of $\Lambda-\Lambda$ since we now have a lattice $L=M\Z^d$. As \cref{eq:zero_set_orthogonal}  defines the set difference as the set of all differences, subtracting two elements from $L$ gives another element of $L$, and we get $L-L=L$.
\end{remark}

What is more interesting about \cref{thrm:periodic_equivalence} is that it has a strong connection to the proofs where Keller's \namecref{conj:keller_tiling} is disproved. Notably, if the \namecref{conj:keller_tiling} is false for $\R^d$, the tiling in $\R^n$ for some $n\geq d$ that counterexample Keller's \namecref{conj:keller_tiling} has two additional properties \cite{lagariasKellerCubetilingConjecture1992}: First, the centers of all the unit cubes are in $\frac{1}{2}\Z^n$. Secondly, the tiling is periodic and contains the period lattice $2\Z^n$. The latter is the important one because it means that the counterexamples can be periodic. Specifically, in the case where Keller's \namecref{conj:keller_tiling} was disproved for $d\geq10$, they found a periodic set of translations of the form $\Lambda= K + (2\Z)^{10}$, where $\bral{K} = 2^{10} = 1024$ elements. Using \cref{thrm:periodic_equivalence}, we have that the tiling pair where no two tiles share an entire face $\brac{I^{10},\Lambda}$ is also a spectral pair in $\R^{10}$ \cite{jorgensenSpectralPairsCartesian2001}. As Mackey \cite{mackeyCubeTilingDimension2002} improved this result further by finding a set of translations of the form $\Lambda= K + (2\Z)^{8}$ where $\bral{K} = 2^{8} = 256$ elements, we also have that the tiling pair $\brac{I^{8},\Lambda}$ is a spectral pair in $\R^8$ using \cref{thrm:periodic_equivalence}. What is interesting about these spectral pairs is that $\Lambda-\Lambda$ does not contain one of the canonical unit basis vectors for $\R^{10}$ \cite{jorgensenSpectralPairsCartesian2001}, which also is true in $\R^{8}$ with the resolution of the \namecref{conj:keller_tiling} \cite{brakensiekResolutionKellerConjecture2020}. Hence, we know that in all $d\geq8$, periodic tilings sets exist where no two tiles share an entire face, with the added fact that the difference $\Lambda-\Lambda$ does not contain one of the canonical basis vectors of $\R^d$.
\begin{remark}
    \SigridComment{remark on the restatement of Keller's conjecture? }
\end{remark}

This result is important for the previous \cref{ex:cannot_class_all}. In particular, the \namecref{ex:cannot_class_all} presents a method for creating spectra that make $\brac{I^d,\Lambda}$ an $\R^d$-spectral pair by giving an explicit \namecref{eq:cannot_class_all}, \cref{eq:cannot_class_all}, for the set of points in $\Lambda$. Note that there are variations of \labelcref{eq:cannot_class_all} that result from a permutation of the $d$ coordinates; however, the consequential result is that the corresponding combinations given by \labelcref{eq:cannot_class_all} for $d\geq8$, \emph{cannot} catalog \textsc{all} possible spectra that make $\brac{I^d,\Lambda}$ an $\R^d$-spectral pair. Said differently, \labelcref{eq:cannot_class_all} cannot provide all possible spectra for $I^d$. That this is the same dimension where Keller's \namecref{conj:keller_tiling} fails first is no coincidence and follows from the following: Since there exist tilings $\Lambda-\Lambda$ for all $d\geq8$ that do not contain one of the canonical base vectors, then $\Lambda$ is \SigridChange{in a form/of a type/...} not given by \labelcref{eq:cannot_class_all} as one in the \namecref{eq:cannot_class_all} need at least one of the canonical basis vectors.

\SigridComment{Regarding your comment on referencing the non-trivial stuff, see the image in the e-mail.}

We now turn to the discussion of the independent proofs of \cref{thrm:main_result} by Iosevich and Pedersen \cite{iosevichSpectralTilingProperties1998} and Lagarias, Reeds, and Wang in \cite{lagariasOrthonormalBasesExponentials2000}, with further improvements by Kolountzakis in \cite{kolountzakisPackingTilingOrthogonality2000} and Li in \cite{liCharacterizationsSpectraTilings2004}. 

To prove \cref{thrm:main_result}, Iosevich and Pedersen in \cite{iosevichSpectralTilingProperties1998} use an approach based on a geometric argument. This geometric argument is analogous to a lemma used by Perron to prove Keller's \namecref{thrm:keller_tiling} in \cite{perronUeberLueckenloseAusfuellung1940}. In fact, we showed the tiling set component of this geometric argument in the proof of Keller's \namecref{thrm:keller_tiling}, \cref{sec:proof_keller}. The result of the geometric argument is that specific parts of a tiling set or spectrum for the unit cube can be translated independently of the rest and remain a tiling set or spectrum. A simple example of this geometric argument for dimension two is shown in \cref{fig:single_shift_horizontal_tiling,fig:single_shift_vertical_tiling} for tiling sets, or in \cref{fig:single_shift_vertical} for spectra, which show that one can translate a row or column independently of the complement of that row or column. Additionally, when showing both directions of \cref{thrm:main_result}, another central result is Bessel's inequality
\begin{equation*}%\label{eq:bessel}
    \sum\bralMed{\braaMed{f,g_n}}^2 \leq \branMed{f}^2,
\end{equation*}
where $f\in L^2(I^d)$ and $(g_n)$ is an orthonormal family in $L^2(I^d)$. Furthermore, the inequality becomes an equality if and only if the function $f$ is in the closed linear span of $(g_n)$. In the case of the unit cube
\begin{equation*}
    \sum_{\lambda \in \Lambda} \bralMed{\braaMed{e_x,e_\lambda}}^2 \leq 1
\end{equation*}
for any $x\in \R^d$ and $e_\lambda\in E(\Lambda)$, where $E(\Lambda)$ is orthogonal in $L^2(I^d)$. 

\SigridComment{They call lemma 3.3 in IP a sliding lemma. But is it also the geometric argument? The sentence on page 4 is half complete "The geometric argument is based Lemma 3.3, an analogous lemma was used by Perron [Per] in...}




In the independent proof of \cref{thrm:main_result} by Lagarias, Reeds, and Wang in \cite{lagariasOrthonormalBasesExponentials2000}, they first established and proved a more general result before applying this to the unit cube. One of their main results is a criterion for $\Lambda$ being a spectrum for more general sets $\Omega$ in terms of a translational tiling of $\R^d$ by $\Lambda+D$ where $D$ is a \emph{suitable auxiliary set} $D$ in Fourier space. However, the usefulness of their result is also restricted to examples where such a suitable auxiliary set exists. 
\SigridComment{Need to add some more stuff here}


When comparing the proofs, the most notable similarity is that they both make use of Keller's \namecref{thrm:keller_tiling}, \cref{thrm:keller_tiling}, as well as the same underlying terminology, and an argument on an inequality becoming and equality. Other than that, they use entirely different techniques to prove \cref{thrm:main_result}. After the publication of these solutions, there have been a few notable improvements to both proofs. Using a Fourier analytic criterion to characterize tilings by translation, Kolountzakis in \cite{kolountzakisPackingTilingOrthogonality2000} further improved the results from \cite{lagariasOrthonormalBasesExponentials2000}, as well as providing a substantially shorter proof of a result equivalent to their criterion and increasing the class of suitable auxiliary sets $D$. Last, Kolountzakis also showed that one could prove \cref{thrm:main_result} without the use of Keller's \namecref{thrm:keller_tiling}, and as mentioned, generalized it to hold for more general sets using the concepts established by \cite{lagariasOrthonormalBasesExponentials2000}. A notable example in both \cite{lagariasOrthonormalBasesExponentials2000} and \cite{kolountzakisPackingTilingOrthogonality2000} in dimension one is that the set $\Omega\subset \R$ where $\Omega = \brac{0,\frac{1}{2}} \cup  \brac{1,\frac{3}{2}}$ is a non-unit cube where all tiling sets are also spectra and \emph{vice versa}. Furthermore, this set is also an example of a set that has no lattice tiling \cite{lagariasTilingLineTranslates1996}, and hence also no lattice spectra. Lastly, Li in \cite{liCharacterizationsSpectraTilings2004}  provided a cleaner proof of the criterion in \cite{kolountzakisPackingTilingOrthogonality2000} using a simple yet powerful method, which he also showed has applications related to Fuglede's \namecref{conj:fuglede}. \SigridChange{As a closing thought on this thesis,}


As a closing thought on this thesis, ... 

As a closing thought on this thesis, we highlight the important results that the connection between tiling sets and spectra is more direct for the unit cube than for other sets 


the important fact that  

%? Winston quote
%? A main result of this thesis is the following criterion which relates spectra to tilings in the Fourier domain.


% THE MAIN RESULT OF ALL PAPERS:
%!On the other hand, relating spectra to tilings in Fourier space allows them to characterize all spectra for the unit cube geometrically. 

% also
%! the same set $\Lambda\subset\R^d$ is indeed a spectrum and tiling set for $\Omega = I^d$. 
%! Highlight denne direkte linken i chapter 5 med keller. dvs: the connection between tiles and spectrum is more direct for OMEGA=I^d than for other sets due to "NULL sett tingen" and the corresponding result for tilings (Keller). 

%* Word choise for 
%* Cataloging, categorizing, classifying
\end{document}