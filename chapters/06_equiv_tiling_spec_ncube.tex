\documentclass[../thesis.tex]{subfiles}
% Separate preamble for this subfile. This preamble is loaded last, so one can override various functions before \begin{document}

% Better comment extension for Vscode colors these comments differently
% Normal comment color
% * Important information
% ! ALERT
% ? Question
% TODO stuff to do
% // This is strikethrough


\begin{document}  
%* Sammenligning tiling sets <--> spectral sets

%! —————— Villee hatt med korolaret jeg har skrevet opp først, og så argumentert for det. 
%*  Forklar i teksten denne likheten mellom lemma 5.1 (under) og lemma 4.7.
\mycomment{ %* Block comment
\begin{lemma}[Spectral version of Kellers theorem]\label{lem:zero_set_AiSp}
    %* JP: If $\brac{I^d,T}$ is a tiling pair, then $T-T \subset \Zstroke_{I^d}\cup \braq{0}$.\\
    Let $T\subset \R^d$ be a discrete subset. The pair $\brac{\Omega,T}$ is orthogonal if and only if given any pair $\lambda, \lambda' \in T$, such that $\lambda\neq\lambda'$, there exists a $j\in \braq{1,\dots,d}$ so that $\bral{t_j -t_j' } \in \N$
\end{lemma}}

%* skiller seg ut på et viktig sted, i lemmaet over bruker vi link mellom Ortogonalitet og tingen med null settet, mens i min lemma, så sier jeg bare at null settet til funksjonen er denne tingen. jeg sier ingeinting om at dette er ekvivalent med et ortogonalitetskrav. Selv i remarken etterpå sier vi bare at å være "blabla" er det samme som "lemma 4.7 + kompletthet". Ikke at lemma 4.7 erstatter ortogonalitetskravet. 
%* Kan fremheve denne biten i teksten ved å bruke en ligning 

Recall \cref{thrm:keller_tiling}, Keller's theorem. We have an analogous version of the \namecref{thrm:keller_tiling} aptly named \emph{the spectral version of Keller's theorem} \cite{iosevichSpectralTilingProperties1998}, which we have previously introduced as our \cref{lem:zero_set_jp_1_5} with \cref{rem:zero_set_orthogonal}. They are the same, with the distinction being in BLABLABLA. 


Using \cref{thrm:keller_tiling}, and the first part of \cref{rem:zero_set_orthogonal} we have the following \namecref{cor:tiling_pair_implies_orthogonal} from to \cref{thrm:keller_tiling}. \namecref{thrm:keller_tiling}


The following \namecref{cor:tiling_pair_implies_orthogonal} follows directly from Keller's theorem (\cref{thrm:keller_tiling}), and \cref{lem:zero_set_AiSp}.

%! The following result shows that any tiling set for the cube is orthogonal. It is a critical step in our proof that any tiling set for the cube must be a spectrum for the cube and should be compared with the spectral version of Keller's theorem. 
\begin{corollary}\label{cor:tiling_pair_implies_orthogonal}
    If $\brac{I^d, \Lambda}$ is a tiling pair, then $\brac{I^d,T}$ is orthogonal. %Meaning that 
    %\begin{equation*}
    %    T-T \subset \Zstroke_{I^d}\cup \braq{0}
    %\end{equation*}
\end{corollary}
%! —————— END
%! og så sagt at "nå skal vi se at vi får akkurat de samme settene".  

%* "Vi har sett at vi har akkurat de samme tiling mengdene som spektral mengdene for en dimensjon, og nå skal vi se det for to dimensjoner og der har vi et analogt resulat til 4.20 og beviset vil følge de samme linjene. (Nå for tilings!)

Moreover, as a consequence of Keller's theorem, we have that these tiling sets must also look like they do in \cref{thrm:class_all_shift_2d}.

%* This section will be closely related to theorem 4.20 and its proof
\begin{theorem}\label{thrm:class_all_tiling_2d}
    %* If one knows that we have a tiling set for the unit cube in two dimensions, then this is our only two possibilities. 
\end{theorem}

\begin{proof}
    %* similar proof of the one of theorem 4.20, and which considers exactly the same
    %* redusert versjon, når du ser det blir helt likt referert tilbake til theorem 4.20. 
    %* Her vil man bruke keller til dette
\end{proof}

%! Fra 4.20
%! \SigridComment{ekskludert delen om at det også er et tiling set, det er vel hensiktsmessik og splitte disse opp, og komme tilbake til det i neste del, for å så komme med fulle og hele theorem 3.2.} 





% transkribert lydlogg
% Her ville jeg hentet frem "nå skal vi se at en konsekvens av kellers theorem" er at disse tiling setsene de må også nødvendigivs se ut som de gjør i theorem 4.20

% Og da kommer det til å komme et bevis som linger veldig mye på beviset for 4.20

% som tar for seg akkurat det samme

% jeg tenker på et theorem som ikke finnes i denne seksjonen som sier at
% Hvis du vet at du har et tiling set for enhetskuben i to dimensjoner så er dette de to mulighetene vi har. 

% da kommer det noe som til en hvis grad vil speile det som står på side 23 til 25.

%ville kanskje hatt med korolaret jeg har skrevet opp først, og så argumentert for det. 
% og så sagt at "nå skal vi se at vi får akkurat de samme settene".  

% Kun for dimensjon 2 her, dimensjon 1 har vi gjort oss ferdig med, fordi jeg har skrevet om dette i tilingskapittelet. 

% Kan si det muntlig: " Nå har vi allerede sett at d 
% "Vi har sett at vi har akkurat de samme tiling mengdene som spektral mengdene for en dimensjon, og nå skal vi se det for to dimensjoner"
% Der har vi et analogt resulat til 4.20 og beviset vil følge de samme linjene.

% For tilings, men i redusert versjon, der når du ser det blir helt likt referert tilbake til theorem 4.20. 

% Og så vil man da bruke Keller til dette

% ikke ta med lemma 5.1, det er det samme som 4.7 ikke si det samme igjen. Forklar i teksten denne likheten.
\end{document}