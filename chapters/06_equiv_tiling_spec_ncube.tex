\documentclass[../thesis.tex]{subfiles}
% Separate preamble for this subfile. This preamble is loaded last, so one can override various functions before \begin{document}

% Better comment extension for Vscode colors these comments differently
% Normal comment color
% * Important information
% ! ALERT
% ? Question
% TODO stuff to do
% // This is strikethrough


\begin{document}  
%* Sammenligning tiling sets <--> spectral sets

%! —————— Villee hatt med korolaret jeg har skrevet opp først, og så argumentert for det. 
%*  Forklar i teksten denne likheten mellom lemma 5.1 (under) og lemma 4.7.
%\mycomment{ %* Block comment
\begin{lemma}[Spectral version of Kellers theorem]\label{lem:zero_set_AiSp}
    %* JP: If $\brac{I^d,T}$ is a tiling pair, then $T-T \subset \Zstroke_{I^d}\cup \braq{0}$.\\
    Let $T\subset \R^d$ be a discrete subset. The pair $\brac{\Omega,T}$ is orthogonal if and only if given any pair $\lambda, \lambda' \in T$, such that $\lambda\neq\lambda'$, there exists a $j\in \braq{1,\dots,d}$ so that $\bral{t_j -t_j' } \in \N$
\end{lemma}
%}

%* skiller seg ut på et viktig sted, i lemmaet over bruker vi link mellom Ortogonalitet og tingen med null settet, mens i min lemma, så sier jeg bare at null settet til funksjonen er denne tingen. jeg sier ingeinting om at dette er ekvivalent med et ortogonalitetskrav. Selv i remarken etterpå sier vi bare at å være "blabla" er det samme som "lemma 4.7 + kompletthet". Ikke at lemma 4.7 erstatter ortogonalitetskravet. 
%* Kan fremheve denne biten i teksten ved å bruke en ligning 

Recall \cref{thrm:keller_tiling}, Keller's theorem. We have an analogous version of the \namecref{thrm:keller_tiling} aptly named \emph{the spectral version of Keller's theorem} \cite{iosevichSpectralTilingProperties1998}, which we have previously introduced as our \cref{lem:zero_set_jp_1_5} with \cref{rem:zero_set_orthogonal}. They are the same, with the distinction being in BLABLABLA. 


Using \cref{thrm:keller_tiling}, and the first part of \cref{rem:zero_set_orthogonal} we have the following \namecref{cor:tiling_pair_implies_orthogonal} from to \cref{thrm:keller_tiling}. \namecref{thrm:keller_tiling}


The following \namecref{cor:tiling_pair_implies_orthogonal} follows directly from Keller's theorem (\cref{thrm:keller_tiling}), and \cref{lem:zero_set_AiSp}.

%! The following result shows that any tiling set for the cube is orthogonal. It is a critical step in our proof that any tiling set for the cube must be a spectrum for the cube and should be compared with the spectral version of Keller's theorem. 
\begin{corollary}\label{cor:tiling_pair_implies_orthogonal}
    If $\brac{I^d, \Lambda}$ is a tiling pair, then $\brac{I^d,T}$ is orthogonal. %Meaning that 
    %\begin{equation*}
    %    T-T \subset \Zstroke_{I^d}\cup \braq{0}
    %\end{equation*}
\end{corollary}
%! —————— END
%! og så sagt at "nå skal vi se at vi får akkurat de samme settene".  

%* "Vi har sett at vi har akkurat de samme tiling mengdene som spektral mengdene for en dimensjon, og nå skal vi se det for to dimensjoner og der har vi et analogt resulat til 4.20 og beviset vil følge de samme linjene. (Nå for tilings!)

We have now seen that we have exactly the same tiling sets as spectra for $I^d$ in dimension one, and we will now show it for dimension two. Here we have an \SigridChange{analogous/equivalent(?)} result to \cref{thrm:class_all_shift_2d}, which we state in its entirety for convenience. It is worth noting that it is as a consequence of Keller's theorem that we have that these tiling sets must also look like they do in \cref{thrm:class_all_shift_2d}.

%* If one knows that we have a tiling set for the unit cube in two dimensions, then this is our only two possibilities. 
\begin{theorem}\label{thrm:class_all_tiling_2d}
    The only subsets $\Lambda \subset \R^2$ such that $\Lambda$ is a \SigridChange{tiling set/\textsc{tiling set}} for $\Omega = I^2$ belong to one of the two classes
    \begin{equation}\label{eq:2d_all_shift_class_1_tiling}
        \Lambda = \braqMed{
            \begin{pmatrix}
            m + \alpha \\
            n + \beta_m
            \end{pmatrix} : m,n \in  \Z
            }
    \end{equation}
    or
    \begin{equation}\label{eq:2d_all_shift_class_2_tiling}
        \Lambda = \braqMed{
            \begin{pmatrix}
            m + \beta_n\\
            n + \alpha
            \end{pmatrix} : m,n \in  \Z
            }
    \end{equation}
    where $\alpha \in \R$ is fixed and $\brac{\beta_m}_{m\in \Z}$ is an infinite sequence with $\beta_m \in [0,1)$.
    
\end{theorem}

\begin{proof}
    %! kødd
    %* similar proof of the one of theorem 4.20, and which considers exactly the same
    %* redusert versjon, når du ser det blir helt likt referert tilbake til theorem 4.20. 
    %* Her vil man bruke keller til dette
    Now, given an arbitrary point $\lambda' \in \Z +\alpha$, assume that $\lambda'\notin \Lambda$ so that we have a case where $\Lambda \subset \Z+\alpha$. However, by using \cref{def:tiling}, observe now that we have an entire interval, specifically $I+\lambda'$, where $\indicator{I}{x - \lambda'} = 0$. As $\mes{I+\lambda'} = 1 $, this is non-negligible, and we do not achieve
    \begin{equation*}
        \sum_{\lambda \in \Lambda\setminus \braq{\lambda'}} \indicator{I}{x-\lambda} = 1, \text{\space} a.e. \quad x \in \R^d.
    \end{equation*}
    %* As $\mes{I+\lambda'} = 1 $, this is non-negligible, and we do not achieve $\indicator{I}{x-\lambda} = 1$ almost everywhere for $x\in\R^d$ with the set $\Lambda\setminus \braq{\lambda'}$. % In short: That is, we have a non-negligible region with a value not equal to one.
    Hence $\lambda' \in \Lambda$, which shows that $\Lambda = \Z+\alpha$. This concludes the proof that there are no other tiling sets for $I$ in dimension $d=1$. 

    %* Beginning
    % If we now consider a second element $\lambda'\in \Lambda$ with the opposite assumptions for $\lambda_1'$ and $\lambda_2'$ we have exactly the same case 
    % Likewise, if we follow the exact same arguments following the above result in the proof of \cref{thrm:class_all_shift_2d}, we can first show that 

    %Likewise, if we further assume $\lambda\notin \Z^2$, and first take $\lambda_1 \in \intnozero$ we also get that $\lambda_2\notin \Z$. By now, following the exact same arguments following the above assumption from the proof of \cref{thrm:class_all_shift_2d}, we can first show that 

    Let $\Lambda \subset \R^d$. Keller's theorem, \cref{thrm:keller_tiling}, immediately shows that there must be an integer distance between two distinct points $\lambda,\lambda' \in \Lambda$ if $\Lambda$ is to be a tiling set for $I^2$. That gives us that at least one of $\brac{\lambda_1-\lambda_1'}$ or $\brac{\lambda_2-\lambda_2'}$ is in $\intnozero$. As in the proof of \cref{thrm:class_all_shift_2d}, we can assume that the zero-element $\brac{0,0}$ is in $\Lambda$, which would result in the same two options for $\lambda_1$ and $\lambda_2$ for all $\lambda\in \Lambda$. That is, at least one of them must be a non-zero integer. Likewise, if we further assume $\lambda\notin \Z^2$, and that $\lambda_1 \in\intnozero$, we have a case that is identical to the first of the two possibilities in the proof of \cref{thrm:class_all_shift_2d}. If we now follow the exact same arguments from the previous proof, we can first show that 
    \begin{align*}  %* ————  den mengden til høyre er ALLTID større en Lambda ———— 
        \Lambda \subseteq \braqMed{ \bracMed{\lambda_1, \lambda_2} : \lambda_1 \in \Z, \lambda_2 \in \R},
    \end{align*}
    and then that $\Lambda$ is a subset of \labelcref{eq:2d_all_shift_class_1_tiling}. That is,
    %Furthermore, by following the same argument as in the proof of \cref{thrm:class_all_shift_2d}, we now have that $\Lambda$ is, in fact, a subset of \labelcref{eq:2d_all_shift_class_1_tiling}. That is, 
    \begin{equation*}  %* ———— DENNE inklusjonen DERIMOT, der er mengden til høyre POTENSIELT større en Lambda, men vi må ha likhet ————
        \Lambda \subseteq \underbrace{\braqMed{\begin{pmatrix} m \\ n+\beta_m\end{pmatrix} : m,n \in \Z}}_{A}.
    \end{equation*}
    If we now assume that $\Lambda$ is incomplete with respect to the set $A$, then there would be an arbitrary point $\lambda' \in A$ that is not in $\Lambda$. However, now $\Lambda$ would not be a tiling set anymore. Observe that we would have an entire square, specifically the square at
    \begin{equation*}
        %I^2 + \lambda' = \bras{0+\lambda_1,1+\lambda_1}\times \bras{0+\lambda_2,1+\lambda_2} = \bras{0+m,1+m}\times \bras{0+n+\beta_m,1+n+\beta_m}, 
        %I^2 + \lambda' = \bras{\bras{0+m}},1+m}\times \bras{0+n+\beta_m,1+n+\beta_m}, 
        I^2 + \lambda' = \bras{\bracMed{0+m},\bracMed{1+m}}\times \bras{\bracMed{0+n+\beta_m},\bracMed{1+n+\beta_m}}, 
    \end{equation*}
    with no value, violating the tiling condition in \cref{def:tiling} as we require the value of $1$ almost everywhere for all $x\in \R^2$, and $\mes{I^2 + \lambda'}$ is non-negligible. Hence $\lambda'\in \Lambda$ for $\Lambda$ to be a tiling set. This shows that $\Lambda$ cannot be a direct subset of $A$, which means that we must have equality, and $\Lambda$ must in this first case be of the form \labelcref{eq:2d_all_shift_class_1_tiling}. %where we would not have the value of $1$, violating the tiling condition in \cref{def:tiling}. We require the value of $1$ almost everywhere for $x\in \R^2$, and $\mes{I^2 + \lambda'}$ is non-negligible.



\end{proof}






% transkribert lydlogg
% Her ville jeg hentet frem "nå skal vi se at en konsekvens av kellers theorem" er at disse tiling setsene de må også nødvendigivs se ut som de gjør i theorem 4.20

% Og da kommer det til å komme et bevis som linger veldig mye på beviset for 4.20

% som tar for seg akkurat det samme

% jeg tenker på et theorem som ikke finnes i denne seksjonen som sier at
% Hvis du vet at du har et tiling set for enhetskuben i to dimensjoner så er dette de to mulighetene vi har. 

% da kommer det noe som til en hvis grad vil speile det som står på side 23 til 25.

%ville kanskje hatt med korolaret jeg har skrevet opp først, og så argumentert for det. 
% og så sagt at "nå skal vi se at vi får akkurat de samme settene".  

% Kun for dimensjon 2 her, dimensjon 1 har vi gjort oss ferdig med, fordi jeg har skrevet om dette i tilingskapittelet. 

% Kan si det muntlig: " Nå har vi allerede sett at d 
% "Vi har sett at vi har akkurat de samme tiling mengdene som spektral mengdene for en dimensjon, og nå skal vi se det for to dimensjoner"
% Der har vi et analogt resulat til 4.20 og beviset vil følge de samme linjene.

% For tilings, men i redusert versjon, der når du ser det blir helt likt referert tilbake til theorem 4.20. 

% Og så vil man da bruke Keller til dette

% ikke ta med lemma 5.1, det er det samme som 4.7 ikke si det samme igjen. Forklar i teksten denne likheten.
\end{document}