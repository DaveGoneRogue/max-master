\documentclass[../thesis.tex]{subfiles}
% Separate preamble for this subfile. This preamble is loaded last, so one can override various functions before \begin{document}

% Better comment extension for Vscode colors these comments differently
% Normal comment color
% * Important information
% ! ALERT
% ? Question
% TODO stuff to do
% // This is strikethrough


\begin{document}  
%* Sammenligning tiling sets <--> spectral sets

%! —————— Villee hatt med korolaret jeg har skrevet opp først, og så argumentert for det. 
%* skiller seg ut på et viktig sted, i lemmaet over bruker vi link mellom Ortogonalitet og tingen med null settet, mens i min lemma, så sier jeg bare at null settet til funksjonen er denne tingen. jeg sier ingeinting om at dette er ekvivalent med et ortogonalitetskrav. Selv i remarken etterpå sier vi bare at å være "blabla" er det samme som "lemma 4.7 + kompletthet". Ikke at lemma 4.7 erstatter ortogonalitetskravet. 
%* Kan fremheve denne biten i teksten ved å bruke en ligning 
%*  Forklar i teksten denne likheten mellom keller spectral og lemma 4.7. mitt zero set lemma for I^D

Recall \cref{thrm:keller_tiling}, Keller's \namecref{thrm:keller_tiling}. We have an equivalent version, or more specifically, an equivalent restatement of \cref{thrm:keller_tiling}, which closely relates to the previously mentioned spectral version of Keller's \namecref{thrm:keller_tiling}, \cref{lem:zero_set_AiSp}. The equivalence of \Cref{thrm:keller_tiling} and the following \cref{thrm:keller_alternative} follows by rewriting the subtraction of two distinct elements in \cref{thrm:keller_tiling} into $(\Lambda-\Lambda)\setminus\braq{0}$, and by observing that $\Zstroke_{I^d}$ from \cref{lem:zero_set_jp_1_5} captures the rest of the requirements.

\begin{theorem}[Keller's theorem reformulation]\label{thrm:keller_alternative}
    If $\brac{I^d,\Lambda}$ is a tiling pair, then 
    \begin{equation*}
        \Lambda - \Lambda \subseteq \Zstroke_{I^d} \cup \braq{0}.
    \end{equation*} 
\end{theorem}

When using the alternative version, the connection to the spectral version of Keller's \namecref{thrm:keller_tiling}, \cref{lem:zero_set_AiSp}, becomes apparent in that Keller's requirement for a cube tiling corresponds to the exponentials being orthogonal for the same discrete set $\Lambda$.

\begin{corollary}\label{cor:tiling_pair_implies_orthogonal}
    %! The following result shows that any tiling set for the cube is orthogonal. It is a critical step in our proof that any tiling set for the cube must be a spectrum for the cube and should be compared with the spectral version of Keller's theorem. 
    %If $\brac{I^d, \Lambda}$ is a tiling pair, then $\brac{I^d,\Lambda}$ is an orthogonal pair.\\
    If $\brac{I^d, \Lambda}$ is a tiling pair, then $\Lambda$ gives an orthogonal set of exponentials $E(\Lambda)$.
\end{corollary}

Until now, we have only seen that we have exactly the same tiling sets as spectra for $I^d$ in dimension one. We will now see that the same is true for dimension two. Here we have an analogous result to \cref{thrm:class_all_shift_2d}, which we state in its entirety for convenience. It is worth emphasizing that it, in this case, is as a consequence of Keller's \namecref{thrm:keller_tiling} that we have that these tiling sets must also look like they do in \cref{thrm:class_all_shift_2d}.

\begin{theorem}\label{thrm:class_all_tiling_2d}  %* If given a tiling set for I^2, then this is our only two possibilities. 
    The only subsets $\Lambda \subset \R^2$ such that $\Lambda$ is a tiling set for $\Omega = I^2$ belong to one of the two classes
    \begin{equation}\label{eq:2d_all_shift_class_1_tiling}
        \Lambda = \braqMed{
            \begin{pmatrix}
            m + \alpha \\
            n + \beta_m
            \end{pmatrix} : m,n \in  \Z
            }
    \end{equation}
    or
    \begin{equation}\label{eq:2d_all_shift_class_2_tiling}
        \Lambda = \braqMed{
            \begin{pmatrix}
            m + \beta_n\\
            n + \alpha
            \end{pmatrix} : m,n \in  \Z
            }
    \end{equation}
    where $\alpha \in \R$ is fixed and $\brac{\beta_m}_{m\in \Z}$ is an infinite sequence with $\beta_m \in [0,1)$.
\end{theorem}

\begin{proof}
    Let $\Lambda \subset \R^d$. Keller's theorem, \cref{thrm:keller_tiling}, immediately shows that there must be an integer distance between two distinct points $\lambda,\lambda' \in \Lambda$ if $\Lambda$ is to be a tiling set for $I^2$. That gives us that at least one of $\brac{\lambda_1-\lambda_1'}$ or $\brac{\lambda_2-\lambda_2'}$ is in $\intnozero$. That said, as in the proof of \cref{thrm:class_all_shift_2d} while using \cref{thrm:keller_alternative}, we can assume that the zero-element $\brac{0,0}$ is in $\Lambda$, which would result in the same two options for $\lambda_1$ and $\lambda_2$ for all $\lambda\in \Lambda$. That is, at least one of them must be a non-zero integer. Likewise, if we further assume $\lambda\notin \Z^2$, and that $\lambda_1 \in\intnozero$, we have a case identical to the first of the two possibilities in the proof of \cref{thrm:class_all_shift_2d}. If we now follow the exact same arguments from that proof, we can first show that 
    \begin{align*}  %* ————  den mengden til høyre er ALLTID større en Lambda ———— 
        \Lambda \subseteq \braqMed{ \bracMed{\lambda_1, \lambda_2} : \lambda_1 \in \Z, \lambda_2 \in \R},
    \end{align*}
    and then that $\Lambda$ is a subset of \labelcref{eq:2d_all_shift_class_1_tiling}, with $\lambda_1 = m$ and $\lambda_2 = n+\beta_m$. That is,
    %Furthermore, by following the same argument as in the proof of \cref{thrm:class_all_shift_2d}, we now have that $\Lambda$ is, in fact, a subset of \labelcref{eq:2d_all_shift_class_1_tiling}. That is, 
    \begin{equation*}  %* ———— DENNE inklusjonen DERIMOT, der er mengden til høyre POTENSIELT større en Lambda, men vi må ha likhet ————
        %\Lambda \subseteq \underbrace{\braqMed{\begin{pmatrix} m \\ n+\beta_m\end{pmatrix} : m,n \in \Z}}_{A}.
        \Lambda \subseteq \braqMed{\begin{pmatrix} m \\ n+\beta_m\end{pmatrix} : m,n \in \Z}.
    \end{equation*}
    If we now assume that $\Lambda$ is incomplete with respect to the set \labelcref{eq:2d_all_shift_class_1_tiling}, then there would be an arbitrary point $\lambda'$ in \labelcref{eq:2d_all_shift_class_1_tiling} that is not in $\Lambda$. However, now $\Lambda$ is not tiling set anymore. Observe that we would have an entire square, specifically the square at
    \begin{equation*}
        %I^2 + \lambda' = \brasMed{0+\lambda_1,1+\lambda_1}\times \bras{0+\lambda_2,1+\lambda_2} = \bras{0+m,1+m}\times \bras{0+n+\beta_m,1+n+\beta_m}, 
        %I^2 + \lambda' = \brasMed{\bras{0+m}},1+m}\times \bras{0+n+\beta_m,1+n+\beta_m}, 
        I^2 + \lambda' = \brasMed{\bracMed{0+m},\bracMed{1+m}}\times \brasMed{\bracMed{0+n+\beta_m},\bracMed{1+n+\beta_m}}, 
    \end{equation*}
    with no value. As $\mes{I^2 + \lambda'}$ is non-negligible, this violates \cref{def:tiling} where we require the value of $1$ for almost every $x\in \R^2$. Hence $\lambda'\in \Lambda$ for $\Lambda$ to be a tiling set. This shows that $\Lambda$ cannot be a proper subset of \labelcref{eq:2d_all_shift_class_1_tiling}, which means that we must have equality.% if the use of A instead of \labelcref: and $\Lambda$ must, in this first case, be of the form \labelcref{eq:2d_all_shift_class_1_tiling}.
    
    If we now go back a few steps and assume the opposite, that is $\lambda_2 \in \intnozero$; we have a case identical to the second of the two possibilities in the proof of \cref{thrm:class_all_shift_2d}. Again, by following the same arguments, we can show that $\Lambda$ is a subset of \labelcref{eq:2d_all_shift_class_2_tiling}. That is,
    \begin{equation*}  %* ———— OG DENNE inklusjonen, der er mengden til høyre POTENSIELT større en Lambda, men vi må ha likhet ————
        %\Lambda \subseteq \underbrace{\braqMed{\begin{pmatrix} m +\beta_n \\ n\end{pmatrix} : m,n \in \Z}}_{B}.
        \Lambda \subseteq \braqMed{\begin{pmatrix} m +\beta_n \\ n\end{pmatrix} : m,n \in \Z}.
    \end{equation*}
    If we again assume that $\Lambda$ is incomplete with respect to the set in \labelcref{eq:2d_all_shift_class_2_tiling}, we can similarly show that $\Lambda$ cannot be a proper subset of \labelcref{eq:2d_all_shift_class_2_tiling}. We must have equality. To conclude, $\brac{I^2, \Lambda}$ is a tiling pair in the case where $\Lambda$ is given by either \labelcref{eq:2d_all_shift_class_1_tiling} or \labelcref{eq:2d_all_shift_class_2_tiling}, and there are no other tiling sets for the unit cube in dimension $d=2$.
\end{proof}

\begin{remark}
    By using \cref{cor:tiling_pair_implies_orthogonal} and a few more results from the proof of \cref{thrm:class_all_shift_2d}, we could have shortened the above proof of \cref{thrm:class_all_tiling_2d}, especially the opening. This follows from the following observations. The \namecref{cor:tiling_pair_implies_orthogonal} implies that the tiling pair is also orthogonal. In the proof of \cref{thrm:class_all_shift_2d}, we argue from the orthogonality of $E(\Lambda)$, then it must be that $\Lambda$ belong to either of the two classes of \labelcref{eq:2d_all_shift_class_1} or \labelcref{eq:2d_all_shift_class_2}, or in this case \labelcref{eq:2d_all_shift_class_1_tiling} or \labelcref{eq:2d_all_shift_class_2_tiling} to be specific even though they are the same. Nevertheless, this would allow us to skip the first part of our proof of \cref{thrm:class_all_tiling_2d}, leaving us with only showing the final equality using the tiling argument from \cref{def:tiling}.
\end{remark}

It should be stressed that when working with a spectrum, it is due to the connection between the zero-set and orthogonality that the spectrum must look like they do. Similarly, when working with a tiling set, it is due to Keller's theorem. That said, it is interesting that Keller's theorem can be reformulated in a manner that aligns with the concept of the zero-set, which only strengthens the interconnectedness between spectral sets and tiling sets. \SigridChange{this last sentence I am a bit unsure about, maybe kolonutz has something to say here.}

Now for the equivalence of tiling sets and spectra for the unit cube. We remark that Jorgensen and Pedersen in \cite{jorgensenSpectralPairsCartesian2001} further showed that one has the same tiling sets as spectra for the unit cube in dimension three and classified all of them. Additionally, if we add the assumption that $\Lambda$ is periodic, we have the following result presented after the \namecref{def:periodic_set}.
\begin{definition}(Periodic set)\label{def:periodic_set}
    Let $A\subset\R^d$ be a discrete subset. Then $A$ is \emph{periodic} if there exists a finite set $K\subset\R^d$ and an invertible $d\times d$ matrix $M$ consisting of real entries such that 
    \begin{equation*}
        A = K + M\Z^d. \qedhere
    \end{equation*}
\end{definition}
\begin{remark}
    How to state that this does not violate our earlier interpretation of periodicity in tilings
\end{remark}

\begin{theorem}\label{thrm:periodic_equivalence}
    Let $\Lambda= K + L$, where $K\subset \R^d$ is a subset with $\bral{K}$ elements, and $L=M\Z^d$ is the lattice in $\R^d$ generated by the columns of $M$, an invertible $d\times d$ matrix with real entries. If $L\cap(K-K)= \braq{0}$, then the following are equivalent:
    \begin{enumerate}[label=(\roman*), topsep=-3pt]
        \item $\brac{I^d,\Lambda}$ is a spectral pair
        \item $\brac{I^d,\Lambda}$ is a tiling pair  \SigridChange{They have a tiny bit different definition here}
        \item $\bral{K} = \bral{\det(M)}$ \textsc{and} $L+\brac{K-K}\subseteq \Zstroke_{I^d} \cup \braq{0}$  \SigridChange{This dont make sense}
    \end{enumerate}
\end{theorem}

\begin{remark}
    Regarding this  $L+\brac{K-K}\subseteq \Zstroke_{I^d} \cup \braq{0}$, blabla, how to get this right?
\end{remark}

In addition to the classifications, Jorgensen and Pedersen also proved \cref{thrm:periodic_equivalence} in \cite{jorgensenSpectralPairsCartesian2001}, showing that \cref{thrm:periodic_equivalence} is a solution to the periodic case of the main result, \cref{thrm:main_result}. What is more interesting about \cref{thrm:periodic_equivalence} is that it has a strong connection to the proofs where Keller's \namecref{conj:keller_tiling} is disproved. Notably, if the \namecref{conj:keller_tiling} is false for $\R^d$, the tiling in $\R^n$ for some $n\geq d$ that counterexample Keller's \namecref{conj:keller_tiling} has two additional properties \cite{lagariasKellerCubetilingConjecture1992}: First, the centers of all the unit cubes are in $\frac{1}{2}\Z^n$. Secondly, the tiling is periodic and contains the period lattice $2\Z^n$. The latter is the important one because it means that the counterexamples can be periodic. Specifically, in the case where Keller's \namecref{conj:keller_tiling} was disproved for $d\geq10$, they found a periodic set of translations of the form $\Lambda= K + (2\Z)^{10}$, where $\bral{K} = 2^{10} = 1024$ elements. Using \cref{thrm:periodic_equivalence}, we have that the tiling pair where no two tiles share an entire face $\brac{I^{10},\Lambda}$ is also a spectral pair in $\R^{10}$ \cite{jorgensenSpectralPairsCartesian2001}. As Mackey \cite{mackeyCubeTilingDimension2002} improved this result further by finding a set of translations of the form $\Lambda= K + (2\Z)^{8}$ where $\bral{K} = 2^{8} = 256$ elements, we also have that the tiling pair $\brac{I^{8},\Lambda}$ is a spectral pair in $\R^8$ using \cref{thrm:periodic_equivalence}. What is interesting about these spectral pairs is that $\Lambda-\Lambda$ does not contain one of the canonical unit basis vectors for $\R^{10}$ \cite{jorgensenSpectralPairsCartesian2001}, which also is true in $\R^{8}$ with the resolution of the \namecref{conj:keller_tiling} \cite{brakensiekResolutionKellerConjecture2020}. Hence, we know that in all $d\geq8$, periodic tilings sets exist where no two tiles share an entire face, with the added property that they do not contain one of the canonical basis vectors of $\R^d$.

This result is important for the previous \cref{ex:cannot_class_all}. In particular, the \namecref{ex:cannot_class_all} presents a method for creating spectra that make $\brac{I^d,\Lambda}$ an $\R^d$-spectral pair by giving an explicit \namecref{eq:cannot_class_all}, \cref{eq:cannot_class_all}, for the set of points in $\Lambda$. Note that there are variations of \labelcref{eq:cannot_class_all} that result from a permutation of the $d$ coordinates; however, the import result is that the corresponding combinations given by \labelcref{eq:cannot_class_all} for $d\geq8$, \emph{cannot} catalog \textsc{all} possible spectra that make $\brac{I^d,\Lambda}$ an $\R^d$-spectral pair. Said differently, \labelcref{eq:cannot_class_all} cannot provide all possible spectra for $I^d$. That this is the same dimension where Keller's \namecref{conj:keller_tiling} fails first is no coincidence and follows from the following: Since there exist tilings $\Lambda-\Lambda$ for all $d\geq8$ that do not contain one of the canonical base vectors, then $\Lambda$ cannot be on the form given by \labelcref{eq:cannot_class_all}, as in the different permutation of the \labelcref{eq:cannot_class_all} one uses at least one of the canonical basis vectors.
%* the import result is that the corresponding combinations given by \labelcref{eq:cannot_class_all} for the dimension $d\geq8$ cannot !!!
%* given by the equation? / the permutations? — Most likely the latter



\vspace*{3cm}





\SigridComment{closing the thesis}

Notes, do not read yet


%! OM ekvivalens og deepdive i bevisene
%* packing tiling, orthognonality
The purpose of our paper is to give an alternative and, perhaps, more illuminating proof of this fact, which is based on a characterization of translational tiling by a Fourier analytic criterion.

%* Fra Kineseren om det andre beviset 
Iosevich and Pedersen [7] later and independently established the above-mentioned conjecture of Jorgensen and Pedersen by a different approach basing on the geometric argument, which is analogous to the argument used by Perron [20] in his proof of Keller’s conjecture for dimension np6: Kolountzakis [11] gave an alternative proof of this fact, which is based on a characterization of translational tiling by a Fourier analytic criterion. 

%* Bevisenen i L,R,W skiller seg også fra de andre i at de viser at det holder for andre sett en linjære transformerte n-cubes
Partly this has been showd by Lagarias, Reeds, Wang where they gave a one dimensional example of more general sets, $\Omega=\bras{0,1}\cup \bras{2,3}$  that satisfied the conjecture \cite{lagariasOrthonormalBasesExponentials2000}.

%  Tiling the line with translations of one tile
Note that this example have no lattice tiling.
% The analogue of Theorem 1 is false in higher dimensions, e.g. the unit square T in R2 gives infinitely many nonperiodic tilings of R2 which are translation- inequivalent. Theorem 1 also fails in general for regions T admitting a monohe- dral tiling of R, as shown in Example 1. A final observation is that there are pro- totiles T that tile R by translation but have no lattice tilings, e.g. T = [0, 1]∪[2, 3].
% Theorem 1 asserts periodicity of all tilings, but it does not give any informa- tion about the cosets of such a periodic tiling. The main result of the paper is the following rationality result for such cosets.


%! the same set $\Lambda\subset\R^d$ is indeed a spectrum and tiling set for $\Omega = I^d$. 
%! Highlight denne direkte linken i chapter 5 med keller. dvs: the connection between tiles and spectrum is more direct for OMEGA=I^d than for other sets due to "NULL sett tingen" and the corresponding result for tilings (Keller). 




%! If solving the periodic case, 
% Defenition from chapter 4 of J&P: A discrete set T ⊂ Rd is periodic if a finite set L ⊂ Rd exists and an invertible d × d matrix R with real entries such that T = L + RZd.

%! fra periodicity of spectrum in dimension one
%Definition 1.3. A set S ⊆ Rd is called (fully) periodic if there exists a lattice L ⊆ Rd (a discrete subgroup of Rd with d linearly independent generators; the period lattice) such that S + t = S for all t ∈ L. We call a translation tiling periodic if the set of translations is periodic.

%! Også def på pdf side 2 i Tiling the line with translates of one tile 

%* Jørgen pedersen skriver: 
%* The fact that this simple tiling pattern for the cube $I^d$ in $d$-dimensions is broken for $d=10$ follows from examples of Lagarias and Show. (beviset for conjecture større en 10). It is shown there that for each $d\geq 10$, there exists a tiling of $\R^d$ by translates of $I^d$ such that no two tiles have a complete face in common. These examples also demonstrate, see section 4, that if $d\geq 10$, then the corresponding combinations given in $2.2$ (denne ligningen har jeg også med, må kommentere noe der!! Linke med conjecturen) do not supply all possible spectra for $I^d$. 

%* Om 2.2 skriver de følgende:
%* Of course there are obvious modifications resulting of permutations of the d-coordinates; but when $d\geq10$, these configurations do not suffice for cataloging all possible spectra $\Lambda$ which turn $(I^d,\Lambda)$ into an $\R^d$-spectral pair. see section 4. 

%* SEKSJON 4 
%* går litt mer i detalj på conjecturen og hva den er uløst for. 


% transkribert lydlogg
% Her ville jeg hentet frem "nå skal vi se at en konsekvens av kellers theorem" er at disse tiling setsene de må også nødvendigivs se ut som de gjør i theorem 4.20

% Og da kommer det til å komme et bevis som linger veldig mye på beviset for 4.20

% som tar for seg akkurat det samme

% jeg tenker på et theorem som ikke finnes i denne seksjonen som sier at
% Hvis du vet at du har et tiling set for enhetskuben i to dimensjoner så er dette de to mulighetene vi har. 

% da kommer det noe som til en hvis grad vil speile det som står på side 23 til 25.

%ville kanskje hatt med korolaret jeg har skrevet opp først, og så argumentert for det. 
% og så sagt at "nå skal vi se at vi får akkurat de samme settene".  

% Kun for dimensjon 2 her, dimensjon 1 har vi gjort oss ferdig med, fordi jeg har skrevet om dette i tilingskapittelet. 

% Kan si det muntlig: " Nå har vi allerede sett at d 
% "Vi har sett at vi har akkurat de samme tiling mengdene som spektral mengdene for en dimensjon, og nå skal vi se det for to dimensjoner"
% Der har vi et analogt resulat til 4.20 og beviset vil følge de samme linjene.

% For tilings, men i redusert versjon, der når du ser det blir helt likt referert tilbake til theorem 4.20. 

% Og så vil man da bruke Keller til dette

% ikke ta med lemma 5.1, det er det samme som 4.7 ikke si det samme igjen. Forklar i teksten denne likheten.
\end{document}