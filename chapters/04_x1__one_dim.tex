\documentclass[../thesis.tex]{subfiles}
% Seperate preamble for this subfile. This preamble is loaded last, so it may be used to override various functions.

% Better comment extension for Vscode colors these comments differently
% Normal comment color
% * Important information is highlighted
% ! ALERT
% ? Question
% TODO stuff to do
% // this is strikethrough


\begin{document}

For $d=1$ and $\Lambda=\Z$ we have the following set of exponential functions
\begin{equation}
    \onedexp
\end{equation}
%This set is sometimes referred to as the \emph{complex trigonometric system}. 
Since the complex exponential functions $e_{\lambda}$ are continuous and Rienman integrable on $I$, they are also Lebesgue integrable on $I$. In other words $\onedexp \subset L^2(I)$. %which imply $e_\lambda \in L^2(I)$ for all $\lambda \in \Z$. 
\begin{lemma}
    The set of exponential functions $\onedexp$ is an orthonormal basis for $L^2(I)$
\end{lemma}
\begin{proof}
    We begin with showing orthogonality. Let $e_{\lambda}, e_{\lambda'} \in \onedexp $ so that $\lambda \neq \lambda'$ and compute the inner product
    \begin{align*} 
        \langle e_{\lambda},e_{\lambda'} \rangle_{L^2(I)} &= \int_0^1 e_{\lambda}(t)\overline{e_{\lambda'}(t)} dt\\ 
        &= \int_0^1 e^{2 \pi i {\lambda}t} e^{-2 \pi i {\lambda'}t} dt\\
        &= \int_0^1 e^{2 \pi i (\lambda-\lambda')t} dt\\
        &=\frac{1}{2 \pi i (\lambda-\lambda')}\left( e^u  \big| _0^{2 \pi i (\lambda-\lambda')}\right) & (\text{using }u=2 \pi i (\lambda-\lambda')t)\\
        &=\frac{1-1}{2 \pi i (\lambda-\lambda')} & (\text{since } e^{2 \pi i (\lambda-\lambda')} = 1)\\
        &= 0.
    \end{align*}
    Hence $\onedexp$ is orthogonal. Furthermore, if  $\lambda =\lambda'$ the inner product is
    \begin{align}\label{eq:exp_norm_one}
        \| e_\lambda \|^2_{L^2(I)} 
        = \langle e_{\lambda},e_{\lambda} \rangle_{L^2(I)} 
        = \int_0^1 e^{2 \pi i (\lambda-\lambda)t} dt
        = \int_0^1 e^{0} dt = t  \big| _0^{1}= 1
    \end{align}
    which shows that the system $\left\{ e_{\lambda} \right\}_{\lambda\in \mathbb{Z}}$ is not only orthogonal, it is also orthonormal. 
    % * ------------------------------------------------------------------
    % * BASIS PROOOF

    Going forward, to show that we have an orthonormal basis we must prove that $\onedexp$ is complete in $L^2(I)$. By\cref{thrm:orthonormal_equivalences} this is equivalent to showing $\spn{\onedexp}$ is dense in $L^2(I)$ with respect to the \Ltwonorm. 
    %More specificaly, if $\|f-f'\|_{L^2(I)} < \varepsilon'$ for $\varepsilon' > 0$ and two functions $f,f'$ where $f\in L^2(I)$ and $f'\in \spn{\onedexp}$, follows from defenition of dense that $\spn{\onedexp}$ is dense in $L^2(I)$.
    %
    %THIS IS THE SAME AS the closed span, which is the set of all elements a_i of the span(E) that converges to a. If this closed span is the whole of X then E is dense. So by showing E is dense w.r.t a norm, is essentially showing the closed span = the set.
    %
    %More specificaly, choose any function $f\in L^2(I)$ and fix a real number $\varepsilon >0$. If we can show that there is a function $T \in \spn{\onedexp}$ such that $\|f-T\|_{L^2(I)} < 3 \varepsilon$, then we are done as this implies completeness from the defenition of closed span .
    %First, choose any function $f\in L^2(I)$ and fix a real number $\varepsilon >0$. %Our goal in the end is to show that there is a function $T \in \spn{\onedexp}$ such that $\|f-T\|_{L^2(I)} < 3 \varepsilon$.

    First, choose any function $f\in L^2(I)$ and fix a real number $\varepsilon >0$. Since $C(I)$ is dense in $L^2(I)$ with respec to the $L^2$-norm, we know that there exist a continuous function $g \in C(I)$ such that
    \begin{equation}
        \| f-g \|_{L^2(I)}^2 = \int_0^1 \left|f(t)-g(t) \right|^2dt < \varepsilon^2
    \end{equation}
    Now, using the fact that $C_{\text{per}}(I)$ is dense in $C(I)$, we have a function $h\in C_{\text{per}}(I)$ such that 
    \begin{equation}
        \| g-h \|_{L^2(I)}^2 = \int_0^1 \left|g(t)-h(t) \right|^2dt < \varepsilon^2
    \end{equation}

    \begin{theorem}[Weierstrass Approximation Theorem \cite{durenInvitationClassicalAnalysis2012}]\label{thrm:weierstrassApprox}
        Let $q(x)$ be continuous on the interval $[0, 1]$, with $q(0)=q(1)$. Then for each $\varepsilon>0$ there is a trigonometric polynomial $T(x)$ such that $|q(x)-T(x)|<\varepsilon$ for $0 \leq x \leq 1$.
    \end{theorem}

    By using \cref{thrm:weierstrassApprox} on $h\in C_{\text{per}}(I)$ we get directly that there exist a trigonometric polynomial $T(t)$  such that $|h(t)-T(t)|<\varepsilon$. Furthermore, note that $T(t)$ in fact a member of $\spn{\onedexp}$ since 
    \begin{equation*}
        \spn{\onedexp}=\left\{\sum_{\lambda=-N}^N c_\lambda e^{2\pi i \lambda t} : \lambda \in \N, c_\lambda \in \C\right\},
    \end{equation*}
    and a trigonometric polynomial of degree $N$ and period $P$ that in general can be expressed as
    \begin{equation*}
        p_N(x)=\sum_{n=-N}^N c_n e^{i \frac{2 \pi}{P} n x}, x\in \R.
    \end{equation*}
    When the period $P=1$, as is in our case, we see that $P_N(x)= T(t)$. Continuing on, using the \Ltwonorm we have that for %$e^{2 \pi it}$ is a trigonometric polynomial by definition. Set $p(t)=T(t)$ and using the \Ltwonorm we have
    \begin{align*}
        \| h-T \|_{L^2(I)}^2 =& \int_0^1 \left|h(t)-T(t) \right|^2dt \\ 
        \leq &\int_0^1 \varepsilon^2 dt\\ 
        =& \varepsilon^2.
    \end{align*}
    Now using the triangle inequality to combine the above estimates, we get
    \begin{align*} 
        \| f-T\|_{L^2(I)} \quad \leq&\quad  \| f-g\|_{L^2(I)} + \| g-h\|_{L^2(I)} +\| h-T\|_{L^2(I)}\\
        <& \quad \varepsilon + \varepsilon +\varepsilon \\
        =& \quad 3 \varepsilon
    \end{align*}
    For this reason, we can approximate every $f\in L^2(I)$ by a function $T \in \spn{\onedexp}$ as closely as we want. In other words, $\spnclos{\onedexp}=L^2(I)$, and by \cref{def:closure} $\spn{\onedexp}$ is dense in $L^2(I)$ which finalizes our proof%, and this finalizes our proof. %we have shown that $\onedexp$ indeed is complete. This 
\end{proof}

\begin{remark}
    Since $\onedexp$ is an orthonormal basis for $L^2(I)$, it is clearly also an orthogonal basis. By direct application of \cref{def:spectral_set} we have that $\brac{\Omega, \Lambda}=\brac{I,\Z}$ is a spectral pair.
\end{remark}



Vet ikke hvor jeg vil ha denne, generalisering av ONB for hele $L^2([0,1]^d)$, som jo burde stemme.
\begin{lemma}
    The set of exponential functions $\alldexp$ is an orthonormal basis for $L^2(I^d)$
\end{lemma}

\begin{proof}
    Let $x\in \R^d$ and $\lambda, \lambda' \in Z^d$. Note that they are all vectors on the form $x=(x_1,\dots, x_n, \dots, x_d)$.
    %Using vector notation: $x=(x_1,\dots x_d)$, $\lambda=(\lambda_1, \dots, \lambda_d)$, and $\kappa=(\kappa_1, \dots, \kappa_d)$.
    Show that we can rewrite 
    \begin{align*}
        \langle e_{\lambda},e_{\lambda'} \rangle_{L^2(I^d)} &= \int_{[0,1]^d} e^{2\pi i \langle\lambda, x\rangle} e^{-2 \pi i \langle \lambda', x\rangle} dx \\
        &= \int_{[0,1]^d} e^{2\pi i  (\lambda_1x_1 + \dots +\lambda_d x_d)} e^{-2\pi i  (\lambda_1' x_1 + \dots +\lambda_d' x_d)} dx\\
        &= \int_{[0,1]^d} e^{2\pi i  (\lambda_1 -\lambda_1')x_1} \cdots e^{2\pi i  (\lambda_d -\lambda_d')x_d} dx\\
        &= \int_0^1 e^{2\pi i  (\lambda_1- \lambda_1')x_1} \int_0^1 e^{2\pi i  (\lambda_2 - \lambda_2')x_2}  \cdots \int_0^1 e^{2\pi i  (\lambda_d - \lambda_d')x_d} dx_1 dx_2 \dots dx_d \\
        &=\prod_{n=1}^d \int_0^1 e^{2\pi i  (\lambda_n- \lambda_n')x_n} d x_n 
    \end{align*}
    Show the following 
    if $\lambda = \lambda'$ we get that its equal to 1
    if $\lambda \neq \lambda'$, then we must show that for some index $n$ we have that $\lambda_n \neq \lambda_n'$ which is equal to 0, forcing everyting to be 0, i.e integral with index n vanish.
    That is orthonormality.

    Show basis either by generalising the proof over to d dimensions
    
    or by using stone-weierstrass in the complex version to show the system is first dense in $C_{per}(I^d)$
    and then, since $C_{per}(I^d)$ is dense in $L^2(I^d)$, then the system is dense in $L^2(I^d)$. 

\end{proof}

\end{document}