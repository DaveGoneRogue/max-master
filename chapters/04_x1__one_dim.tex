\documentclass[../thesis.tex]{subfiles}
% Separate preamble for this subfile. This preamble is loaded last, so it may be used to override various functions.

% Better comment extension for Vscode colors these comments differently
% Normal comment color
% * Important information is highlighted
% ! ALERT
% ? Question
% TODO stuff to do
% // this is strikethrough


\begin{document}
In one dimension, the unit cube is simply the unit interval $I=\bras{0,1}$. It is a well-known result from Fourier analysis that $\Lambda = \Z$ is a spectrum for $I$. In fact, not only is the system
\begin{equation*}
    E(\Z) = \braqMed{\indicator{I}{t} e^{2\pi i t \lambda} : \lambda \in \Z}
\end{equation*} 
an orthogonal basis for $L^2(\Omega)$, but it is also an orthonormal basis.
\begin{lemma}\label{lem:exp_onb_onedim}
    The set of exponential functions $E\brac{\Z}$ is an orthonormal basis for $L^2(I)$
    % For $t\in \R$, then the set of exponential functions $E\brac{\Z}$ is an orthonormal basis for $L^2(I)$
\end{lemma}
\begin{proof}
    We begin with showing orthogonality. Let $e_{\lambda}, e_{\lambda'} \in E(\Z)$ with $\lambda \neq \lambda'$. %Then, since complex exponential functions are continuous and Rienman integrable on $I$, they are also Lebesgue integrable on $I$. We compute
    \begin{align}\label{eq:orthog_one_dim_req}
        \langle e_{\lambda},e_{\lambda'} \rangle_{L^2(I)} &= \int_0^1 e_{\lambda}(t)\overline{e_{\lambda'}(t)} dt \nonumber\\ 
        &= \int_0^1 e^{2 \pi i {\lambda}t} e^{-2 \pi i {\lambda'}t} dt \nonumber\\
        &= \int_0^1 e^{2 \pi i (\lambda-\lambda')t} dt \nonumber\\
        &=\frac{1}{2 \pi i (\lambda-\lambda')} \brac{e^{2 \pi i (\lambda - \lambda')} -1} \\
        &=\frac{1-1}{2 \pi i (\lambda-\lambda')} & (\text{since } e^{2 \pi i (\lambda-\lambda')} = 1) \nonumber\\
        &= 0 \nonumber.
    \end{align}
    Hence the system $E(\Lambda)$ is orthogonal. Furthermore, if $\lambda =\lambda'$ then the inner product is
    \begin{align*}\label{eq:exp_norm_one}
        \| e_\lambda \|^2_{L^2(I)} 
        = \langle e_{\lambda},e_{\lambda} \rangle_{L^2(I)} 
        = \int_0^1 e^{2 \pi i (\lambda-\lambda)t} dt
        = \int_0^1 e^{0} dt = 1
    \end{align*}
    which shows that $E(\Z)$ is not only orthogonal, it is also orthonormal. 
    % * ------------------------------------------------------------------
    % * BASIS PROOOF

    Let us now see that $E(\Z)$ is complete in $L^2(I)$. By \cref{thrm:orthonormal_equivalences}, this is equivalent to showing $\spn{E(\Z)}$ is dense in $L^2(I)$ with respect to the \Ltwonorm. 
    %More specificaly, if $\|f-f'\|_{L^2(I)} < \varepsilon'$ for $\varepsilon' > 0$ and two functions $f,f'$ where $f\in L^2(I)$ and $f'\in \spn{\onedexp}$, follows from defenition of dense that $\spn{\onedexp}$ is dense in $L^2(I)$.
    %
    %THIS IS THE SAME AS the closed span, which is the set of all elements a_i of the span(E) that converges to a. If this closed span is the whole of X, then E is dense. So by showing E is dense w.r.t a norm, is essentially showing the closed span = the set.
    %
    %More specificaly, choose any function $f\in L^2(I)$ and fix a real number $\varepsilon >0$. If we can show that there is a function $T \in \spn{\onedexp}$ such that $\|f-T\|_{L^2(I)} < 3 \varepsilon$, then we are done as this implies completeness from the definition of closed span.
    %First, choose any function $f\in L^2(I)$ and fix a real number $\varepsilon >0$. %Our goal, in the end, is to show that there is a function $T \in \spn{\onedexp}$ such that $\|f-T\|_{L^2(I)} < 3 \varepsilon$.

    Fix a function $f\in L^2(I)$ and a real number $\varepsilon >0$. Since $\Cper(I)$ is dense in $L^2(I)$ by \cref{lem:c_per_dense_c_and_dense_L2}, we know that there exists a continuous function $g \in \Cper(I)$ such that
    \begin{equation}\label{eq:estimate_1}
        \| f-g \|_{L^2(I)}^2 = \int_0^1 \left|f(t)-g(t) \right|^2dt < \varepsilon^2
    \end{equation}
    We then use the Weierstrass Approximation Theorem to approximate $g$ with a trigonometric polynomial. 
    \begin{theorem}[Weierstrass Approximation Theorem]\label{thrm:weierstrassApprox}
        Let $q\in \Cper (I)$. Then for each $\varepsilon>0$ there exists a trigonometric polynomial %of degree $N$
        \begin{equation}
            T(t) = \sum_{\lambda=-N}^N c_\lambda e^{2\pi i \lambda t},
        \end{equation}
        such that $|q(t)-T(t)|<\varepsilon$ for $0 \leq t \leq 1$ \cite[p.~202]{durenInvitationClassicalAnalysis2012}.
    \end{theorem}
    From \cref{thrm:weierstrassApprox}, we get immediately that there exists a trigonometric polynomial $T(t)$ such that $|g(t)-T(t)|<\varepsilon$ for $0 \leq t \leq 1$. Note that $T(t)$ is a member of $\spn{E(\Z)}$ since % N er degree! mao, lambda er degreE? / frequency?
    \begin{equation*}
        \spn{E(\Z)}=\left\{\sum_{\lambda=-N}^N c_\lambda e_\lambda : N \in \N, c_\lambda \in \C\right\}
    \end{equation*}
    %is simply a set of trigonometric polynomials. Continuing on, using the \Ltwonorm we have that
    We have that
    \begin{align}\label{eq:estimate_2}
        \| g-T \|_{L^2(I)}^2 = \int_0^1 \left|g(t)-T(t) \right|^2dt \leq \int_0^1 \varepsilon^2 dt = \varepsilon^2.
    \end{align}
    Now, using the triangle inequality to combine the estimates in \labelcref{eq:estimate_1} and \labelcref{eq:estimate_2}, we get
    \begin{align*}
        \| f-T\|_{L^2(I)} \leq  \| f-g\|_{L^2(I)} +\| g-T\|_{L^2(I)} < \varepsilon + \varepsilon =  2 \varepsilon
    \end{align*}
    This shows that $\spn{E(\Z)}$ is dense in $L^2(I)$, which finalizes our proof.
    %For this reason, we can approximate every $f\in L^2(I)$ by a function $T \in \spn{\onedexp}$ as closely as we want. In other words, $\spnclos{\onedexp}=L^ 2(I)$, and by \cref{def:closure} $\spn{\onedexp}$ is dense in $L^2(I)$ which finalizes our proof%, and this finalizes our proof. %we have shown that $\onedexp$ indeed is complete. This 
\end{proof}

%The proof above can easily be generalized to higher dimensions, where one can see that $\Lambda = \Z^d$ is a spectrum for $I^d$. This result is summarised in the following lemma.
The proof above can easily be generalized to higher dimensions, where $\Lambda = \Z^d$ is a spectrum for $I^d$.
\begin{lemma}
    The set of exponential functions $E(\Z^d) = \braqMed{\indicator{\Omega}{t} e_\lambda(t) : \lambda \in \Z^d}$ is an orthonormal basis for $L^2(I^d)$.
\end{lemma}

However, whereas we will see in \cref{sec:spec_higher_dim} that $I^d$ for $d>1$ has many other spectra, the flexibility for $d=1$ is minimal. 

%! Må skrives om når det kommer resultat om skift, i.e MÅ KORTES NED
We have shown that $\Lambda = \Z$ is a spectrum for $I$. What other sets have the same property? We know from \cref{eq:orthog_one_dim_req} that $e^{2\pi i(\lambda-\lambda')}$ must be equal to $1$, otherwise the elements of $E(\Z)$ are not pairwise orthogonal. This only happens when the exponent  $2\pi i (\lambda - \lambda')$ is some \emph{integer} multiple of $2 \pi i$. That is to say, $e^{2\pi i(\lambda-\lambda')} = 1$ only when $(\lambda-\lambda') \in \Z$. There are \textsc{no} other sets that satisfy this. However, observe that we have some flexibility when choosing $\Lambda = \Z$. As the following will show, we can shift $\Z$ with any fixed real number $\alpha$, and \cref{eq:orthog_one_dim_req} will still be equal to $1$. This can easily be seen from 
\begin{equation}
    \langle e_{\lambda+\alpha},e_{\lambda'+\alpha} \rangle_{L^2(I)} = \int_0^1 e^{2 \pi i ((\lambda+\alpha) - (\lambda'+\alpha))t} dt = \int_0^1 e^{2 \pi i (\lambda-\lambda')t} dt = \langle e_{\lambda},e_{\lambda'} \rangle_{L^2(I)}.
\end{equation}
Hence, $\Lambda = \Z + \alpha$ is also a spectrum for $I$. If $\alpha$ was not fixed we would have $(\lambda+\alpha) - (\lambda'+\alpha) \neq (\lambda - \lambda')$. Note that if $\alpha=0$, we have the original case. Thus, we technically only have a single choice for $\Lambda$ when $d=1$, that is, $\Lambda = \Z+ \alpha$. 

Another question is whether we need all of $\Z +\alpha$. In other words, can one select a subset $A\subset \Z + \alpha$, where $\Lambda = A$ still is a spectral pair for $I$? Observe the following. Given $\lambda, \gamma \in \Z+\alpha$, and let $e_{\gamma}$ denote the single element not in $E(A)$. This means that $e_\gamma$ is orthogonal to all elements of $E(A)$, i.e that $\braa{e_{\lambda}, e_{\gamma}} = 0$ for all $\lambda \in \Z +\alpha\setminus \gamma$. However, as shown in \cref{lem:ONB_alternative_def}, the only element in $L^2(I)$ orthogonal to all elements $e_{\lambda} \in E(A)$ is the zero-element. Thus we have a contradiction, and we must have that $A = \Z+\alpha$.

Thus, we have established the following result.
\begin{proposition}
    For a fixed $\alpha \in \R$, the only subsets $\Lambda \in \R$ such that $\Lambda$ is a spectrum for $I$ are the translates $\Lambda_\alpha = \Z + \alpha$
\end{proposition}
%In other words, we know that $\lambda$ must range over a set of integers. 
%
% FIN SETTNING: Showing that an orthonormal basis cannot be a strict subset of another orthonormal basis for the same space. 
\end{document}