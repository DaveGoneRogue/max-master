\documentclass[../thesis.tex]{subfiles}
% Seperate preamble for this subfile. This preamble is loaded last, so it may be used to override various functions.

% Better comment extension for Vscode colors these comments differently
% Normal comment color
% * Important information is highlighted
% ! ALERT
% ? Question
% TODO stuff to do
% // this is strikethrough


\begin{document}

For $d=1$ we have the following set of exponential functions
\begin{equation}
    \{ e^{2\pi i \lambda t } : \lambda \in \Lambda\}
\end{equation}

The complex trigonometric system is orthogonal on $L^2([0,1])$. This follows from the following; As the complex exponential functions are continuous, they are also Lebesgue integrable on $[0,1]$. Let $e_{\lambda}, e_{\lambda'} \in \left\{ e_{\lambda} \right\}_{\lambda\in \mathbb{Z}}$ so that $\lambda \neq \lambda'$ and compute the inner product
\begin{align*} 
\langle e_{\lambda},e_{\lambda'} \rangle_{L^2([0,1])} 
&= \int_0^1 e_{\lambda}(t)\overline{e_{\lambda'}(t)} dt\\ &= \int_0^1 e^{2 \pi i {\lambda}t} e^{-2 \pi i {\lambda'}t} dt\\
&= \int_0^1 e^{2 \pi i (\lambda-\lambda')t} dt\\
&=\frac{1}{2 \pi i (\lambda-\lambda')}\left( e^u  \big| _0^{2 \pi i (\lambda-\lambda')}\right)\\
&=\frac{e^{2 \pi i (\lambda-\lambda')}-1}{2 \pi i (\lambda-\lambda')}\\
&= 0
\end{align*}

By change of variables, and later using that $(\lambda-\lambda') \in \mathbb{Z}$ imply $e^{2 \pi i (\lambda-\lambda')} = 1$. Hence $\left\{ e_{\lambda} \right\}_{\lambda\in \mathbb{Z}}$ is orthogonal. Furthermore, if  $\lambda =\lambda'$ the inner product is
\begin{align}\label{eq:exp_norm_one}
    \| e_\lambda \|^2_{L^2([0,1])} 
    = \langle e_{\lambda},e_{\lambda} \rangle_{L^2([0,1])} 
    = \int_0^1 e^{2 \pi i (\lambda-\lambda)t} dt
    = \int_0^1 e^{0} dt = t  \big| _0^{1}= 1
\end{align}

which shows that the system $\left\{ e_{\lambda} \right\}_{\lambda\in \mathbb{Z}}$ is not only orthogonal, it is also orthonormal. 


% * ------------------------------------------------------------------
% * BASIS PROOOF
To show that $\left\{ e_{\lambda} \right\}_{\lambda\in \mathbb{Z}}$ is a basis, we must show that it is a complete sequence in the sense of \cref{thrm:orthonormal_equivalences} \cref{eq:oe_a}, which is that $\overline{\operatorname{span}} \left( \left\{ e_{\lambda} \right\}_{\lambda\in \mathbb{Z}} \right) = L^2{([0,1])}$.

First, choose any function $f\in L^2([0,1])$ and fix a real number $\varepsilon >0$. Our goal is to show that there is a function $p \in \operatorname{span} \left( \left\{ e_{\lambda} \right\}_{\lambda\in \mathbb{Z}} \right)$ such that $\|f-p\|_{L^2([0,1])} < 3 \varepsilon$, which will imply completeness from the equivalence of \cref{thrm:orthonormal_equivalences} \cref{eq:oe_c} and \cref{thrm:orthonormal_equivalences} \cref{eq:oe_a}.

Since $C[0,1]$ is dense in $L^2([0,1])$ \cite{heilMetricsNormsInner2018}, we know that there exist a continuous function $g \in C[0,1]$ such that %$\| f-g \|_{L^2([0,1])} < \varepsilon$. If $g$ is the zero function, then we can take $p=0$ and we are done with the proof
\begin{equation}
    \| f-g \|_{L^2([0,1])}^2 = \int_0^1 \left|f(t)-g(t) \right|^2dt < \varepsilon^2
\end{equation}
% * BUILD UP TO h function 
%If $g$ is the zero function, then we can take $p=0$ and we are done with the proof. Thus we consider the case where $g$ is not identically zero. Going forward, our goal is to show that $\left\{ e_{\lambda} \right\}_{\lambda\in \mathbb{Z}}$ is complete in the subspace $C_{\text{per}}[0,1]$ with respect to the $\|\cdot\|_u$-norm. Here, $C_{\text{per}}[0,1]$ is the set of all periodic, continuous, real-valued, functions on the interval $[0,1]$:
Going forward, our goal is to show that $\left\{ e_{\lambda} \right\}_{\lambda\in \mathbb{Z}}$ is complete in the subspace $C_{\text{per}}[0,1]$ using the weierstrass theorem \cite{durenInvitationClassicalAnalysis2012}. We do not require that $g \in C_{\text{per}}[0,1]$, meaning that the requirement $g(0)=g(1)$ need not hold. Instead we will construct a function $h$ that belongs to $C_{\text{per}}[0,1]$. Let  $\delta = \varepsilon^2/(8\|g\|_\infty)$, and $h$  be given by %, and which satisfies 
%\begin{equation*}
%    \| g-h \|_{L^2([0,1])}^2 = \int_0^1 \left|g(t)-h(t) \right|^2dt < \varepsilon^2.
%\end{equation*}
%Let  $\delta = \varepsilon^2/(8\|g\|_\infty)$, and $h$  be given by
\begin{equation*}
    h(t) = 
    \begin{cases} 0, &  t=0,\\  
        \text{linear}, &  0<t<0+\delta,\\ 
        g(t), & 0+\delta \leq t \leq 1-\delta,\\ 
        \text{linear}, &  1-\delta <t<1,\\ 
        0, &  t=1,
    \end{cases}
\end{equation*}
% kan lage formel for h på ax+b formel: $ -g(1-\delta) / \delta \cdot x + g(1-\delta)/ \delta $ <- dette er høyre side. $dy/dx \cdot x + b$
as can be seen from Figure 1, on the interval $[0+\delta, 1-\delta]$ we have $g(t)-h(t)= 0$ for all $t$, and for all other values of $t$ the 'disance' is at worst 
\begin{equation*}
    \left|g(t)-h(t) \right| \leq |g(t)| + |-h(t)| \leq \|g \|_{\infty} + \|g \|_{\infty} = 2 \|g \|_{\infty}
\end{equation*}
by first using the triangle inequality, and then that
\begin{equation*}
    |h(t)| \leq \|h\|_{\infty} = \sup_{t\in[0,1]} |h(t)| = \sup_{t\in[0+\delta, 1-\delta]} |h(t)| = \sup_{t\in[0+\delta, 1-\delta]} |g(t)| \leq \sup_{t\in[0, 1]} |g(t)| =\| g\|_{\infty}
\end{equation*}
with equality in the second step from the fact that $h$ is linear and goes towards zero as it approaches $0$ or $1$, and therefore attains its greatest value on the interval $[0+\delta,1-\delta]$. Now,
\begin{align*}
    \| g-h \|_{L^2}^2 &=  \int_0^{0+\delta} \left|g(t)-h(t) \right|^2dt + \int_{0+\delta}^{1-\delta} \left|g(t)-h(t) \right|^2dt +\int_{1-\delta}^{1} \left|g(t)-h(t) \right|^2dt\\ 
    &\leq \int_0^{0+\delta} (2 \| g\|_\infty)^2dt + 0 +\int_{1-\delta}^{1} (2 \| g\|_\infty)^2dt\\
    &=  4 \delta \| g\|_\infty^2 + 4 \delta \| g\|_\infty^2\\ 
    &= \varepsilon^2
\end{align*}

% * PART 3   % construct vs. define
Recall the unit circle $\mathbb{T}$ where each point $z \in \mathbb{T}$ is given by $z= e^{2\pi i t}$ for $t \in \mathbb{R}$, and the 1-periodicity of the values $z$.  That is, all values of $t$ that differ by an integer amount correspond to the same point $z$. Using this, we construct a function $H: \mathbb{T} \longrightarrow \mathbb{C}$ of  $z\in\mathbb{T}$ by $H(e^{2 \pi i t})=h(t)$ for $t\in[0,1]$. Since $h \in C_{\text{per}}[0,1]$ we have that $H \in C(\mathbb{T})$. That is, since $h$ is continuous and satisfies $h(0) = h(1)$, it implies that $H$ is continuous on $\mathbb{T}$.
%\begin{lemma}
%    The collection 
%    \begin{equation*} 
%        A = \operatorname{span}(\{z^n\})_{n\in \mathbb{Z}} = \left\{ \sum_{n=-N}^{N} c_nz^n : N\in\mathbb{N}, c_n \in \mathbb  {C} \right\}
%    \end{equation*}
%    is dense in $C(\mathbb{T})$.
%\end{lemma}
%by the Stone weierstrass theorem 0.
%overgang fra $\mathbb{N}$ til $\mathbb{Z}$
%// \begin{theorem}[Weierstrass Approximation Theorem \cite{durenInvitationClassicalAnalysis2012}]
%//     Let $q(x)$ be continuous on the interval $[-\pi, \pi]$, with $q(-\pi)=q(\pi)$. Then for each $\varepsilon>0$ there is a trigonometric polynomial $T(x)$ such that $|q(x)-T(x)|<\varepsilon$ for $-\pi \leq x \leq \pi$.
%// \end{theorem}
\begin{theorem}[Weierstrass Approximation Theorem \cite{durenInvitationClassicalAnalysis2012}]
    Let $q(x)$ be continuous on the interval $[0, 1]$, with $q(0)=q(1)$. Then for each $\varepsilon>0$ there is a trigonometric polynomial $T(x)$ such that $|q(x)-T(x)|<\varepsilon$ for $0 \leq x \leq 1$.
\end{theorem}

% Lemma 1 implies that there exist a complex polynomial 
Now using the Weierstrass Approximation theorem with $q(x)= H(z)$ and $T(x) = P(z)$ we have that there exist a complex polynomial
\begin{equation*}
    P(z)= \sum_{n=-N}^{N}c_n z^n, \quad \text{where }z\in \mathbb{T},    
\end{equation*}
such that $| H(z)-P(z) | < \varepsilon$. Set $p(t) =P(e^{2\pi it})$ and observe that
%\begin{equation*}
%    \| H-P \| \leq \| H-P \|_{\infty} = \sup_{z\in \mathbb{T}} |H(z)-P(z)| < \varepsilon.
%\end{equation*}
% using that that $\operatorname{span}(z^n)_{n\in \mathbb{Z}}$ is dense in $C(\mathbb{T})$. Set $p(t) =P(e^{2\pi it})$ and observe that
\begin{equation*}
P(z)= P(e^{2\pi it})= \sum_{n=-N}^{N}c_n e^{2\pi int}, \quad \text{where }t\in [0,1],
\end{equation*}
gives an element of $\operatorname{span} \left( \left\{ e_{\lambda} \right\}_{\lambda\in \mathbb{Z}} \right)$ with $| h(t)-p(t) | < \varepsilon$.
%\begin{equation*}
%    \| h-p \|_{\infty} = \sup_{t\in [0,1]} |h(t)-p(t)| < \varepsilon.
%\end{equation*}
%However, we need to measure distance using the $L^2$-norm and not the uniform norm. 
And in using the $L^2$-norm we have
%\begin{align*}
%    \| h-p \|_{L^2([0,1])}^2 =& \int_0^1 \left|h(t)-p(t) \right|^2dt, \\ 
%    \leq &\int_0^1 \left\|h(t)-p(t) \right\|_{\infty}^2dt,\\
%    \leq &\int_0^1 \varepsilon^2 dt,\\ 
%    =& \varepsilon^2.
%\end{align*}
\begin{align*}
    \| h-p \|_{L^2([0,1])}^2 =& \int_0^1 \left|h(t)-p(t) \right|^2dt, \\ 
    \leq &\int_0^1 \varepsilon^2 dt,\\ 
    =& \varepsilon^2.
\end{align*}
Now using the triangle inequality to combine the above estimates, we get
\begin{align*} 
    \| f-p\|_{L^2([0,1])} \quad \leq&\quad  \| f-g\|_{L^2([0,1])} + \| g-h\|_{L^2([0,1])} +\| h-p\|_{L^2([0,1])}\\
    <& \quad \varepsilon + \varepsilon +\varepsilon \\
    =& \quad 3 \varepsilon
\end{align*}
For this reason, we can approximate every $f\in L^2([0,1])$ by a function $p \in \operatorname{span} \left( \left\{ e_{\lambda} \right\}_{\lambda\in \mathbb{Z}} \right)$ as closely as we want. In other words,  $\operatorname{span} \left( \left\{ e_{\lambda} \right\}_{\lambda\in \mathbb{Z}} \right)$ is dense in $L^2([0,1])$, and therefore $\left\{ e_{\lambda} \right\}_{\lambda\in \mathbb{Z}}$ is complete. %By theorem 2 $\left\{ e_{\lambda} \right\}_{\lambda\in \mathbb{Z}}$ is an orthonormal basis for $L^2([0,1])$. 



\end{document}