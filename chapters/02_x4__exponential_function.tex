\documentclass[../thesis.tex]{subfiles}
% Separate preamble for this subfile. This preamble is loaded last, so one can override various functions before \begin{document}

% Better comment extension for Vscode colors these comments differently
% Normal comment color
% * Important information
% ! ALERT
% ? Question
% TODO stuff to do
% // This is strikethrough


\begin{document}

Given $\lambda \in \R^d$, the complex exponential function $e_{\lambda}: \R^d \rightarrow \C$ with frequency $\lambda$ is 
\begin{equation*}
    e_{\lambda}(t) = e^{2 \pi i \inpl{t}} = \cos{\bracMed{2 \pi \braaMed{\lambda,t}}} + i \sin{\bracMed{2 \pi \braaMed{\lambda,t}}}, \quad t\in \mathbb{R}^d.
\end{equation*}
Here, $\braaMed{\lambda, t}$ is the euclidean inner product in \cref{def:dot_prod}. Throughout this thesis we denote by $E\brac{\Lambda}$ the exponential system
\begin{align*}
    %E\brac{\Lambda} = \braq{e_\lambda}_{\lambda \in \Lambda} = \braq{e^{2\pi i \lambda t } : \lambda \in \Lambda}
    % E\brac{\Lambda} = \braqMed{e_\lambda}_{\lambda \in \Lambda} = \braqMed{e^{2\pi i \inpl{t}} : \lambda \in \Lambda}.
    %*DENNNEE\brac{\Lambda} =& \braqMed{e_\lambda}_{\lambda \in \Lambda} = \braqMed{e^{2\pi i \inpl{t}}}_{\lambda \in \Lambda} = \braqMed{e^{2\pi i \inpl{t}} : \lambda \in \Lambda}.\\
    %E\brac{\Lambda} =& \braq{e_\lambda}_{\lambda \in \Lambda} = \braq{e^{2\pi i \inpl{t}}}_{\lambda \in \Lambda}= \braq{e^{2\pi i \inpl{t}} : \lambda \in \Lambda}.
    E\brac{\Lambda} = \braqMed{e_\lambda(t) : \lambda \in \Lambda} = \braqMed{e^{2\pi i \braaMed{\lambda,t}} : \lambda \in \Lambda}, \quad t\in \R^d,
\end{align*}
\SigridChange{where $\Lambda$ is a a subset of $\R^d$}. We will think of $E(\Lambda)$ as a system of functions in $L^2(\Omega)$ for some subset $\Omega \subseteq \R^d$. In this setting, we intuitively understand $e_\lambda$ to be $e_\lambda$ restricted to $\Omega$. % that is $\indicatorNoVar{\Omega}$

\begin{example}
    In the case where $d=1$ and $\Lambda = \Z$ we have what some authors refer to as \emph{the complex trigonometric system} \cite{heilMetricsNormsInner2018} \cite{encyclopediaofmathematicsTrigonometricSystem},
    \begin{align*}
        %E\brac{\Z} = \braqMed{e_n}_{n \in \Z} = \braqMed{e^{2\pi i t n} : n \in \Z}, %\quad t\in \R.
        %E\brac{\Z} = \braqMed{e_n : n \in \Z} = \braqMed{e^{2\pi i t n} : n \in \Z}, \quad t\in \R.
        E\brac{\Z} &= \braqMed{e^{2\pi i t n} : n \in \Z}, \quad t\in \R.
        %E\brac{\Z} &= \braqMed{e_n : n \in \Z}\\
        %E\brac{\Z} &= \braqMed{e_n(t) : n \in \Z}\quad t\in \R.
    \end{align*}
    Note that for an arbitrary $n \in \Z$ we have
    \begin{equation*}
        \bralMed{e_{n}(t) }= \bralMed{e^{2 \pi i n t}} = 1 \quad \text{for all } t\in \R.
    \end{equation*}
    Furthermore, observe that $e_n$ is 1-\emph{periodic}, because
    \begin{equation*}
        e_n(t+1) = e^{2 \pi i n (t+1)} = e^{2 \pi i n t} e^{2 \pi i n} = e^{2 \pi i n t} = e_n(t).
    \end{equation*}
    %The 1-periodicity of $e_n$ is important. If we know values of $e_n(t)$ for $t$ on an interval of length $1$, we know the values for all points $t\in \mathbb{R}$. For example, this allows us to restrict our domain from all of $\mathbb{R}$ to just the $t$'s that lie on the interval $[0,1]$, the unit interval. 
    If we know the values of $e_n(t)$ for $t$ on an interval of length $1$, we know its values for all points $t\in \mathbb{R}$. %This allows us to restrict our domain from all of $\mathbb{R}$ to the unit interval $[0,1]$.
\end{example}
%Furthermore, let $\mathbb{T}=\{z\in \mathbb{C}: |z|=1\}$ be the unit circle in the complex plane and note that each point $z\in\mathbb{T}$ is on the form $z=e^{2 \pi it}$ for $t$ a real number 



\end{document}