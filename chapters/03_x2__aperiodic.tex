\documentclass[../thesis.tex]{subfiles}
% Separate preamble for this subfile. This preamble is loaded last, so one can override various functions before \begin{document}

% Better comment extension for Vscode colors these comments differently
% Normal comment color
% * Important information
% ! ALERT
% ? Question
% TODO stuff to do
% // This is strikethrough


\begin{document}
%
%
%\begin{remark}
%    Nevertheless, by arranging unit cubes into a \emph{new} tile, one can get an aperiodic tiling for all $d\geq3$, as done in \cite{lagariasKellerCubetilingConjecture1992}. (Max: Given that it is actually true that this organization/new tile/ of unit cubes EXCLUSIVELY tiles non-periodically. Not known to me yet). The fact that the new tile only consists of unit cubes in a particular arrangement is not equivalent to the statement that there \emph{are} aperiodic cube tilings in all $d\geq3$ as stated in \cite{lagariasOrthonormalBasesExponentials2000,liuUniformityNonUniformGabor2003}, as one considers a tiling which, in fact, is \textsc{not} a cube, only consisting of cubes; and by the very definition of aperiodicity means that it must exclusively tile non-periodically, which \emph{cube tilings} intrinsically do not as highlighted above. Furthermore, \cite{lagariasOrthonormalBasesExponentials2000} provides no source for this "well-known" statement, and \cite{liuUniformityNonUniformGabor2003} makes a reference to \cite{lagariasKellerCubetilingConjecture1992}, but the explanation here is both insufficient and unclear, as this result is \textsc{never} stated anywhere. (Max: If I am wrong with this, then I cannot read. This can very well be the case when it comes to math, but I am pretty certain of this latter sentence. It can be true that the result is implied by the paper or somehow obvious in a way I cannot understand, but that is not a clear and sufficient statement of the result from my undergraduate perspective.)
%\end{remark}
\end{document}