\documentclass[../thesis.tex]{subfiles}
% Separate preamble for this subfile. This preamble is loaded last, so one can override various functions before \begin{document}

% Better comment extension for Vscode colors these comments differently
% Normal comment color
% * Important information
% ! ALERT
% ? Question
% TODO stuff to do
% // This is strikethrough


\begin{document}
%
%
%\begin{remark}
%    Nevertheless, by arranging unit cubes into a \emph{new} tile, one can get an aperiodic tiling for all $d\geq3$, as done in \cite{lagariasKellerCubetilingConjecture1992}. (Max: Given that it is actually true that this organization/new tile/ of unit cubes EXCLUSIVELY tiles non-periodically. Not known to me yet). The fact that the new tile only consists of unit cubes in a particular arrangement is not equivalent to the statement that there \emph{are} aperiodic cube tilings in all $d\geq3$ as stated in \cite{lagariasOrthonormalBasesExponentials2000,liuUniformityNonUniformGabor2003}, as one considers a tiling which, in fact, is \textsc{not} a cube, only consisting of cubes; and by the very definition of aperiodicity means that it must exclusively tile non-periodically, which \emph{cube tilings} intrinsically do not as highlighted above. Furthermore, \cite{lagariasOrthonormalBasesExponentials2000} provides no source for this "well-known" statement, and \cite{liuUniformityNonUniformGabor2003} makes a reference to \cite{lagariasKellerCubetilingConjecture1992}, but the explanation here is both insufficient and unclear, as this result is \textsc{never} stated anywhere. (Max: If I am wrong with this, then I cannot read. This can very well be the case when it comes to math, but I am pretty certain of this latter sentence. It can be true that the result is implied by the paper or somehow obvious in a way I cannot understand, but that is not a clear and sufficient statement of the result from my undergraduate perspective.)
%\end{remark}

In this section, we will show the following result. 
\SigridComment{ or }

The topic for this \namecref{sec:aperi_cube} is a "new" result first put forward by Lagarias (, Reeds, and Wang??) in \cite{lagariasOrthonormalBasesExponentials2000}, however no proof or reference to the literature was attached \SigridChange{(needs rephrasing!)}. The statement has later been referenced in \cite{liuUniformityNonUniformGabor2003}, although with incorrect reference. After correspondence with Lagarias himself, the main result of this \namecref{sec:aperi_cube} was submitted as a problem for the American Math Monthly. 

\SigridComment{not sure how to intro this}

Correct citation?
\cite{haugeTitle}

\begin{theorem}
    There exists a translational tiling of unit cubes $I^d + \lambda$ in $\R^d$ for $\lambda\in \Lambda$ in all dimensions $d\geq3$, in which the tiling set $\Lambda$ constitutes an aperiodic tiling. 
\end{theorem}

For convenience, we will label cubes in the tiling by their corner coordinates which will be given by the vector $x = \brac{x_1,\dots,x_d} \in \R^d$. Note that it is essential that all cubes are labeled using the same coordinate. Furthermore, we will denote a unit cube that is parallel to an axis by
\begin{equation*}
    \mathcal{C}\bracMed{x_1,\dots,x_d} = \braqMed{(x_1,\dots,x_d) + (t_1,\dots,t_d) : 0 \leq t_j \leq 1}
\end{equation*} 
where $x\in \R^d$ is the specified corner coordinate. We say that this is an \emph{axis-paralell unit cube}. 

The proof will provide a specific way to construct an aperiodic tiling of unit cubes in any dimension from a fully periodic tiling of unit cubes in $\R^d$. More specifically, we will consider a lattice tiling of unit cubes. As such, all cube corners will initially be in the integer lattice. From this, we will create a new tiling denoted by $T_n$ by the following translation. This insures
%* BLABLA FINAL REMARK NOTE
%* and final sentence "exatly two one dimensional " bla bla



Before generalizing it to higher dimensions, we prove the result for dimension $d=3$. Here we displace two columns of cubes with corner coordinates 
\begin{align*}
    \bracMed{m_1,0,1},& \quad m_1 \in \Z,\\
    \bracMed{0,1,m_3},& \quad m_3 \in \Z.
\end{align*}
and moving each of them with the respective translation vectors $w_1 = \brac{\frac{1}{2},0,0}$ and $w_3 = \brac{0,0,\frac{1}{2}}$ so that
\begin{align*}
    \bracMed{m_1,0,1} \xlongrightarrow[]{w_1} \bracMed{m_1 + \frac{1}{2} ,0,1}, \quad  &\text{ for all } m_1 \in \Z,\\
    \bracMed{0,1,m_3} \xlongrightarrow[]{w_3} \bracMed{0,1,m_3+\frac{1}{2}},    \quad  &\text{ for all }  m_3 \in \Z.
\end{align*}
Observe that the column of cubes covers exactly the same set after the translation since the translation is in the same direction as the column itself 
\begin{align*}
    \braqMed{ \bracMed{x_1, 0+t_2,1+t_3} : 0 \leq t_j \leq 1, x_1\in\R} \xlongrightarrow[]{w_1} & \braqMed{ \bracMed{x_1, 0+t_2,1+t_3} : 0 \leq t_j \leq 1, x_1\in\R},\\
    \braqMed{ \bracMed{0+t_1,1+t_2,x_3} : 0 \leq t_j \leq 1, x_3\in\R} \xlongrightarrow[]{w_3} & \braqMed{ \bracMed{0+t_1,1+t_2,x_3} : 0 \leq t_j \leq 1, x_3\in\R}.
\end{align*}
In addition, note that both shifted columns are \emph{disjoint} in $\R^3$ since their middle corner coordinates $x_2$ differ by a non-zero integer. Hence, the new tiling $T_3$ is still a unit cube tiling of $\R^3$. To show that $T_3$ is an aperiodic tiling, consider the following. Let $v=\brac{a_1,a_2,a_3}\in \R^3$ be a vector which we associate with the translation $\mathbf{T}_v:T_3 \rightarrow T_3$ (correct?? )
\begin{equation*}
    \mathbf{T}_v \bracMed{x_1,x_2,x_3} = \bracMed{x_1+a_1, x_2+a_2, x_3+a_3} 
\end{equation*}
where $\brac{x_1,x_2,x_3}$ is the point which the translation acts on. If the translation maps cube-corners to cube-corners, then the tiling will be invariant under $\mathbf{T}_v$. We now show that $v=\brac{0,0,0}$ is the only vector that preserves cube-corners and makes $\mathbf{T}_v$ invariant (map?). 

We first show that the translation vector $v\in \Z^3$. Consider a $2\times 2\times 2$ block of $8$ unit cubes in $T_3$, which do not intersect any part of our translated columns. Note that all cubes in our block have integer coordinates. When translating the block, its image under the translation must also be a $2\times 2\times 2$ block of $8$ unit cubes in $T_3$. Since our shifted columns are disjoint, there will be at most $4$ unit cubes in our shifted columns, and the rest must still have integer coordinates. As such, $v$ maps an integer vector to another integer vector. That is, $v$ maps some cube-corner from the initial block to a cube-corner in the image block, which has an integer corner vector. Hence, $v\in \Z$. 

We now show that the translation vector $v\in \brac{0,0,0}$. 


%*  Gammel Avslutnint.
%* Evnt ha en fotnonte som forklarer kontekts?

Hence we can state the following result using the above as proof

\begin{theorem}
    There exists a translational tiling of unit cubes $I^d + \lambda$ in $\R^d$ for $\lambda\in \Lambda$ in all dimensions $d\geq3$, in which the tiling set $\Lambda$ constitutes an aperiodic tiling. 
\end{theorem}

\end{document}