\documentclass[../thesis.tex]{subfiles}
% Separate preamble for this subfile. This preamble is loaded last, so one can override various functions before \begin{document}

% Better comment extension for Vscode colors these comments differently
% Normal comment color
% * Important information
% ! ALERT
% ? Question
% TODO stuff to do
% // This is strikethrough


\begin{document}
%
%
%\begin{remark}
%    Nevertheless, by arranging unit cubes into a \emph{new} tile, one can get an aperiodic tiling for all $d\geq3$, as done in \cite{lagariasKellerCubetilingConjecture1992}. (Max: Given that it is actually true that this organization/new tile/ of unit cubes EXCLUSIVELY tiles non-periodically. Not known to me yet). The fact that the new tile only consists of unit cubes in a particular arrangement is not equivalent to the statement that there \emph{are} aperiodic cube tilings in all $d\geq3$ as stated in \cite{lagariasOrthonormalBasesExponentials2000,liuUniformityNonUniformGabor2003}, as one considers a tiling which, in fact, is \textsc{not} a cube, only consisting of cubes; and by the very definition of aperiodicity means that it must exclusively tile non-periodically, which \emph{cube tilings} intrinsically do not as highlighted above. Furthermore, \cite{lagariasOrthonormalBasesExponentials2000} provides no source for this "well-known" statement, and \cite{liuUniformityNonUniformGabor2003} makes a reference to \cite{lagariasKellerCubetilingConjecture1992}, but the explanation here is both insufficient and unclear, as this result is \textsc{never} stated anywhere. (Max: If I am wrong with this, then I cannot read. This can very well be the case when it comes to math, but I am pretty certain of this latter sentence. It can be true that the result is implied by the paper or somehow obvious in a way I cannot understand, but that is not a clear and sufficient statement of the result from my undergraduate perspective.)
%\end{remark}

In this section, we will show the following result. 
\SigridComment{ or }

The topic for this \namecref{sec:aperi_cube} is a "new" result first put forward by Lagarias (, Reeds, and Wang??) in \cite{lagariasOrthonormalBasesExponentials2000}, however no proof or reference to the literature was attached \SigridChange{(needs rephrasing!)}. The statement has later been referenced in \cite{liuUniformityNonUniformGabor2003}, although with incorrect reference. After correspondence with Lagarias himself, the main result of this \namecref{sec:aperi_cube} was submitted as a problem for the American Math Monthly. 

\SigridComment{not sure how to intro this}

Correct citation?
\cite{haugeTitle}

\begin{theorem}
    There exists a translational tiling of unit cubes $I^d + \lambda$ in $\R^d$ for $\lambda\in \Lambda$ in all dimensions $d\geq3$, in which the tiling set $\Lambda$ constitutes an aperiodic tiling. 
\end{theorem}

For convenience, we will label cubes in the tiling by their corner coordinates which will be given by the vector $x = \brac{x_1,\dots,x_d} \in \R^d$. Note that it is essential that all cubes are labeled using the same coordinate. Furthermore, we will denote a unit cube that is parallel to an axis by
\begin{equation*}
    \mathcal{C}\bracMed{x_1,\dots,x_d} = \braqMed{(x_1,\dots,x_d) + (t_1,\dots,t_d) : 0 \leq t_j \leq 1}
\end{equation*} 
where $x\in \R^d$ is the specified corner coordinate. We say that this is an \emph{axis-paralell unit cube}. 

The proof will provide a specific way to construct an aperiodic tiling of unit cubes in any dimension from a fully periodic tiling of unit cubes in $\R^d$. More specifically, we will consider a lattice tiling of unit cubes. As such, all cube corners will initially be in the integer lattice. From this, we will create a new tiling denoted by $T_n$ by the following translation. This insures
BLABLA FINAL REMARK NOTE



Before generalizing it to higher dimensions, we prove the result for dimension $d=3$. 




%*  Gammel Avslutnint.
%* Evnt ha en fotnonte som forklarer kontekts?

Hence we can state the following result using the above as proof

\begin{theorem}
    There exists a translational tiling of unit cubes $I^d + \lambda$ in $\R^d$ for $\lambda\in \Lambda$ in all dimensions $d\geq3$, in which the tiling set $\Lambda$ constitutes an aperiodic tiling. 
\end{theorem}

\end{document}