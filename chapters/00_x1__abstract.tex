




%\thispagestyle{plain}

%* Option 2
\begin{center}\textsf{\textbf{\large Abstract}}\end{center} 

Let $\Omega=I^d$ denote the $d$-dimensional unit cube in $\R^d$. Consider the set of exponential functions $e_\lambda(t)$ for $\lambda \in \Lambda$ where $\Lambda$ is some subset of $\R^d$. When these functions form an orthogonal basis for the Hilbert space of $L^2$-functions on $\Omega$, we say that $(\Omega,\Lambda)$ is a spectral pair. Furthermore, if the set of translates $\Lambda'$ tile $\R^d$ using $\Omega$, we say that $(\Omega,\Lambda')$ is a tiling pair. In this thesis, we study the following special case of Fugeledes conjecture which has been proven true in all dimensions by two independent papers: Let $\Omega\subset\R^d$, then $(I^d,\Lambda)$ is a spectral pair if and only if $(I^d,\Lambda)$ is a tiling pair. We will show this result for $d\leq2$ and, in doing so, also classify all spectra and tiling pairs in these dimensions. Related to this, we will also prove Keller's theorem. Last and most important, in joint work with J. Lagarias, we prove a new construction on aperiodic tilings. 

\begin{center}\textsf{\textbf{\large Sammendrag}}\end{center} 

La $\Omega=I^d$ angi den $d$-dimensionale enhetskube i $\R^d$. Ta i betraktning mengden med eksponensialfunksjoner $e_\lambda(t)$ for $\lambda \in \Lambda$ hvor $\Lambda$ er en undermengde av $\R^d$. Når disse funksjonene er en ortogonal basis for Hilbert rommet med $L^2$-funksjonene på $\Omega$, sier vi at $(\Omega,\Lambda)$ er et spektralpar. Videre, hvis mengden med translasjoner $\Lambda'$ flislegger $\R^d$ med $\Omega$, sier vi at $(\Omega,\Lambda')$ er et flisleggingspar. I oppgaven studerer vi det følgende spesialtilfelle av Fugeledes formodning som har blitt bevist i alle dimensjoner av to uavhengige artikler: La $\Omega\subset\R^d$, da vil $(I^d,\Lambda)$ være et spektralpar hvis og bare hvis $(I^d,\Lambda)$ er et flisleggingspar. Vi vil vise dette resultatet for $d\leq2$, og med det også klassifisere alle spektral og flisleggingspar for disse dimensjonene. Relatert til dette vil vi også bevise Keller's teorem. Sist og blant det viktigste, i sammarbeid med J. Lagarias vil vi bevise en ny konstruksjon av en aperiodisk flislegging.  


 