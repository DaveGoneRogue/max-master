




\thispagestyle{plain}

%* Option 2
\begin{center}\textsf{\textbf{\large Abstract}}\end{center} 

Lorem ipsum dolor sit amet, consectetur adipiscing elit, sed do eiusmod tempor incididunt ut labore et dolore magna aliqua. Accumsan lacus vel facilisis volutpat est velit egestas. Senectus et netus et malesuada fames ac turpis egestas. Molestie at elementum eu facilisis sed odio morbi. Id interdum velit laoreet id donec.

\begin{center}\textsf{\textbf{\large Sammendrag}}\end{center} 

Lorem ipsum dolor sit amet, consectetur adipiscing elit, sed do eiusmod tempor incididunt ut labore et dolore magna aliqua. Accumsan lacus vel facilisis volutpat est velit egestas. Senectus et netus et malesuada fames ac turpis egestas. Molestie at elementum eu facilisis sed odio morbi. Id interdum velit laoreet id donec.


 
%* J&P
In this thesis we shal consider the set of exponential functions $e_\lambda(t)$ on a set 

Let $\Omega\subset \R^d$ be a subset with finite and positive measure, and consider the corresponding Hilbert space of $L^2$-functions on $\Omega$. 

There is a well known conjecture that states that 

that $(\Omega,\Lambda)$ is a spectral pair if and only if the translates of some set $\Omega'$ by the vectors of $\Lambda$ tile $\R^d$. 

We will investigte this conjecture in the special case where $\Omega=I^d$, the $d$-dimensional unit cube, and show this conjecture for $d\leq2$. 

We will also prove kelers conjecture, and show the conjecture for all d when $\Lambda$ has the added asumption to be periodic. 


%* Max
conjecture For which two independent proofs have shown a solution

%* L&R
classifying all spectra for the unit cube by showing that L is a spectrum if and only if BLABLA is a tiling of $\R^d$ by translates of unit cubes. 

%* 
Let BLABLA denote the unit cube in BLABLA, and consider a discrete subset $\Lambda\subset \R^d$. We show that the exponentials bLABLA, form an ONB for L2 OMega, if the translates form a tiling of $\R^d$. 


%* Max two
Let $\Omega=I^d$ denote the $d$-dimensional unit cube in $\R^d$. Consider the set of exponential functions $e_\lambda(t)$ for $\lambda \in \Lambda$ where $\Lambda$ is some subset of $\R^d$. When these functions form an orthogonal basis for the Hilbert space of $L^2$-functions on $\Omega$, then we say that $(\Omega,\Lambda)$ is a spectral pair. There is a conjecture thatt states BLABLA.
We will show this conjecture for BLABLA, as well as BLABLA.

