\documentclass[../thesis.tex]{subfiles}
% Seperate preamble for this subfile. This preamble is loaded last, so it may be used to override various functions.

% Better comment extension for Vscode colors these comments differently
% Normal comment color
% * Important information is highlighted
% ! ALERT
% ? Question
% TODO stuff to do
% // this is strikethrough


\begin{document}


intro %til jorgensen og pedersen om construction for høyere dim. 


In \cite{jorgensenSpectralPairsCartesian2001} Steen P. and Palle E. presented a method on the construction of spectral pairs in higher dimensions. It is a recursive technique in what they dub "a cross-product construction" using "factors" in lower dimensions, and applies for any two spectral pairs $\brac{\Omega_i,\Lambda_i}$ where $i=1,2$ and in arbitrary dimensions $d_1$ and $d_2$. The technique is presented in the following theorem

\begin{theorem}[Construction of spectra, \cite{jorgensenSpectralPairsCartesian2001}] \label{thrm:construction_spectra} %! SKRIV OM, er nesten reeen copy paste
    Let $\brac{\Omega_1,\Lambda_1}$ be a spectral pair in dimension $d_1$, and let $\Omega_2$ be a set of positive finite measure in dimension $d_2$. Suppose that for each $\lambda_1 \in \Lambda_1$, that $\Lambda_2=\Lambda(\lambda_1)$ is a discrete subset of $\mathbb{R}^{d_2}$ such that $\brac{\Omega_2,\Lambda_2}$ is a spectral pair.

    If $\Lambda=\braq{\brac{\lambda_1,\lambda_2}: \lambda_1\in \Lambda_1, \lambda_2 \in \Lambda_2}$ then $\brac{\Omega_1\times\Omega_2, \Lambda}$ is a spectral pair in $d_1+d_2$ dimensions. 
\end{theorem}
%[p.~3]
\begin{proof}
    $\dots$
    $\dots$
\end{proof}


\begin{example}
    Example of using \cref{thrm:construction_spectra} to construct a spectral pair for $d=2$ dimensions. 
    We have shown that $\brac{\Omega_1,\Lambda_1} = \brac{I, \Z}$ is a spectral pair in dimension $d_1=1$. Clearly $\brac{\Omega_2,\Lambda_2} = \brac{I, \Z}$ is a spectral pair in dimension $d_2=1$. Now, since $\Lambda  = \braq{\brac{\lambda_1,\lambda_2}\in \Z \times \Z }$ we have that $\brac{I \times I, \Lambda}$ is a spectral pair in $1+1=2$ dimensions.

    
\end{example}










\end{document}