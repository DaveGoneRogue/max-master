\documentclass[../thesis.tex]{subfiles}
% Separate preamble for this subfile. This preamble is loaded last, so it may be used to override various functions.

% Better comment extension for Vscode colors these comments differently
% Normal comment color
% * Important information is highlighted
% ! ALERT
% ? Question
% TODO stuff to do
% // this is strikethrough


\begin{document}
In \cite{jorgensenSpectralPairsCartesian2001}, Pedersen and Jorgensen present a way of constructing spectral pairs in higher dimensions from spectral pairs in lower dimensions. 

The method closely resembles a cross-product construction of vectors in higher dimensions using vectors from lower dimensions. It applies to any pair of spectral pairs in arbitrary dimensions.  %! MK. Hva menes her
\begin{theorem}[Construction of spectra]\label{thrm:construction_spectra}
    Let $\brac{\Omega_1,\Lambda_1}$ be a spectral pair in dimension $d_1$, and let $\Omega_2$ be a set of positive finite measure in dimension $d_2$. Suppose for each $\lambda_1 \in \Lambda_1$ that $\Lambda_2=\Lambda(\lambda_1)$ is a discrete subset of $\R^{d_2}$ such that $\brac{\Omega_2,\Lambda_2}$ is a spectral pair. If 
    \begin{equation}
        %\Lambda = \braq{\lambda: \lambda \in  \Lambda_1 \times \Lambda_2} = \braq{\brac{\lambda_1, \lambda_2}: \lambda_1\in \Lambda_1, \lambda_2 \in \Lambda_2} 
        \Lambda = \braq{\brac{\lambda_1, \lambda_2}: \lambda_1\in \Lambda_1, \lambda_2 \in \Lambda_2} 
    \end{equation}
    then $\brac{\Omega_1\times\Omega_2, \Lambda}$ is a spectral pair in $d_1+d_2$ dimensions. 
\end{theorem}
% * In \cite{jorgensenSpectralPairsCartesian2001} Steen P. and Palle E. presented a method on the construction of spectral pairs in higher dimensions. It is a recursive technique in what they dub "a cross-product construction" using "factors" in lower dimensions, and applies for any two spectral pairs $\brac{\Omega_i,\Lambda_i}$ where $i=1,2$ and in arbitrary dimensions $d_1$ and $d_2$. The technique is presented in the following theorem
%[p.~3]
\textcolor{green}{kommentar: 1. er det klundrete å vite hvilken $\Lambda$, eller mer spesifikt, $E(\Lambda)$ man ser på her. 2. Er det klundrete for folk å vite at $t$ er følgende vektor $t = (t_1,t_2)$ med $\Omega_1$ and $t_2 \in \Omega_2$}

\begin{proof}
    We start by showing that the set of exponentials $E(\Lambda)$
    %\begin{equation}
    %    E(\Lambda) = \braq{e^{2\pi i \braa{\lambda, t} } : \lambda \in \Lambda_1\times\Lambda_2}
    %\end{equation} 
    % given in \cref{eq:exp_set_all_d}
    is pairwise orthogonal in $L^2\brac{\Omega_1 \times \Omega_2}$. Given two elements $e_\lambda,e_{\lambda'} \in E(\Lambda)$ we have %the following inner product %Note that $t$ denotes the vector $(t_1,t_2)$ where $t_1 in \Omega_1$ and $t_2 \in \Omega_2$
    \begin{align*}
        \braa{e_\lambda,e_{\lambda'}}_{L^2\brac{\Omega_1 \times \Omega_2}} 
        &= \int_{\Omega_1} \int_{\Omega_2} e_{\lambda}(t) \overline{e_{\lambda'}(t)} dt_2 dt_1\\ 
        &= \int_{\Omega_1} \int_{\Omega_2} e^{2\pi i \braa{\lambda, t} } e^{-2\pi i  \braa{\lambda', t}} dt_2 dt_1\\ 
        &= \int_{\Omega_1} \int_{\Omega_2} e^{2\pi i \brac{\lambda_1 t_1+\lambda_2 t_2}} e^{-2\pi i\brac{\lambda_1' t_1+\lambda_2' t_2}} dt_2 dt_1\\ 
        &= \int_{\Omega_1} \int_{\Omega_2} e^{2\pi i \brac{\lambda_1- \lambda_1'}t_1} e^{2\pi i \brac{\lambda_2 - \lambda_2'}t_2} dt_2 dt_1\\ 
        &= \int_{\Omega_1} e^{2\pi i  \brac{\lambda_1- \lambda_1'}t_1} \bracMed{\int_{\Omega_2}  e^{2\pi i \brac{\lambda_2 - \lambda_2'}t_2} dt_2} dt_1
    \end{align*}
    %! Start på på vanskelig å følge, MUntlig kommentar, må skrives om
    Observe the following for $\lambda_1, \lambda_1' \in \Lambda_1$. If $\lambda_1 \neq \lambda_1'$, then the final integral vanishes since $\brac{\Omega_1, \Lambda_1}$ is a spectral pair. If $\lambda_1 = \lambda_1'$, then the final integral vanishes if $\lambda_2 \neq \lambda_2'$ since $\brac{\Omega_2, \Lambda_2}$ is a spectral pair. However, if $\lambda_2 = \lambda_2'$, we have $\lambda = \lambda'$, and that 
    \begin{align*}
        \braa{e_\lambda, e_\lambda}_{L^2\brac{\Omega_1 \times \Omega_2}}
        &= \int_{\Omega_1} \int_{\Omega_2} e_{\lambda}(t) \overline{e_{\lambda}(t)} dt_2 dt_1
        =\int_{\Omega_1} \int_{\Omega_2} |e^{2 \pi i \lambda t}|^2 dt_2 dt_1
        = \int_{\Omega_1} \int_{\Omega_2} |1|^2 dt_2 dt_1\\
        &= \mes{\Omega_1}\mes{\Omega_2} \neq 0
    \end{align*}
    %! Slutt på vanskelig å følge
    To show that $E(\Lambda)$ is complete in $L^2(\Omega_1 \times \Omega_2)$, let $f\in L^2\brac{\Omega_1 \times \Omega_2}$ and assume it is orthogonal to $\spn{E(\Lambda)}$. For all $e_\lambda \in E(\Lambda)$ the inner product factors as
    \begin{align*} % REMEMBER t is a vector, i.e t=(t_1,t_2)
        \braa{e_\lambda,f}_{L^2\brac{\Omega_1 \times \Omega_2}}
        &= \int_{\Omega_1} \int_{\Omega_2} e_\lambda(t) \overline{f(t)} dt_2dt_1 \\
        &= \int_{\Omega_1} \int_{\Omega_2} e^{2\pi i  (\lambda_1 t_1 + \lambda_2 t_2)} \overline{f(t_1,t_2)} dt_2 dt_1 \\
        &= \int_{\Omega_1} e^{2 \pi i \lambda_1 t_1} \int_{\Omega_2}e^{2 \pi i \lambda_2 t_2} \overline{f(t_1,t_2)} dt_2 dt_1 \\
        %&= 0
    \end{align*}
    If we fix $\lambda_2$, we can denote the inner integral as 
    \begin{equation}\label{eq:inner_eq}
        F(t_1) := \int_{\Omega_2} e^{2 \pi i \lambda_2 t_2} \overline{f(t_1,t_2)} dt_2,
    \end{equation}
    %and rewrite the rest as
    and note that $F\in L^2(\Omega_1)$. We have 
    \begin{equation}\label{eq:outter_eq}
        \braa{e_{\lambda}, f}_{L^2(\Omega_1\times \Omega_2)} = \int_{\Omega_1} e^{2 \pi i \lambda_1 t_1} F(t_1) dt_1 := \braa{e_{\lambda_1}, F}_{L^2(\Omega_1)}.
    \end{equation}
    Using the fact that $E(\Lambda_1)$ is complete in $L^2(\Omega_1)$, we have from \cref{lem:ONB_alternative_def} that $F(t_1)=0$ for almost every $t_1 \in \Omega_1$. Now as we must have \labelcref{eq:inner_eq} equal to zero, observe that since $\lambda_2$ was arbitrarily, and we know that $E(\Lambda_2)$ is complete in $L^2(\Omega_2)$, then the same lemma implies that $f=0$. Thus, $E(\Lambda)$ is complete in $L^2\brac{\Omega_1 \times \Omega_2}$ which finalizes the proof.
    %Using the fact that $E(\Lambda_1)$ is complete in $L^2(\Omega_1)$, we have from \cref{lem:ONB_alternative_def} and almost all $t_1 \in \Omega_1$, that the only element in $L^2(\Omega_1)$ that results in $\braa{e_{\lambda_1}, F} = 0$ for all $e_{\lambda_1} \in E(\Lambda_1)$ is the zero-element $F=0$. Now as we must have \cref{eq:inner_eq} equal to zero, observe that as we fixed $\lambda_2$ arbitrarily, and we know that $E(\Lambda_2)$ is complete in $L^2(\Omega_2)$, then the same lemma also implies $f=0$. Thus, $E(\Lambda)$ is complete in $L^2\brac{\Omega_1 \times \Omega_2}$ which finalizes our proof.
    %For almost all $t_1 \in \Omega_1$ \cref{eq:outter_eq}
\end{proof}

Before continuing, we will apply \cref{thrm:construction_spectra} to some simple examples to illustrate its usefulness.
The simplest case to apply \cref{thrm:construction_spectra}

We will show some illustrative examples to show that in this
In fact, we can use \cref{thrm:construction_spectra} to show the following. 

A useful application and illustrative example of \cref{thrm:construction_spectra} is to apply it to the one-dimensional case of the unit cube.

\begin{example}\label{exmp:first_construction}
    Let $\brac{\Omega_1,\Lambda_1}=\brac{I, \Z}$, which we know is a spectral pair in dimension $d_1=1$. Clearly $\brac{\Omega_2,\Lambda_2} = \brac{I, \Z}$ is also a spectral pair in dimension $d_2=1$. Now, letting 
    \begin{equation*}
        \Lambda  = \braq{\brac{\lambda_1,\lambda_2}\in \Z \times \Z },
    \end{equation*}
    it follows directly from \cref{thrm:construction_spectra} that $\brac{I \times I, \Lambda}$ is a spectral pair in $1+1=2$ dimensions.
\end{example}

\mycomment{ %! My Comment 
\begin{example}
    Let $\brac{\Omega_1,\Lambda_1} = \brac{I^2, \Lambda}$, where $\Lambda  = \braq{\brac{\lambda_1,\lambda_2}\in \Z \times \Z }$.
    Using either the spectral pair $\brac{\Omega_2,\Lambda_2} = \brac{I, \Z}$ in dimension $d_2=1$ or $\brac{\Omega_2,\Lambda_2} = \brac{I \times I, \Lambda}$ in dimension $d_2=2$, we can again use \cref{thrm:construction_spectra} to directly show that both
    $\brac{I \times I \times I, \Lambda'}$ and $\brac{I \times I \times I \times I, \Lambda''}$ is a spectral pair in $2+1=3$ and $2+2=4$ dimensions respectivly. Here $\Lambda'=\braq{\brac{\lambda_1,\lambda_2,\lambda_3}\in \Z \times \Z \times \Z}$ and $\Lambda''=\braq{\brac{\lambda_1,\lambda_2,\lambda_3,\lambda_4}\in \Z \times \Z \times \Z \times \Z}$
\end{example}
}

\begin{example}\label{exmp:second_construction}
    Let $\brac{\Omega_1,\Lambda_1} = \brac{I^2, \Lambda}$, where $\Lambda  = \braq{\brac{\lambda_1,\lambda_2}\in \Z \times \Z }$.
    Using the same spectral set $\brac{\Omega_2,\Lambda_2} = \brac{I, \Z}$ as in \cref{exmp:first_construction}, we can again use \cref{thrm:construction_spectra} to directly show that $\brac{I^2 \times I, \Lambda'}$ is a spectral pair in $2+1=3$ dimensions with 
    \begin{equation*}
        \Lambda'=\braq{\brac{\lambda_1,\lambda_2}\in \Z^2 \times \Z}.
    \end{equation*}
    Similarly, we can show that $\brac{I^2 \times I^2, \Lambda''}$ is a spectral pair in $2+2=4$ dimensions with
    \begin{equation*}
        \Lambda''=\braq{\brac{\lambda_1,\lambda_2}\in \Z^2 \times \Z^2} \qedhere
    \end{equation*}
\end{example}

This recursive way of constructing spectra for higher dimensions is quite useful. Recall our unproven \cref{lem:z_d_in_higer_d}. It is easy to see that applying \cref{thrm:construction_spectra} \emph{ad infinitum} can be used as a proof for the \cref{lem:z_d_in_higer_d}.
%and also allows us to take the preceding examples as a proof for our unproven. 
%Recall our unproven \cref{lem:z_d_in_higer_d}. It is easy to see that the 
%We could continue doning this all the way up 
%for which the preceding argument is a proof of our unproven \cref{lem:z_d_in_higer_d}. 
%Recalling our unproven \cref{lem:z_d_in_higer_d} one can 

\labelcref{exmp:first_construction} and \labelcref{exmp:second_construction} are both examples of lattice spectra. This becomes evident when looking at FIGUREXX, which illustrates the grid-like pattern of a lattice. However, the higher dimensions allow for some newfound flexibility in that we can have spectra that are not lattices. 

\end{document}