\documentclass[../thesis.tex]{subfiles}
% Separate preamble for this subfile. This preamble is loaded last, so one can override various functions before \begin{document}

% Better comment extension for Vscode colors these comments differently
% Normal comment color
% * Important information
% ! ALERT
% ? Question
% TODO stuff to do
% // this is strikethrough


\begin{document}

Before tackling spectral sets in higher dimensions, we first emphasize the importance of \cref{eq:zero_set_equation} and generalize it to higher dimensions. 
\begin{lemma}\label{lem:zero_set_jp_1_5}
    Let $\Omega=I^d$. The zero-set for the function
    \begin{equation*}
        F_{\Omega} (z):= \int_\Omega e^{2 \pi i \braa{z,t}} dt
    \end{equation*}
    is the set
    \begin{equation*}
        \Zstroke_{I^d} = \braqMed{ z = \brac{z_1,\dots,z_d}\in \C^d \setminus \braqMed{0} : \exists \space j \in \braqMed{1,\dots, d} \text{ such that } z_j \in  \intnozero}
    \end{equation*}
\end{lemma}

\mycomment{ %! BLOKK kommentar
\begin{example}
    The simplest zero-set for $F_{\Omega}$ is when $d=1$, which is
    \begin{equation*}
        %\Zstroke_{I} = \braqMed{ z \in \C \setminus \braqMed{0} : \exists \space z \in  \intnozero} = \braq{z: z \in \intnozero}
        %\Zstroke_{I} = \braqMed{z: z \in \intnozero} \qedhere
        \Zstroke_{I} = \intnozero \qedhere
    \end{equation*}
\end{example}
}

\begin{proof} %! [Proof of  \cref{lem:zero_set_jp_1_5}] % Hvis proof med navn
    Given a vector $z \in C^d$ we can factor $F_{\Omega}$ for $\Omega = I^d$ as follows
    \begin{align*}
        F_{I^d} (z) &= \int_{I^d} e^{2 \pi i \braa{z,t}} dt\\
        &= \int_{I^d} e^{2\pi i  (z_1 t_1 + \dots +z_d t_d)} dt\\
        &= \int_{I^d} e^{2\pi i z_1 t_1}  \cdots e^{2\pi i z_d t_d}  dt\\
        &= \int_{I} e^{2\pi i z_1 t_1} \int_{I} e^{2\pi i z_2 t_2}  \cdots \int_{I} e^{2\pi i z_d t_d}  dt_1 dt_2\dots dt_d\\
        &=\prod_{j=1}^d \int_I e^{2\pi i z_j t_j} dt_j
    \end{align*}
    If $z=0$, then $e^{2\pi i z_j t_j} = 1$ and
    \begin{align*}
        F_{I^d} (0) = \prod_{j=1}^d \int_I 1 \space dt_j = \brac{\mes{I}}^d = 1.
    \end{align*}
    Thus we must have $z\in \C^d\setminus\braq{0}$. Assume now that $z\neq 0$. Integration yields
    \begin{align*}
        F_{I^d} (z) = \prod_{j=1}^d \frac{e^{2 \pi i z_j} -1}{2 \pi i z_j}.
    \end{align*}
    This equation is zero whenever at least one element $z_j\in \intnozero$ using that the exponential equals one in this case, and also not to get an undefined fraction. Thus, the zero set for $F_{I^d}$ is $\Zstroke_{I^d}$ which completes the proof.
\end{proof}

\begin{remark}
    Given a spectral pair $(I^d,\Lambda)$, consider the set
    \begin{equation*}
        \Lambda - \Lambda = \braqMed{\lambda-\lambda' : \lambda,\lambda' \in \Lambda}.
    \end{equation*}
    It follows that the spectral pair property is equivalent to the non-zero elements of $\Lambda - \Lambda$ being contained in the zero-set of $F_{\Omega}$ and that the set $E(\Lambda)$ is complete in $L^2(\Omega)$. More specifically, if $(I^d,\Lambda)$ is a spectral pair, then
    \begin{equation*} %! En veis implikasjon !! spectral pair ->> DEtte stemmer. Lignende resultat for dette er Keller, og er ikke trivielt
        \Lambda - \Lambda \subset \Zstroke_{I^d} \cup \braq{0}.
    \end{equation*}
    This follows immediately from the observation that
    \begin{equation*}
        \braaMed{e_{\lambda},e_{\lambda'} }_{L^2(I^d)} = \int_\Omega e^{2 \pi i \braa{(\lambda-\lambda'),t}} dt = F_{\Omega} (\lambda-\lambda'),
    \end{equation*}
    %which is zero whenever $\lambda \neq \lambda'$, that is $\lambda-\lambda'\neq 0$, which imply $\lambda-\lambda' \in \Z_{I^d}$
    which is zero whenever $\lambda \neq \lambda'$, which implies $\lambda-\lambda' \in \Zstroke_{I^d}$
\end{remark}

\begin{example}
    As we have verified in \labelcref{eq:zero_set_equation}, the zero-set for $F_{I}$ is  $\Zstroke_{I} = \intnozero$. Furthermore, when $\Lambda=\Z$, the non-zero elements of $\Lambda - \Lambda$ is simply $\intnozero$. then clearly
    \begin{equation*}
        \intnozero \subset \Zstroke_{I} \cup \braq{0} = \Z.
    \end{equation*}
    Since we have shown $E(\Z)$ is complete in $L^2(I)$, this shows that $\brac{I,\Z}$ is a spectral pair.
\end{example}

%* ———————————————————————————————————————— Del 2, construction of spectra
In \cite{jorgensenSpectralPairsCartesian2001}, Pedersen and Jorgensen present a way of constructing spectral pairs in higher dimensions from spectral pairs in lower dimensions. 

\begin{theorem}[Construction of spectra]\label{thrm:construction_spectra}
    Let $\brac{\Omega_1,\Lambda_1}$ be a spectral pair in $\R^{d_1}$, and let $\Omega_2$ be a set of positive finite measure in $\R^{d_2}$. Suppose for each $\lambda_1 \in \Lambda_1$ that $\lambfunc$ is a discrete subset of $\R^{d_2}$ such that $\brac{\Omega_2,\lambfunc}$ is a spectral pair. If 
    \begin{equation*}
        \Lambda = \braq{\brac{\lambda_1, \lambda_2}: \lambda_1\in \Lambda_1, \lambda_2 \in \lambfunc} 
    \end{equation*}
    then $\brac{\Omega_1\times\Omega_2, \Lambda}$ is a spectral pair in $\R^{d_1+d_2}$ dimensions. 
\end{theorem}

\begin{remark}
    It can be helpful to think of $\lambfuncNoVar$ as a function that assigns a discrete subset in $\R^{d_2}$ to each $\lambda_1$-value such that $\brac{\Omega_2,\lambfunc}$ is itself a spectral pair. %However, $\lambfunc$ should never be considered anything other than a set of points. 
\end{remark}

\begin{proof}[Proof of  \cref{thrm:construction_spectra}] %* Meningen med to mellomrom før \cref her for bedre spacing:)
    Given $\Lambda$ as constructed in \cref{thrm:construction_spectra}, we start by showing that the set of exponentials $E(\Lambda)$ is orthogonal in $L^2\brac{\Omega_1 \times \Omega_2}$. For two elements $e_\lambda,e_{\lambda'} \in E(\Lambda)$ we have %the following inner product %Note that $t$ denotes the vector $(t_1,t_2)$ where $t_1 in \Omega_1$ and $t_2 \in \Omega_2$
    \begin{align*}
        \braa{e_\lambda,e_{\lambda'}}_{L^2\brac{\Omega_1 \times \Omega_2}} 
        &= \int_{\Omega_1} \int_{\Omega_2} e_{\lambda}(t) \overline{e_{\lambda'}(t)} dt_2 dt_1\\ 
        &= \int_{\Omega_1} \int_{\Omega_2} e^{2\pi i \braa{\lambda, t} } e^{-2\pi i  \braa{\lambda', t}} dt_2 dt_1\\ 
        &= \int_{\Omega_1} \int_{\Omega_2} e^{2\pi i \brac{\lambda_1 t_1+\lambda_2 t_2}} e^{-2\pi i\brac{\lambda_1' t_1+\lambda_2' t_2}} dt_2 dt_1\\ 
        &= \int_{\Omega_1} \int_{\Omega_2} e^{2\pi i \brac{\lambda_1- \lambda_1'}t_1} e^{2\pi i \brac{\lambda_2 - \lambda_2'}t_2} dt_2 dt_1\\ 
        &= \int_{\Omega_1} e^{2\pi i  \brac{\lambda_1- \lambda_1'}t_1} \bracMed{\int_{\Omega_2}  e^{2\pi i \brac{\lambda_2 - \lambda_2'}t_2} dt_2} dt_1.
    \end{align*}
    If $\lambda_2 \neq \lambda_2'$, then as $\lambda_2, \lambda_2' \in \lambfunc$ the inner integral will be zero from the fact that $\brac{\Omega_2, \lambfunc}$ is a spectral pair. Conversely, if $\lambda_2 = \lambda_2'$ the resulting integral will factor as % THE inner integral will be equal to $\mes{\Omega_2}$, and the resulting integral will be
    \begin{align*}
        = \mes{\Omega_2} \int_{\Omega_1} e^{2\pi i  \brac{\lambda_1- \lambda_1'}t_1} dt_1.
    \end{align*}
    Now, if one assumes $\lambda_1 \neq \lambda_1'$, then as both $\lambda_1, \lambda_1' \in \Lambda_1$ the integral will be zero from the fact that $\brac{\Omega_1, \Lambda_1}$ is a spectral pair. However, if $\lambda_1 = \lambda_1'$, observe that we have the case where $\lambda = \lambda'$, and
    \begin{align*}
        \braa{e_\lambda, e_\lambda}_{L^2\brac{\Omega_1 \times \Omega_2}}
        &= \int_{\Omega_1} \int_{\Omega_2} e_{\lambda}(t) \overline{e_{\lambda}(t)} dt_2 dt_1
        =\int_{\Omega_1} \int_{\Omega_2} |e^{2 \pi i \braa{\lambda, t}}|^2 dt_2 dt_1
        = \int_{\Omega_1} \int_{\Omega_2} |1|^2 dt_2 dt_1\\
        &= \mes{\Omega_2}\mes{\Omega_1} \neq 0
    \end{align*}
    To show that $E(\Lambda)$ is complete in $L^2(\Omega_1 \times \Omega_2)$, let $f\in L^2\brac{\Omega_1 \times \Omega_2}$ and assume it is orthogonal to $\spn{E(\Lambda)}$. For all $e_\lambda \in E(\Lambda)$ the inner product is
    \begin{align*} % REMEMBER t is a vector, i.e t=(t_1,t_2)
        \braa{e_\lambda,f}_{L^2\brac{\Omega_1 \times \Omega_2}}
        &= \int_{\Omega_1} \int_{\Omega_2} e_\lambda(t) \overline{f(t)} dt_2dt_1 \\
        &= \int_{\Omega_1} \int_{\Omega_2} e^{2\pi i  (\lambda_1 t_1 + \lambda_2 t_2)} \overline{f(t_1,t_2)} dt_2 dt_1 \\
        &= \int_{\Omega_1} e^{2 \pi i \lambda_1 t_1} \int_{\Omega_2}e^{2 \pi i \lambda_2 t_2} \overline{f(t_1,t_2)} dt_2 dt_1
        %&= 0
    \end{align*}
    If we fix $\lambda_2$, we denote the inner integral by \SigridComment{Fiksere $\lambda_2$ er unødvendig? må ikke da F, være en funksjon av $\lambda_2$ også?}
    \begin{equation}\label{eq:inner_eq}
        F(t_1) := \int_{\Omega_2} e^{2 \pi i \lambda_2 t_2} \overline{f(t_1,t_2)} dt_2,
    \end{equation}
    and note that $F\in L^2(\Omega_1)$. We thus have % This allows us to rewrite our initial inner product into
    \begin{equation*}%\label{eq:outter_eq}
        \braa{e_{\lambda}, f}_{L^2(\Omega_1\times \Omega_2)} = \int_{\Omega_1} e^{2 \pi i \lambda_1 t_1} F(t_1) dt_1 = \braa{e_{\lambda_1}, F}_{L^2(\Omega_1)} = 0, \quad \text{ for all } \lambda_1 \in \Lambda_1 .
    \end{equation*}
    Using the fact that $E(\Lambda_1)$ is complete in $L^2(\Omega_1)$, we have from \cref{lem:ONB_alternative_def} that $F(t_1)=0$ for almost every $t_1 \in \Omega_1$. Now as we must have \labelcref{eq:inner_eq} equal to zero a.e. observe that since $\lambda_2$ was arbitrarily, and we know that $E(\lambfunc)$ is complete in $L^2(\Omega_2)$, then using \cref{lem:ONB_alternative_def} again implies that $f=0$ a.e. Thus, $E(\Lambda)$ is complete in $L^2\brac{\Omega_1 \times \Omega_2}$ which finalizes the proof.
    %Using the fact that $E(\Lambda_1)$ is complete in $L^2(\Omega_1)$, we have from \cref{lem:ONB_alternative_def} and almost all $t_1 \in \Omega_1$, that the only element in $L^2(\Omega_1)$ that results in $\braa{e_{\lambda_1}, F} = 0$ for all $e_{\lambda_1} \in E(\Lambda_1)$ is the zero-element $F=0$. Now as we must have \cref{eq:inner_eq} equal to zero, observe that as we fixed $\lambda_2$ arbitrarily, and we know that $E(\Lambda_2)$ is complete in $L^2(\Omega_2)$, then the same lemma also implies $f=0$. Thus, $E(\Lambda)$ is complete in $L^2\brac{\Omega_1 \times \Omega_2}$ which finalizes our proof.
    %For almost all $t_1 \in \Omega_1$ \cref{eq:outter_eq}
\end{proof}

%* ———————————————————————————————————————— Del 3, Eksempler på theoremet
Let us now give some examples using \cref{thrm:construction_spectra} to illustrate its ease of use and newfound flexibility. 
\begin{example}\label{ex:first_construction}
    Let $\brac{\Omega_1,\Lambda_1}=\brac{I, \Z}$, which we know is a spectral pair in dimension $d_1=1$. Clearly $\brac{\Omega_2,\lambfunc}$ is also a spectral pair in dimension $d_2=1$ if $\Omega_2=I$ and $\lambfunc = \Z$ for all $\lambda_1 \in \Lambda_1$. Now, letting 
    \begin{equation}\label{eq:first_construction}
        \Lambda  = \braqMed{\brac{\lambda_1,\lambda_2}: \lambda_1 \in \Z, \lambda_2 \in \lambfunc },
    \end{equation}
    it follows directly from \cref{thrm:construction_spectra} that $\brac{I^2, \Lambda}$ is a spectral pair in $1+1=2$ dimensions.
\end{example}

Recall our unproven \cref{lem:z_d_in_higer_d}. It is easy to see that applying \cref{thrm:construction_spectra} $d$ times will provide a proof of this result. \cref{ex:first_construction} is an example of a lattice spectrum, but the increase in dimension allows for more flexibility in that we can also construct spectra that are not lattices. %This becomes evident when looking at FIGUREXX, which illustrates the grid-like pattern of a lattice. 
\begin{example}\label{ex:single_shift_vertical}
    For instance, letting $\Omega_1=\Omega_2 = I$ in \cref{thrm:construction_spectra}, we can choose $\Lambda_1 = \Z$ and 
    \begin{align}\label{eq:single_shift_vertical}
        \lambfunc = \begin{cases}        
            \Z & \text{for all } \lambda_1 \in \Lambda_1\setminus\braq{\lambda_1'},\\        
            \Z+\beta & \text{if } \lambda_1=\lambda_1',\text{ where } \beta \in \mathbb{R}.   
        \end{cases}
    \end{align}
    Then the resulting spectrum for $I^2$ will be the same as \labelcref{eq:first_construction} in \cref{ex:first_construction} but with $\lambfuncNoVar$ now given by \labelcref{eq:single_shift_vertical}. This spectrum is illustrated in \cref{fig:multiple_shift_vertical} with $\beta = 0.5$.
    %\begin{equation}\label{eq:single_shift_vertical_spectrum}
    %    \Lambda = \braqMed{\brac{\lambda_1,\lambda_2}: \lambda_1 \in \Z, \lambda_2 \in \lambfunc},
    %\end{equation}
    %which is illustrated in \cref{fig:single_shift_vertical} with $\beta = 0.5$.
\end{example} %* for the two spectral pairs $\brac{I,\Z}$ and $\brac{I,\lambfunc}$. NEI, dette er et valg av L(.) mer enn et spektrum.  og i den grad det er et spektrum så er jo hvert enkelt L(.) et spektrum for I.

Furthermore, there is no reason only to consider one shift for one $\lambda_1'$ value as done in \cref{ex:single_shift_vertical}. 
\begin{example}\label{ex:multiple_shift_vertical}
    Again, letting $\Omega_1=\Omega_2 = I$ in \cref{thrm:construction_spectra}, we can choose $\Lambda_1 = \Z$ and 
    \begin{align}\label{eq:multiple_shift_vertical}
        \lambfunc = \begin{cases}
            & \text{ \space \space }\vdots\\        
            \Z+0 & \text{for } \lambda_1 = -1 \in \Lambda_1\\
            \Z+0.3 & \text{for } \lambda_1 = \text{\space\space\space} 0 \in \Lambda_1,\\
            \Z+0.8 & \text{for } \lambda_1 = \text{\space\space\space} 1 \in \Lambda_1,\\
            \Z+0.5 & \text{for } \lambda_1 = \text{\space\space\space} 2 \in \Lambda_1,\\
            \Z+0.9 & \text{for } \lambda_1 = \text{\space\space\space} 3 \in \Lambda_1,\\
            \Z-0.3 & \text{for } \lambda_1 = \text{\space\space\space} 4 \in \Lambda_1,\\
            & \text{ \space \space }\vdots
        \end{cases}
    \end{align}
    Then $\Lambda$ in \labelcref{eq:first_construction} is the resulting spectrum for $I^2$ with $\lambfuncNoVar$ now given by \labelcref{eq:multiple_shift_vertical}. This spectrum is illustrated in \cref{fig:multiple_shift_vertical}. Note that there is no reason to consider shifts $\beta \geq 1$ as we can always write the shift as some value $n+\beta$ where $n\in \Z$ and $\beta \in [0,1)$.
\end{example}

\begin{remark}
    Moreover, we can, in addition to choosing $\Omega_1=\Omega_2 = I$ and $\lambfuncNoVar$ given by \labelcref{eq:multiple_shift_vertical} as in \cref{ex:multiple_shift_vertical} also shift $\Lambda_1$, with, for example, $\alpha = 0.4$. %as illustrated in \cref{fig:multiple_shift_horizontal}.
    At last, we can also shift the roles of $\Lambda_1$ and $\Lambda_2$ to get a new spectrum. That is,
    \begin{equation*}
        \Lambda = \braqMed{\bracMed{\lambda_1, \lambda_2}: \lambda_1 \in \lambfuncGen{\lambda_2}, \lambda_2 \in \Lambda_2 },
        %\Lambda = \braqMed{\begin{pmatrix}\lambda_1 \\ \lambda_2 \end{pmatrix}: \lambda_1 \in \Z, \lambda_2 \in \lambfunc}
        %\Lambda = \begin{pmatrix}\lambfuncGen{\lambda_2} \\ \Lambda_2 \end{pmatrix}
    \end{equation*}
    %The latter of which is illustrated in \cref{fig:multiple_shift_horizontal}.
    which is illustrated in \cref{fig:multiple_shift_horizontal}.
\end{remark}

As has been observed, increasing the dimension by one from one opens up several new possibilities in our choices for the spectrum. This is most evident in our many choices for $\lambfuncNoVar$. To capture the flexibility illustrated in \cref{ex:first_construction,ex:single_shift_vertical,ex:multiple_shift_vertical}, we introduce the following \namecref{lem:beta_shift}.
%! Ikke whitespace i multiple references.

%! FIGURE INPUT



\begin{figure}[t]%h!
    \centering
    \begin{subfigure}{.47\textwidth}
        \centering
        \includegraphics[width=0.9\linewidth]{spec_no_shift.jpg}
        \caption{Lattice spectra}
        \label{fig:lattice_spectra}
    \end{subfigure}\quad
    \begin{subfigure}{.47\textwidth}
        \centering
        \includegraphics[width=0.9\linewidth]{spec_single_shift.jpg}
        \caption{Single shift vertical}
        \label{fig:single_shift_vertical}
    \end{subfigure}\\
    \begin{subfigure}{.47\textwidth}
        \centering
        \includegraphics[width=0.9\linewidth]{multiple_shift_left_zero.jpg}
        \caption{Multiple individual shifts vertical}
        \label{fig:multiple_shift_vertical}
    \end{subfigure}\quad
    \begin{subfigure}{.47\textwidth}
        \centering
        \includegraphics[width=0.9\linewidth]{multiple_shift_left_zero_horizontal.jpg}
        \caption{Multiple individual shifts horizontal}
        \label{fig:multiple_shift_horizontal}
    \end{subfigure}
    \caption{Illustration of the following pair of spectral pairs. In \labelcref{fig:lattice_spectra} we have $\brac{I,\Z}$ and $\brac{I,\lambfunc}$.  In \labelcref{fig:single_shift_vertical} we have $\brac{I,\Z}$ and $\brac{I,\lambfunc}$, where $\lambfunc$ is given by \labelcref{eq:single_shift_vertical}. In \labelcref{fig:multiple_shift_vertical} we have $\brac{I,\Z}$ and $\brac{I,\lambfunc}$, where $\lambfunc$ is given by \labelcref{eq:multiple_shit_func}. In \labelcref{fig:multiple_shift_horizontal} we have $\brac{I,\Z+0.4}$ and $\brac{I,\lambfunc}$, where $\lambfunc$ is given by \labelcref{eq:multiple_shit_func}.}
    \label{fig:spectra_figures}
\end{figure}


%* ———————————————————————————————————————— Del 4, Lemma som fanger den nye fleksibiliteten
\begin{lemma}\label{lem:beta_shift}
    Let $\brac{\Omega_1,\Lambda_1}$ and $\brac{\Omega_2,\Lambda_2}$ be spectral pairs in dimensions $d_1$ and $d_2$ respectivly. Furthermore, define an arbitrary function $\betafunc: \Lambda_1 \rightarrow \R^{d_2}$, and let
    % $\betafunc: \Lambda_1 \rightarrow \R^{d_2}$ to denote the shift in the spectrum for $\Omega_2$ at each $\lambda_1$-value, and let %* unødvendig?
    \begin{equation}
        \Lambda = \Lambda_\beta
        = \braqMed{\begin{pmatrix}
            \lambda_1 \\
            \lambda_2+\betafunc(\lambda_1)
            \end{pmatrix}
        : \lambda_1 \in \Lambda_1 \text{ and } \lambda_2 \in \Lambda_2}.
    \end{equation}
    Then $\brac{\Omega_1 \times \Omega_2,\Lambda_\beta}$ is a spectral pair in $d_1+d_2$ dimensions. 
\end{lemma}
%\begin{proof} %! No proof needed here
%    As we assumed that $\brac{\Omega_2,\Lambda_2}$ is a spectral pair, it follows directly from \cref{lem:spectrum_shift_is_spectrum} that $\brac{\Omega_2,\Lambda_2+\beta}$ is also a spectral pair for any $\beta \in \R^{d_2}$ (with elements populated by $\betafunc(\lambda_1)$)
%\end{proof}
\begin{remark}
    We can also shift the spectrum for $\Omega_1$ by a vector $\alpha \in \R^{d_1}$, to obtain the spectrum 
    \begin{equation*}
        \Lambda = \Lambda_{\alpha,\beta} 
        = \braqMed{\begin{pmatrix}
            \lambda_1 +\alpha\\
            \lambda_2+\betafunc(\lambda_1)
            \end{pmatrix}
        : \lambda_1 \in \Lambda_1 \text{ and } \lambda_2 \in \Lambda_2}
    \end{equation*}
    for $\Omega_1 \times \Omega_2$.
\end{remark}
%* Her representerer "funksjonen" vi hadde over, både en plaserings funksjon på \lambda_1, i tillegg til å skifte spektrumet akk. her med en reel verdi 
%* Funksjonen beta, poppulerer vektoren beta med elementer som er gitt av funksjonen. 

\begin{example}
    For two spectral sets $\brac{\Omega_1,\Lambda_1}=\brac{I, \Z+\alpha}$ and $\brac{\Omega_2,\Lambda_2}=\brac{I, \Z}$ a direct application of \cref{lem:beta_shift} would imply that $\brac{I^2,\Lambda_{\alpha,\beta}}$ is a spectral pair where
    \begin{equation*}
        \Lambda_{\alpha,\beta} 
        = \braqMed{\begin{pmatrix}
            \lambda_1 +\alpha\\
            \lambda_2+\betafunc_2(\lambda_1)
            \end{pmatrix}
        : \lambda_1 \in \Z +\alpha \text{ and } \lambda_2 \in \Z}. \qedhere
    \end{equation*}
\end{example}


\begin{example}
    By repeated application of \cref{lem:beta_shift} it will follow that $\brac{\Omega,\Lambda}= \brac{I^d, \Lambda}$ is a spectral pair if $\Lambda$ is given by % where the spectrum $\Lambda$ is the set of points given by
    \begin{equation*}
        \begin{pmatrix}
            \lambda_1 +\alpha\\
            \lambda_2+\betafunc_2(\lambda_1)\\
            \lambda_3+\betafunc_3(\lambda_1,\lambda_2)\\
            \vdots\\
            \lambda_d+\betafunc_d(\lambda_1,\dots,\lambda_{d-1})\\
        \end{pmatrix}
    \end{equation*}
    where all the elements $\lambda_1, \dots, \lambda_d \in \Z$ and each $\betafunc_{i+1}:\Z^{i} \rightarrow [0,1)$ are fixed functions.
    %* Skriv om fra JP; It is clear that the modifications result from the permutation of the $d$ coordinates. 
    %! Trengs det egt? Bedre å ikke nevne?
\end{example}

%* ———————————————————————————————————————— Del 5, Klassefisering av alle
%As with \cref{prop:class_all_shift}, we can classify all spectra in dimension two as follows. This significant result was proven by Pedersen and Jorgensen in the same paper \cite{jorgensenSpectralPairsCartesian2001}. 
We can now classify all spectra in dimension two as follows (\cite{jorgensenSpectralPairsCartesian2001}). 
\begin{theorem}\label{thrm:class_all_shift_2d}
    %$\braq{\beta_m \in [0,1) : m \in \Z}$  %* ANDRE skrivemåten for beta_m
    %For a fixed value $\alpha \in \R$ and an infinite sequence $\brac{\beta_m}_{m\in \Z}$, $\beta_m \in [0,1)$,  %* gammel
    The only subsets $\Lambda \subset \R^2$ such that $\Lambda$ is a spectrum for $\Omega = I^2$ are either one of the two classes
    \begin{equation}\label{eq:2d_all_shift_class_1}
        \Lambda = \braqMed{
            \begin{pmatrix}
            m + \alpha \\
            n + \beta_m
            \end{pmatrix} : m,n \in  \Z
            }
    \end{equation}
    or
    \begin{equation}\label{eq:2d_all_shift_class_2}
        \Lambda = \braqMed{
            \begin{pmatrix}
            m + \beta_n\\
            n + \alpha
            \end{pmatrix} : m,n \in  \Z
            }
    \end{equation}
    where $\alpha \in \R$ is fixed and $\brac{\beta_m}_{m\in \Z}$ is an infinite sequence with $\beta_m \in [0,1)$.
    %! \SigridComment{ekskludert delen om at det også er et tiling set, det er vel hensiktsmessik og splitte disse opp, og komme tilbake til det i neste del, for å så komme med fulle og hele theorem 3.2.} 
\end{theorem}
%!remark on the half open interval\\
%!remark om hvorfor sequencen er essensiell i dette tilfellet for å klassifisere alle skiftene\\ %, samenlignet bare med bare noen random skift
%!remark om at vi ikke kan ha ulike skift i to retninger, skiftene i den ene retningen må være fiksert.\\

\begin{proof}
    It follows directly from \cref{lem:beta_shift} that both classes given by \labelcref{eq:2d_all_shift_class_1} and \labelcref{eq:2d_all_shift_class_2} make $\brac{I^2,\Lambda}$ a spectral pair. To show that there are no other possibilities for $\Lambda$, we will make use of the inclusion
    \begin{equation}\label{eq:inclusion_dim_2}
        \Lambda - \Lambda \subset \Zstroke_{I^2} \cup \braq{0}
    \end{equation}
    where the zero-set $\Zstroke_{I^2}$ given in \cref{lem:zero_set_jp_1_5} is
    \begin{equation*}\label{eq:zeroset_dim_2}
        \Zstroke_{I^2} = \braqMed{z = \brac{z_1,z_2} \in \C^2 : \exists \space j \in \braq{1,2} \text{ such that } z_j \in \intnozero}.
    \end{equation*}
    To begin, let $\brac{I^2,\Lambda}$ be a spectral pair, which we know also implies that $\Lambda$ satisfies \labelcref{eq:inclusion_dim_2}. In addition, let us assume that the zero-element $\brac{0,0}$ is in $\Lambda$, as we can always translate $\Lambda$ by one vector so that the zero-element is in $\Lambda$. Now, let $\lambda = \brac{\lambda_1,\lambda_2} \in \Lambda$. From \labelcref{eq:inclusion_dim_2} we must have
    \begin{equation*}%* sidelengs fordi vi ser på elementer som er implisert fra inklusjonen, uten noen antagelser på Lambda
        \bracMed{\lambda_1, \lambda_2} - \bracMed{0,0} \in \Zstroke_{I^2} \cup \braq{0} \Longrightarrow \brac{\lambda_1, \lambda_2} \in \Zstroke_{I^2} \cup \braq{0}.
    \end{equation*}%! SIGRID CHANGE INNE I HER
    That is, $\Lambda \subset \Zstroke_{I^2} \cup \braq{0}$. This means that at least one of $\lambda_1$ or $\lambda_2$ must be a non-zero integer for every $\lambda \in \Lambda$. This argument greatly reduces the possibilities as at least one of $\lambda_1$ or $\lambda_2$ must be a non-zero integer. Assume the minimum, with the added assumption that the other is anything but an integer so that we have $\lambda\notin \Z^2$. There are two ways to do this. The first possibility is to take $\lambda_1$ to be a non-zero integer and take $\lambda_2$ to be a non-integer. Now, take another arbitrary element \SigridChange{$\lambda' = \brac{\lambda_1',\lambda_2'}$} from $\Lambda$ such that $\lambda'\neq \lambda$. Again, using our assumptions, we have two ways to choose $\lambda_1'$ and $\lambda_2'$. Take $\lambda_1'$ to be a non-integer and $\lambda_2'$ to be a non-zero integer. Then, the fact that we now have 
    \begin{align*} %* husk at vi "er glad i non-zero" kun hvis vi sammelinger med null elemenntet i minusstykket $\lambda- (0,0) \in \Zstroke$.
        \brac{\underbrace{\lambda_1}_{\in \intnozero}-\underbrace{\lambda_1'}_{\notin \Z}} \notin \intnozero
        \quad \text{and} \quad
        \brac{\underbrace{\lambda_2}_{\notin \Z}-\underbrace{\lambda_2'}_{\in \intnozero}} \notin \intnozero,
    \end{align*}
    we get that $\brac{\lambda-\lambda'} \notin \Zstroke_{I^2}$ which contradicts \labelcref{eq:inclusion_dim_2}. As such, with our initial assumptions for $\lambda$, we must have the other choice for $\lambda_1'$ and $\lambda_2'$. That is, $\lambda_1'$ must be an integer for any element $\lambda'\in \Lambda$, since this implies
    \begin{align*} %*, which corresponds to the fact that the "x-axis values" distribution is an integer!
        \brac{\underbrace{\lambda_1}_{\in \intnozero}-\underbrace{\lambda_1'}_{\in \Z}} \in \intnozero
        \quad \text{and} \quad
        \brac{\underbrace{\lambda_2}_{\notin \Z}-\underbrace{\lambda_2'}_{\notin \Z}} \notin \intnozero,
    \end{align*}
    and that $\brac{\lambda-\lambda'} \in \Zstroke_{I^2}$. We now have that %! Sier vel egentlig her at $\lambda_1 = m + \alpha$
    \begin{align*}
        \Lambda = \braqMed{ \bracMed{\lambda_1, \lambda_2} : \lambda_1 \in \Z, \lambda_2 \in \R}.
    \end{align*}
    %* For et gitt punkt har jeg et heltall her (i 1. koord.). DET presser frem at den andre, med mindre det er et element i Z^2 , medfølger at alle andre elementer må ha et heltall i første koordinat. KRITISK. 
    %* DVS, forskjellige linjer på x-aksen, da trenger det ikke være en sammenheng
    %* MEN to punkter på samme linje, da skjer det ting -> da må differansen være et heltall
    Going forward, take $\lambda, \lambda' \in \Lambda$ such that $\lambda\neq \lambda'$, and assume $\lambda_1,\lambda'_1 \in \Z$ with $\lambda_1 = \lambda'_1$. Then, from \labelcref{eq:inclusion_dim_2} and the fact that we must have 
    \begin{equation*}
        \brac{\lambda_1-\lambda'_1,\lambda_2-\lambda'_2} = \brac{0,\lambda_2-\lambda'_2}\in \Zstroke_{I^2},
    \end{equation*}
    it follows that $\brac{\lambda_2- \lambda_2'} \in \intnozero$. Otherwise, we would allow for $\lambda_2-\lambda_2'=0$ which would imply that $\lambda=\lambda'$ and yield a contradiction on our assumption that $\lambda\neq\lambda'$. Note that since we still have $\lambda_2\in \R$ we can always write $\lambda_2 = n+\beta_{\lambda_1}$ where $n\in \Z$ and $\beta_{\lambda_1}\in [0,1)$ is the shift. %! NOT, shift at the point $lambda_1$ we only consider points here, and all points are shifted $\beta_m$ amount from their closest integer rounded down. So the construction of $\Lambda$ here is somehwat different from the one we have looked at earlier as we here only consider points in both directions, not how they are placed in relation to one another as the construction lemma!!
    Furthermore, interchanging the vector notation from row to column form, observe $\Lambda$ is, in fact, a subset given by \labelcref{eq:2d_all_shift_class_1}, with $\lambda_1 = m + \alpha$ and $\lambda_2 = n+\beta_m$ where $\alpha = 0$. That is,
    \begin{equation*}
        \Lambda = \braqMed{\begin{pmatrix} m \\ n+\beta_m\end{pmatrix} : m,n \in \Z} 
    \end{equation*} %for all $\lambda = \begin{pmatrix} m \\ n+\beta_m\end{pmatrix} \in \Lambda$.
    From \cref{lem:onb_direct_subset}, we know that we cannot have an orthonormal basis as a direct subset of another orthonormal basis. Thus we must have equality, and $\Lambda$ must, in this first case, be equal to \labelcref{eq:2d_all_shift_class_1}.
    %*which corresponds to the fact that the distribution of 
    %*the elements are at different $\lambda_1$ points fo, \SigridComment{har lyst til å si "x-akse", eller forskjellige linjer, men finner ikke helt riktig ordvalg her} har lyst til å si at minusstykket må være et heltall, og det er det vi vet. Hvis de er på forskjellige x-verdier er det null stress for da er integralet null kunn hvis differansen i eksponenten er et HELTALL!! som er DET MEST eSSensielle her. Er det noe annet er ekspoenenten ikke null, og da er det heller ikke en del av nullmengen.  

    %! SIGRID CHANGE INNE I HER
    Let us go back a few steps and consider the second possibility \SigridChange{for $\lambda = \brac{\lambda_1,\lambda_2}$}. That is, take $\lambda_1$ to be a non-integer and $\lambda_2$ to be a non-zero integer. Now, take another arbitrary element \SigridChange{$\lambda' = \brac{\lambda_1',\lambda_2'}$} from $\Lambda$ such that $\lambda'\neq \lambda$. We showed in the first case that we could not choose $\lambda_1'$ to be a non-integer if $\lambda_1$ was a non-zero integer. Since $\lambda_2$ now is the non-zero integer, we must have that $\lambda_2'$ is an integer for any element $\lambda'\in \Lambda$ to avoid getting a contradiction. Hence
    \begin{align*}
        \brac{\underbrace{\lambda_1}_{\notin \Z}-\underbrace{\lambda_1'}_{\notin \Z}} \notin \intnozero
        \quad \text{and} \quad
        \brac{\underbrace{\lambda_2}_{\in \intnozero}-\underbrace{\lambda_2'}_{\in \Z}} \in \intnozero,
    \end{align*}
    which gives us that $\brac{\lambda-\lambda'} \in \Zstroke_{I^2}$. We now have
    \begin{align*}
        \Lambda = \braqMed{ \bracMed{\lambda_1, \lambda_2} : \lambda_1 \in \R, \lambda_2 \in \Z}.
    \end{align*}
    As in the first case, take now $\lambda, \lambda' \in \Lambda$ such that $\lambda\neq \lambda'$, and assume $\lambda_2,\lambda'_2 \in \Z$ with $\lambda_2 = \lambda'_2$. Then, from \labelcref{eq:inclusion_dim_2} and the fact that we must have 
    \begin{equation*}
        \brac{\lambda_1-\lambda'_1,\lambda_2-\lambda'_2} = \brac{\lambda_1-\lambda'_1,0}\in \Zstroke_{I^2},
    \end{equation*}
    it follows that $\brac{\lambda_1- \lambda_1'} \in \intnozero$ to not get the contradiction that $\lambda=\lambda'$. Again, interchanging the vector notation from row to column form, observe $\Lambda$ is, a subset given by \labelcref{eq:2d_all_shift_class_2}, with $\lambda_1 = m + \beta_n$ and $\lambda_2 = n+\alpha$ where $\alpha = 0$. That is,
    \begin{equation*}
        \Lambda = \braqMed{\begin{pmatrix} m +\beta_n \\ n\end{pmatrix} : m,n \in \Z} 
    \end{equation*} %for all $\lambda = \begin{pmatrix} m \\ n+\beta_m\end{pmatrix} \in \Lambda$.
    From \cref{lem:onb_direct_subset}, we once more get that $\Lambda$ cannot be a subset of \labelcref{eq:2d_all_shift_class_2} and that we must have equality here. To conclude, $\brac{I^2, \Lambda}$ is a spectral pair in the case where $\Lambda$ is given by either \labelcref{eq:2d_all_shift_class_1} or \labelcref{eq:2d_all_shift_class_2}, and there are no other spectra for $I$ in dimension $d=2$.
\end{proof}

\end{document}