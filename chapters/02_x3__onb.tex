\documentclass[../thesis.tex]{subfiles}
% Separate preamble for this subfile. This preamble is loaded last, so one can override various functions before \begin{document}

% Better comment extension for Vscode colors these comments differently
% Normal comment color
% * Important information
% ! ALERT
% ? Question
% TODO stuff to do
% // This is strikethrough


\begin{document}
% * Def complete i form av “and the linear span of $B$ is dense in $H$ relative to the $\|\cdot\|_H$-norm, meaning $\overline{\operatorname{span}(B)} = H$. (fundamental/total)."
%intro til orthogonal bases 
%We begin this section with a short introduction to 
%We begin with the following. 
\begin{definition}[Inner product]\label{def:dot_prod}
    % Let $\lambda=(\lambda_1, \dots, \lambda_d)$ and $t=(t_1,\dots t_d)$ be two vectors in $\R^d$. The inner product, denoted $\braa{\cdot, \cdot}$, between $\lambda$ and $t$ is
    % \begin{equation*}
    %     \braaMed{\lambda,t} = \sum_{n=1}^d \lambda_n t_n= \lambda_1 t_1 + \dots + \lambda_d t_d,
    % \end{equation*}
    % also known as the dot product. 
    Let $x=(x_1, \dots, x_d)$ and $y=(y_1,\dots y_d)$ be two vectors in $\R^d$. The inner product, denoted $\braa{\cdot, \cdot}$, between $x$ and $y$ is
    \begin{equation*}
       \braaMed{x,y} = \sum_{n=1}^d x_n y_n= x_1 y_1 + \dots + x_d y_d,
    \end{equation*}
    also known as the dot product. 
\end{definition}

Additionally, the inner product on $\R^d$ induces the norm $\bran{\cdot}_2 = \sqrt{\braa{\cdot, \cdot}}$, known as the Euclidian norm or Euclidian distance. 

\begin{definition}[Orthogonal and orthonormal]
    Let $V$ be an inner product space, $I$ be an arbitrary index set, and $v_i$ denote a vector in $V$ for some $i\in I$. We have that
    \begin{itemize}%[topsep=0pt]
        \item Any two \textsc{elements} $v_i, v_j \in V$ are \emph{orthogonal}, denoted $v_i \perp v_j$, if $\braa{v_i,v_j} = 0$.
        %\item If $\braa{x_1,x_2} = 0$, then the vectors are \emph{orthogonal} to each other %$x_1$ and $_2$ are \emph{orthogonal} or \emph{pairwise orthogonal}.
        \item A \textsc{set} of vectors $\braq{v_i : i \in I}$ is \emph{orthogonal} if $\braa{v_i,v_j} = 0$ whenever $i\neq j$. % if any two distinct elements are pairwise orthogonal
        \item A \textsc{set} of vectors $\braq{v_i : i \in I}$ is \emph{orthonormal} if it is orthogonal, and each vector $v_i$ satisfies $\bran{v_i}_2 = 1$. %unit vector
        \qedhere
    \end{itemize}
\end{definition}
\begin{remark}
    Any set of non-zero orthogonal vectors can be turned into an orthonormal set by dividing each vector by its own length.
\end{remark}
%\begin{definition}
%    Let $H$ be a Hilbert space. A countable infinite orthonormal set $E = \braq{v_i: i \in I}$, which is complete in $H$, is called an \emph{orthonormal basis} for $H$.
%\end{definition}
\begin{definition}[ONB]
    Let $H$ be a Hilbert space, and let $E = \braq{v_i : i \in I}$ be a countable infinite orthonormal set. If $E$ is complete in $H$, then $E$ is an \emph{orthonormal basis} for $H$.
\end{definition}
Recall that $E$ is complete in $H$ if the linear span of $E$ is dense in $H$ with respect to the \GenNormH, meaning that every vector $x$ in $H$ can be written as the infinite series
%That is, all of the elements of $E$ are pairwise orthogonal, have norm one, and the linear span of $E$ is dense in $H$ with respect to the \GenNormH, meaning $\spnclos{E} = H$. Furthermore, the fact that the span of the basis vectors is dense implies that every vector $x$ in the space $H$ can be written as the sum of infinite series. That is, for a basis vector $v_i \in E$ we have that
\begin{equation*}%\label{eq:orthonorm_equiv}
    x= \sum_{n=1}^{\infty} \langle x, v_i \rangle v_i . %\quad \text{for every } x\in H,
\end{equation*}
The orthogonality property guarantees that the decomposition is unique. %, and in some sense, we can say that \cref{eq:orthonorm_equiv} \emph{characterizes} completeness.
An equivalent restatement of completeness is the following
%\begin{lemma}\label{lem:ONB_alternative_def} % GAMMEL
%    An orthonormal set $\left\{ e_k \right\}$ is complete in a Hilbert space $H$ if and only if $\langle x, e_k \rangle_{H} = 0$ for all $k\in B$ and some $x\in H$, then $x$ is the zero element of $H$.
%\end{lemma}
\begin{lemma}\label{lem:ONB_alternative_def}
    Let $H$ be a Hilbert space and $x\in H$ an arbitrary element. An orthonormal set $E = \braq{v_i: i \in I}$ is complete in $H$ if and only if $\braa{x, v_i}_{H} = 0$ for all $v_i\in E$ implies that $x$ is the zero-element of $H$.
    %An orthonormal set $E = \braq{v_i : i \in I}$ is complete in a Hilbert space $H$ if and only if $\braa{x, v_i}_{H} = 0$ for all $v_i\in E$ and an arbitrary $x\in H$, then $x$ is the zero-element of $H$.
\end{lemma}
\begin{proof}
    Since $E$ is an orthonormal set and $x$ is orthogonal to $E$, we must have $x$ orthogonal to the closure of the linear span of $E$. As the closure of the linear span of $E$ is all of $H$, that is $\spnclos{E} = H$, this implies that $x=0$. %* Det er det eneste elementet med denne egenskapen
\end{proof}
% ikke skriv $x=\overrightarrow{0}$
% Zero vector comment
%Furthermore, given $\Omega \subset \mathbb{R}^d$ have finite positive lebesgue measure, and let $L^2(\Omega)$ be the corresponding Hilbert space of $L^2$ functions on $\Omega$. Completeness can be restated into showing that $f=0$, for $f \in L^2(\Omega)$, is the only solution to $\langle f,e_\lambda \rangle_{H} = 0$ for all $\lambda \in \Lambda$. This comes from the fact that the only vector/function $v$ orthogonal to a dense linear subspace is the zero vector/function. That is to say since $B$ is any orthonormal set/family, and $v$ is orthonormal to $B$, then $v$ is also orthogonal to $\overline{\operatorname{span}(B)}$ , which is the whole space $H$. Ref linmet bok og paperet om spectral pairs.

% ? Note that the fact that the span of the basis vectors is dense implies that every vector in H can be written as the sum of an infinite series. The orthogonality implies that the decomposition is unique.
% Merk overgang fra $\left\{ e_{n} \right\}_{n\in \mathbb{N}}$ til $\left\{ e_{\lambda} \right\}_{\lambda \in \mathbb{Z}}$ 

\begin{lemma}\label{lem:onb_direct_subset}
    Let $E = \braq{v_i: i \in I}$ be an orthonormal basis for a Hilbert space $H$, and $F$ denote a subset of $E$. Then $F$ cannot be an orthonormal basis for $H$ unless $F = E$.
\end{lemma}
\begin{proof}
    Suppose for contradiction that $F$ is a strict subset of $E$, which is complete in $H$. Let $v_\gamma$ denote a non-zero element such that $v_{\gamma}\in E$ and $v_{\gamma}\notin F$. Then $v_{\gamma}$ is orthogonal to all elements of $F$, so by \cref{lem:ONB_alternative_def} $v_{\gamma}$ must be the zero-element. This is a contradiction, so we must have that $F=E$ if $F$ is to be an orthonormal basis for $H$.
    %Construct the set $F$ as a strict subset of $E$, the orthonormal basis for $H$. Naturally, $F$ is also an orthonormal set, but is it complete? Let $v_\gamma$ denote the element that is $v_{\gamma}\in E$ and $v_{\gamma}\notin F$. This means that $e_{\gamma}$ is orthogonal to all elements of $F$. This is a contradiction, as we know from \cref{lem:ONB_alternative_def} that the only element in $H$ that is orthogonal to all elements in $E$ is the zero-element. Thus we must have that $F=E$ if $F$ is to be an orthonormal basis for $H$.
\end{proof}


\end{document}