\documentclass[../thesis.tex]{subfiles}
% Separate preamble for this subfile. This preamble is loaded last, so one can override various functions before \begin{document}

% Better comment extension for Vscode colors these comments differently
% Normal comment color
% * Important information
% ! ALERT
% ? Question
% TODO stuff to do
% // This is strikethrough


\begin{document}
% In addition to the tilings related to Keller's \namecref{conj:keller_tiling}, \emph{periodicity} is another essential concept. A monohedral tiling is \emph{non-periodic} if it admits no period parallelograms \cite{penrosePentaplexityClassNonPeriodic1979}, and \emph{vice versa}. In our case, this intuitively means that if we know the arrangement of the tile with respect to its vertices and edges within a parallelogram, then we can construct the entire tiling by repeating this arrangement. That is, repeating the parallelogram, not the tile. We refer to \cite[p.29-30]{grunbaumTilingsPatterns1987} for an in-depth explanation with great images. Nevertheless, if our tile exclusively tiles non-periodically, then we say it is \emph{aperiodic}. That is, \emph{every} tiling possible with the tile is non-periodic and must necessarily be so \cite[p. 520]{grunbaumTilingsPatterns1987}. It is critical to make the distinction here that in this context, it is the \emph{tile} that is considered to be aperiodic and not the tiling itself. An \emph{aperiodic tiling}, on the other hand, instead means a complete absence of a period (parallelogram \SigridChange{Correct? with this parallelogram addition?}) in \emph{all} coordinates. And as we informally defined above, if there is a period in at least \emph{one} of the coordinates, it is non-periodic tiling; if there is a period in all directions, then it is a periodic tiling\footnote[1]{These terms have been confusing in the papers studied, which is why we stress these distinctions.}. 
As alluded to in the previous paragraph, the concept of \emph{non-periodic} is essential. The geometric definition of a \emph{non-periodic tiling} is that of a monohedral tiling which does not allow any period parallelograms \cite{penrosePentaplexityClassNonPeriodic1979}, and \emph{vice versa} for a \emph{periodic tiling}. Essentially, this means that if we know the arrangement of the tiles with respect to their vertices and edges within a parallelogram, we can construct the entire tiling by repeating this arrangement. It is important to note that we repeat the parallelogram, not the tile itself. For a more detailed explanation accompanied by illustrative figures, please refer to \cite[p.29-30,147-149]{grunbaumTilingsPatterns1987}. 

Briefly, to clarify these terms, the word \emph{aperiodic} in the literature is unclear. It can refer to the characteristics of the \textsc{tile} itself, where if a \textsc{tile} exclusively allows for non-periodic tilings, it is considered to be aperiodic. This indicates that \emph{every} tiling possible with the \textsc{tile} is non-periodic and must necessarily be so \cite[p. 520]{grunbaumTilingsPatterns1987}. We have seen an example of an \emph{aperiodic tile} in \cref{fig:tiling_two}, wherein matching rules force the colored shapes into the actual tiles that are aperiodic \cite{penrosePentaplexityClassNonPeriodic1979}. The unit cube is not an aperiodic tile; see \cref{fig:tiling_one}. This is the only case where authors agree on one meaning of aperiodic, though all with variations in the definition \cite[p. 520, p.154, p.11]{grunbaumTilingsPatterns1987,baakeAperiodicOrder2013,kolountzakisTilingsTranslation2010} to name a few. Many other meanings of aperiodic are given in \cite{baakeAperiodicOrder2013}, which also illustrates some of the additional disagreements in naming the different kinds of symmetry. One finds terms such as \emph{fully periodic}, \emph{sub-periodic}, and \emph{crystallographic} \cite[p. 45]{baakeAperiodicOrder2013}. For this reason, it is critical to distinguish what aperiodic means in our context. What we mean by \textsc{aperiodic} is the following:  % In short, it means different things in different contexts or something completely different altogether. 
\begin{definition}[Aperiodic tiling \cite{haugeAperiodicTilingTranslations2023}]
    Let $\Lambda$ be a tiling set for a subset $\Omega\subset \R^d$ with non-zero measure. If the tiling is \textsc{not} preserved by any non-zero translation 
    \begin{equation*}
        \mathbf{T}_\tau (x) = x+\tau 
    \end{equation*}
    in $\R^d$, then the the tiling set $\Lambda$ constitutes an \emph{aperiodic tiling}. We say that the tiling has \emph{no translational symmetry} and that the \emph{tiling is aperiodic}.
\end{definition}
Moreover, to avoid any further confusion, we will never use the term aperiodic by itself. We also highlight the following; \textsc{all} aperiodic tilings are non-periodic tilings, but \textsc{not all} non-periodic tilings are aperiodic tilings.

In higher dimensions, the unit cube can tile non-periodically. Two simple examples in dimension two are illustrated in \cref{fig:single_shift_horizontal_tiling,fig:single_shift_vertical_tiling}. Try creating a single parallelogram that can be used to tile the entire plane, and one will quickly see that this is impossible. However, one can consider them to be \emph{half-periodic}, in the sense that they do not admit a vertical or horizontal period, respectively, but do allow for a period in the same direction as the shift. That is a horizontal and vertical period, respectively \cite{kolountzakisTilingsTranslation2010}. Another simple example of a non-periodic tiling in dimension $2$ is to shift the $n$'th column (or row) of unit cubes in the vertical (or horizontal) direction with $e^n$, for all $n\in\Z$ \cite{liuUniformityNonUniformGabor2003}. This is illustrated in \cref{fig:tiling_eight}. These examples are significant because it shows that the unit cube can tile non-periodically in all $d\geq 2$. However, surprisingly, the unit cube can also tile aperiodically in all $d\geq 3$, which will be another main aim of this \namecref{chap:tiling} and a topic for the next \cref{sec:aperi_cube}.

First, we highlight the following. It is precisely the higher dimensional tilings that are either non-periodic or counterexample Keller's \namecref{conj:keller_tiling} that we intuitively define to be the exotic and counterintuitive tilings related to the unit cube.

\end{document}