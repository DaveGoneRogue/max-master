\documentclass[../thesis.tex]{subfiles}
% Separate preamble for this subfile. This preamble is loaded last, so one can override various functions before \begin{document}

% Better comment extension for Vscode colors these comments differently
% Normal comment color
% * Important information
% ! ALERT
% ? Question
% TODO stuff to do
% // This is strikethrough


\begin{document}
The counterexamples that disprove Keller's \namecref{conj:keller_tiling} are non-constructive. That is, we do not know to what they correspond. However, there exist other examples of counterintuitive tilings which are easier to visualize given by \emph{non-periodic} tilings. In this \SigridChange{SECTION/}

The counterexamples of KC are hard to visualize. In fact, highlight their periodicity. Emphasizing a contrast to the aperiodic tilings. "Allude to these periodic sets in the discussion of KC in the tilings chapter. 

\begin{definition}[Aperiodic tiling]
    Let $\Lambda$ be a tiling set for a subset $\Omega\subset \R^d$ with non-zero measure. If the tiling is \textsc{not} preserved by any non-zero translation 
    \begin{equation*}
        \mathbf{T}_\tau (x) = x+\tau 
    \end{equation*}
    in $\R^d$, then the the tiling set $\Lambda$ constitutes an \emph{aperiodic tiling}. We say that the tiling has \emph{no translational symmetry} and that the \emph{tiling is aperiodic}.
\end{definition}

It is precisely the higher dimensional tilings that are either non-periodic or counterexample Keller's \namecref{conj:keller_tiling} that we intuitively define to be the exotic and counterintuitive tilings.

%* Aperiodic and exotix
% In particular, by the group of \emph{aperiodic} tilings. 
% Winston: It is precisely the higher dimensional tilings that are either aperiodic or counterexample Keller's \namecref{conj:keller_tiling} that we informally define as the exotic and counterintuitive tilings. 
% Winston:  The formal definition of aperiodic is given by the following. 

% In addition to the tilings related to Keller's \namecref{conj:keller_tiling}, \emph{periodicity} is another essential concept. A monohedral tiling is \emph{non-periodic} if it admits no period parallelograms \cite{penrosePentaplexityClassNonPeriodic1979}, and \emph{vice versa}. In our case, this intuitively means that if we know the arrangement of the tile with respect to its vertices and edges within a parallelogram, then we can construct the entire tiling by repeating this arrangement. That is, repeating the parallelogram, not the tile. We refer to \cite[p.29-30]{grunbaumTilingsPatterns1987} for an in-depth explanation with great images. Nevertheless, if our tile exclusively tiles non-periodically, then we say it is \emph{aperiodic}. That is, \emph{every} tiling possible with the tile is non-periodic and must necessarily be so \cite[p. 520]{grunbaumTilingsPatterns1987}. It is critical to make the distinction here that in this context, it is the \emph{tile} that is considered to be aperiodic and not the tiling itself. An \emph{aperiodic tiling}, on the other hand, instead means a complete absence of a period (parallelogram \SigridChange{Correct? with this parallelogram addition?}) in \emph{all} coordinates. And as we informally defined above, if there is a period in at least \emph{one} of the coordinates, it is non-periodic tiling; if there is a period in all directions, then it is a periodic tiling\footnote[1]{These terms have been confusing in the papers studied, which is why we stress these distinctions.}. 

Clarify two dimension case here:

In addition to the tilings related to Keller's \namecref{conj:keller_tiling}, the concept of \emph{periodicity} is essential. A monohedral tiling is considered \emph{non-periodic} if it does not allow any period parallelograms \cite{penrosePentaplexityClassNonPeriodic1979}, and \emph{vice versa}. Essentially, this means that if we know the arrangement of the tiles with respect to their vertices and edges within a parallelogram, we can construct the entire tiling by repeating this arrangement. It is important to note that we repeat the parallelogram, not the tile itself. For a more detailed explanation accompanied by illustrative figures, please refer to \cite[p.29-30,147-149]{grunbaumTilingsPatterns1987}. 

Furthermore, if a tile exclusively allows for non-periodic tilings, we consider it to be \emph{aperiodic}. This indicates that \emph{every} tiling possible with the tile is non-periodic and must necessarily be so \cite[p. 520]{grunbaumTilingsPatterns1987}. It is crucial to establish a clear distinction in this context: the term 'aperiodic' specifically relates to the characteristics of the \emph{tile} itself rather than the overall tiling. In contrast, when we refer to an \emph{aperiodic tiling}, we precisely describe a tiling configuration that completely lacks any period (parallelogram \SigridChange{Correct? with this parallelogram addition?}) across \emph{all} coordinates/directions. This statement is stronger than a non-periodic tiling, requiring only at least \emph{one} coordinates/directions with no period. Similarly note, if there is a period in \emph{all} coordinates/directions, then it is a periodic tiling. Hence, all aperiodic tilings are non-periodic, but not all non-periodic tilings are aperiodic tilings\footnote[1]{These terms have been confusing for the author in the many papers and books studied (not necessarily cited), which is why we stress these distinctions.} CITE(must find citation for this informal definition?) %* The best definition I have found is the cover of chapter 10 (p. 519 book, or search for "aperiodic tiling." I will check for an informal definition in Aperiodic Order Volume 1: A Mathematical Invitation. )

In higher dimensions, the unit cube can also tile non-periodically. Two simple examples are illustrated in \cref{fig:single_shift_horizontal_tiling,fig:single_shift_vertical_tiling}. Try creating a single parallelogram that can be used to tile the entire plane, and one will quickly see that this is impossible. However, one can consider them to be \emph{half-periodic}, in the sense that they do not admit a vertical or horizontal period, respectively, but do allow for a period in the same direction as the shift. That is a horizontal and vertical period, respectively \cite{kolountzakisTilingsTranslation2010}. Another simple example of a non-periodic tiling in dimension $2$ is to shift the $n$'th column \SigridChange{(or row)} of unit cubes in the vertical \SigridChange{(horizontal)} direction with $e^n$, for all $n\in\Z$ \cite{liuUniformityNonUniformGabor2003}. This is illustrated in \cref{fig:tiling_eight}. These examples are significant because it shows that the unit cube can tile non-periodically in all $d\geq 2$. However, as the unit cube can always be organized into a lattice and hence tile periodically, it is not an aperiodic tile. However, surprisingly, the unit cube can also tile aperiodically in all $d\geq 3$, which will be another main aim of this \namecref{chap:tiling} a topic for \cref{sec:aperi_cube}.

\end{document}