\documentclass[../thesis.tex]{subfiles}
% Seperate preamble for this subfile. This preamble is loaded last, so it may be used to override various functions.

% Better comment extension for Vscode colors these comments differently
% Normal comment color
% * Important information is highlighted
% ! ALERT
% ? Question
% TODO stuff to do
% // this is strikethrough


\begin{document}


%! Generelt intro ish til det under

% ? Intro overgang, noe for å introdusere d-dimensjoner. 
% ? Before expanding up to higher dimensions, let us form the following definitions.

% ! Note that a spectral set $\Omega$ may have more than one spectrum.
% ! Spectral pairs are connected to tilings.

%*—————————————————————————————————————————————————————————————————————————————————————————————————————————————
%* Dette leder dårlig over til den en dimensjonale tingen, som vi skal se på først og se på fleksibiliteten til
% In other words we have shown that $\brac{[0,1],\mathbb{Z}}$ is a spectral pair. This follows directly from \cref{def:spectral_set} with $\Omega = [0,1]$, $\Lambda = \Z $, $d=1$, and in conjunction with the proof that $\{ e^{2\pi i \lambda t} : \lambda \in \mathbb{Z} \}$ is an orthonormal basis for $L^2([0,1])$. As shown in \cite{taoIntroductionMeasureTheory2011}, closed sets are measurable, in our case with positive and finite measure $\operatorname{mes}(\Omega) = 1$. % * using the Lebesgue measure, state this? %! NEI, stryk measure. 

% TODO
% Todo: notation switch from $\{ e_\lambda \}_{\lambda \in \Z}$ to $\{e^{2 \pi i \dots \lambda \dots} : \lambda \in \Z \}$, go back and change

%Before forming a formal definition of a spectral set, 
%In the setting of Spectral sets in we consider two subsets $\Omega,\Lambda$ of $R^d$ where 
%The setting for spectral sets 
%In the context of $L^2(\Omega)$ for $\Omega \subset \R^d$. 
%When considering

Given a subset with finite and positive measure $\Omega$, recall that we denote by $E(\Lambda)$ the set of exponentials
\begin{align}\label{eq:exp_set_all_d}
    E(\Lambda) = \braqMed{\indicator{\Omega}{t} e_\lambda(t) : \lambda \in \Lambda} = \braqMed{\indicator{\Omega}{t} e^{2 \pi i \inpl{t}} : \lambda \in \Lambda}, \quad t\in \R^d%\\
    %E(\Lambda) = \braqMed{\indicator{\Omega}{t} e_{\lambda}(t) : \lambda \in \Lambda}\\
    %E(\Lambda) = \braqMed{\indicator{\Omega}{t} e_{\lambda} : \lambda \in \Lambda}\\
\end{align}
in $L^2(\Omega)$, where $\Lambda$ is a set of frequencies in $\R^d$. Note, in the earlier instances, we omitted notation with the indicator function $\quad$ \textcolor{green}{kommentar: noe om at dette gjør at én $e_\lambda \in E(\Lambda)$ nå fører til at "bor på" $L^2(\R)$ i motsetning til tidligere hvor de kun "har bodd på"/ $L^2(\Omega)$. i.e indikatrofunksjonen redder integralet med $e_\lambda$ fra å være udefinert i $L^2(\R)$. Må da få frem at vi kun ser på om det er komplett i $\Omega$ og ikke over hele $\R$. Dette betyr dog at alle etterfølgende indreprodukt og integraler egenltig kan gjøres over hele $\R$, for å så inskrenke integrasjonsområdet pga. compact support.}
%in the earlier instances, we have omitted notation with the indicator function defined in \cref{def:indicator}
%we have always multiplied the exponential functions with the indicator function from \cref{def:indicator} although it was not explicitly written.
\begin{definition}[Spectral set] \label{def:spectral_set}
    Let $\Omega \subseteq \R^d$ be a subset with $0< \operatorname{mes} \Omega < \infty$. If the set of exponentials $E(\Lambda)$ form an \textsc{orthogonal} basis for $L^2 (\Omega)$ when $\lambda$ ranges over some subset $\Lambda \subset \mathbb{R}^d$, then $\Omega$ is called a \emph{spectral set} and $\Lambda$ is called a \emph{spectrum} for $\Omega$. We say that $(\Omega, \Lambda)$ is a \emph{spectral pair} \cite{liuUniformityNonUniformGabor2003}.
\end{definition}

\textcolor{green}{kommentar: Her kommer det ting}
%! Noe her om hvorfor vi bryr oss om disse
%* YEPSIPEPSI
\begin{equation*}
    f = \sum_{\lambda \in \Lambda} c_\lambda e_\lambda \text{<<<<----  Er unik og } c_\lambda = \braa{f,e_\lambda}
\end{equation*}

We will mainly consider the case when $\Omega=I^d=\bras{0,1}^d$ is the unit cube in $\R^d$. The significance of this case lies in the connection between tiles and spectra for $\Omega=I^d$, which we will explore in \cref{chap:comparison}. Before considering spectral sets in higher dimensions, let us first consider the one-dimensional case and observe the limited flexibility we have in choosing $\Lambda$ here.

\section{The unit cube in dimension one}\label{sec:complx_trig_1d}
    %* Hvilken flex har  vi? Jo, nær sagt ingen
    %* Samtidig: Concerned with the structure of the discrete sets \Lambda, which at the same time serve as spectra for I^d (the basis property) and also are sets of vectors \lambda, which makes the translates \lambda + I^d tile \R^d
    \subfile{04_x1__one_dim.tex}  %! Remember to import


%! this should be deleted and better merged with Jorgen pedersen section below     
%\section{The complex trigonometric system in 2D}\label{sec:complx_trig_2d}
    %\subfile{04_x2__two_dim.tex}

%\section{The complex trigonometric system in 2D for shifted $\lambda$ values} % OLD TITLE
%\section{The complex trigonometric system in 2D for shifted values}
    %\subfile{04_x3__two_dim_shifted}


\section{Spectral sets in higher dimensions}\label{sec:spec_higher_dim}
%\section{Spectral sets for $\Omega = [0,1]^2$}
    \subfile{04_x4__higher_dim}
    %* Jorgen pedersen – generelt og spesialtilfelle



\end{document}