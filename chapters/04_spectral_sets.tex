\documentclass[../thesis.tex]{subfiles}
% Seperate preamble for this subfile. This preamble is loaded last, so it may be used to override various functions.

% Better comment extension for Vscode colors these comments differently
% Normal comment color
% * Important information is highlighted
% ! ALERT
% ? Question
% TODO stuff to do
% // this is strikethrough


\begin{document}


%! Generelt intro ish til det under

% ? Intro overgang, noe for å introdusere d-dimensjoner. 
% ? Before expanding up to higher dimensions, let us form the following defenitions.

% ! Note that a spectral set $\Omega$ may have more than one spectrum.
% ! Spectral pairs are connected to tilings.


%* Dette leder dårlig over til den en dimensjonale tingen, som vi skal se på først og se på fleksibiliteten til
% In other words we have shown that $\brac{[0,1],\mathbb{Z}}$ is a spectral pair. This follows directly from \cref{def:spectral_set} with $\Omega = [0,1]$, $\Lambda = \Z $, $d=1$, and in conjunction with the proof that $\{ e^{2\pi i \lambda t} : \lambda \in \mathbb{Z} \}$ is an orthonormal basis for $L^2([0,1])$. As shown in \cite{taoIntroductionMeasureTheory2011}, closed sets are measurable, in our case with positive and finite measure $\operatorname{mes}(\Omega) = 1$. % * using Lebesgue measure, state this? %! NEI, stryk measure. 

% TODO
% Todo: notation switch from $\{ e_\lambda \}_{\lambda \in \Z}$ to $\{e^{2 \pi i \dots \lambda \dots} : \lambda \in \Z \}$, go back and change

%Before forming a formal defenition of spectral set, 
%In the setting of Spectral sets in we consider two subsets $\Omega,\Lambda$ of $R^d$ where 
%The setting for spectral sets 
%In the context of $L^2(\Omega)$ for $\Omega \subset \R^d$. 
%When considering

The framework of spectral sets involves a subset $\Omega$ of $\R^d$ with finite and positive measure and its corresponding Hilbert space of $L^2$ functions. In this setting, we will consider exponential functions on $\Omega$ on the form 
\begin{equation}\label{eq:exp}
    e_{\lambda}(t) = e^{2 \pi i \inpl{t}}, t\in \Omega, \lambda \in \Lambda
\end{equation}
where $\inpl{t}$ is the euclidian inner product defined in \cref{def:closed_span}. If these exponential functions form an orthogonal basis for $L^2(\Omega)$ when $\lambda$ ranges over some subset $\Lambda$ in $\R^d$, then $\Omega$ is a spectral set. We formalize this into the following definition.

\begin{definition}[Spectral set \cite{liuUniformityNonUniformGabor2003}] \label{def:spectral_set}
    Let $\Omega \subseteq \mathbb{R}^d$ be a measurable subset with $0< \operatorname{mes} \Omega < \infty$, and let $e_{\lambda}$ be an exponential function on $\Omega$ given by $e_{\lambda}(t) = e^{2\pi i \langle \lambda,t  \rangle }$. If the set of exponentials $\{ e^{2\pi i \langle \lambda,t  \rangle } : \lambda \in \Lambda\}$  form an \textsc{orthogonal} basis for $L^2 (\Omega)$ when $\lambda$ ranges over some subset $\Lambda \subset \mathbb{R}^d$, then $\Omega$ is called a \emph{spectral set} and $\Lambda$ is called a \emph{spectrum} for $\Omega$. We say that $(\Omega, \Lambda)$ is a \emph{spectral pair}. 
\end{definition}

We will mainly consider the case when $\Omega=I^d$, where $I^d=\bras{0,1}^d$ is the unit cube in $\R^d$. The significance of this case lies in the more direct connection between tiles and spectrum for $\Omega=I^d$ than for other sets $\Omega$ and is a topic we will explore further in \cref{chap:comparison}.
Before considering spectral sets in higher dimensions, we will first consider the one-dimensional case, and what flexibility we have in choosing $\Lambda$. 

\section{Spectral sets in dimension one}\label{sec:complx_trig_1d}
\section{The unit cube in dimension one}%\label{sec:complx_trig_1d}
%\section{Trigonometric system in $L^2([0,1])$}\label{sec:complx_trig_1d}
%\chapter[]{The complex trigonometric system} % OLD TITLE
%\section{The complex trigonometric system in 1D} % OLD TITLE
    %* Hvilken flex har  vi? Jo, nær sagt ingen
    %* Samtidig: Concerned with the structure of the dicrete sets \Lambda which at the same time serve as spectra for I^d (the basis property), and also are sets of vectors \lambda, which makes the translates \lambda + I^d tile \R^d
    \subfile{04_x1__one_dim.tex}


%! this should be deleted and better merged with Jorgen pedersen section below     
%\section{The complex trigonometric system in 2D}\label{sec:complx_trig_2d}
    %\subfile{04_x2__two_dim.tex}

%\section{The complex trigonometric system in 2D for shifted $\lambda$ values} % OLD TITLE
%\section{The complex trigonometric system in 2D for shifted values}
    %\subfile{04_x3__two_dim_shifted}


\section{Spectral sets in higher dimensions}
%\section{Spectral sets for $\Omega = [0,1]^2$}
    \subfile{04_x4__higher_dim}
    %* Jorgen pedersen – generelt og spesialtilfelle



\end{document}