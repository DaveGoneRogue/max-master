\documentclass[../thesis.tex]{subfiles}
% Seperate preamble for this subfile. This preamble is loaded last, so it may be used to override various functions.

% Better comment extension for Vscode colors these comments differently
% Normal comment color
% * Important information is highlighted
% ! ALERT
% ? Question
% TODO stuff to do
% // this is strikethrough


\begin{document}


%! Generelt intro ish til det under



% ? Intro overgang, noe for å introdusere d-dimensjoner. 
% ? Before expanding up to higher dimensions, let us form the following defenitions.


\begin{definition}[Spectral set \cite{liuUniformityNonUniformGabor2003}] \label{def:spectral_set}
    Let $\Omega \subseteq \mathbb{R}^d$ be a measurable subset with positive and finite (bounded) measure, i.e $0< \operatorname{mes} \Omega < \infty$, and let $e_{\lambda}$ be an exponential function on $\Omega$ given by $e_{\lambda}(t) = e^{2\pi i \langle \lambda,t  \rangle }$. If the set of exponentials $\{ e^{2\pi i \langle \lambda,t  \rangle } : \lambda \in \Lambda\}$  form an \textsc{orthogonal} basis for $L^2 (\Omega)$ when $\lambda$ ranges over some subset $\Lambda$ in $\mathbb{R}^d$, then $\Omega$ is called a \emph{spectral set} and $\Lambda$ is called a \emph{spectrum} for $\Omega$. We say that $(\Omega, \Lambda)$ is a \emph{spectral pair}. 
\end{definition} 

%? SIGRID VIL STRYKE: 
% Note, even though the exponential function $e^{2\pi i \langle \lambda,t  \rangle }$ is defined over $\mathbb{C}$, the inner product is defined over $\R^d$ as $t\in \Omega\subset \R^d$ and $\lambda \in \Lambda \subset \R^d$, and can be calculated using \cref{eq:dot_prod}.

% ! Note that a spectral set $\Omega$ may have more than one spectrum.
% ! Spectral pairs are connected to tilings.


%* Dette leder dårlig over til den en dimensjonale tingen, som vi skal se på først og se på fleksibiliteten til
% In other words we have shown that $\brac{[0,1],\mathbb{Z}}$ is a spectral pair. This follows directly from \cref{def:spectral_set} with $\Omega = [0,1]$, $\Lambda = \Z $, $d=1$, and in conjunction with the proof that $\{ e^{2\pi i \lambda t} : \lambda \in \mathbb{Z} \}$ is an orthonormal basis for $L^2([0,1])$. As shown in \cite{taoIntroductionMeasureTheory2011}, closed sets are measurable, in our case with positive and finite measure $\operatorname{mes}(\Omega) = 1$. % * using Lebesgue measure, state this? %! NEI, stryk measure. 


% TODO
% Todo: notation switch from $\{ e_\lambda \}_{\lambda \in \Z}$ to $\{e^{2 \pi i \dots \lambda \dots} : \lambda \in \Z \}$, go back and change
% Todo: sjekke spacing, for nå er det ikke konsistent: Ny linje, ny linje med spacing, ny lince med spacing etterfulgt av theorem enviroment, etc




\section{Trigonometric system in $L^2([0,1])$}\label{sec:complx_trig_1d}
%\chapter[]{The complex trigonometric system} % OLD TITLE
%\section{The complex trigonometric system in 1D} % OLD TITLE
    %* Hvilken flex har  vi? Jo, nær sagt ingen
    \subfile{04_x1__one_dim.tex}


%! this should be deleted and better merged with Jorgen pedersen section below     
\section{The complex trigonometric system in 2D}\label{sec:complx_trig_2d}
    \subfile{04_x2__two_dim.tex}

%\section{The complex trigonometric system in 2D for shifted $\lambda$ values} % OLD TITLE
\section{The complex trigonometric system in 2D for shifted values}
    \subfile{04_x3__two_dim_shifted}



\section{Spectral sets for $\Omega = [0,1]^2$}
    %* Jorgen pedersen – generelt og spesialtilfelle



\end{document}