\documentclass[../thesis.tex]{subfiles}
% Separate preamble for this subfile. This preamble is loaded last, so one can override various functions before \begin{document}

% Better comment extension for Vscode colors these comments differently
% Normal comment color
% * Important information
% ! ALERT
% ? Question
% TODO stuff to do
% // This is strikethrough


\begin{document}


%! Generelt intro ish til det under

% ? Intro overgang, noe for å introdusere d-dimensjoner. 
% ? Before expanding up to higher dimensions, let us form the following definitions.

% ! Note that a spectral set $\Omega$ may have more than one spectrum.
% ! Spectral pairs are connected to tilings.

%*—————————————————————————————————————————————————————————————————————————————————————————————————————————————
%* Dette leder dårlig over til den en dimensjonale tingen, som vi skal se på først og se på fleksibiliteten til
% In other words we have shown that $\brac{[0,1],\mathbb{Z}}$ is a spectral pair. This follows directly from \cref{def:spectral_set} with $\Omega = [0,1]$, $\Lambda = \Z $, $d=1$, and in conjunction with the proof that $\{ e^{2\pi i \lambda t} : \lambda \in \mathbb{Z} \}$ is an orthonormal basis for $L^2([0,1])$. As shown in \cite{taoIntroductionMeasureTheory2011}, closed sets are measurable, in our case with positive and finite measure $\operatorname{mes}(\Omega) = 1$. % * using the Lebesgue measure, state this? %! No, no need to mention "measure" here. 

% TODO

%Before forming a formal definition of a spectral set, 
%In the setting of Spectral sets in we consider two subsets $\Omega,\Lambda$ of $R^d$ where 
%The setting for spectral sets 
%In the context of $L^2(\Omega)$ for $\Omega \subset \R^d$. 
%When considering

Given a subset with finite and positive measure $\Omega$, recall that we denote by $E(\Lambda)$ the set of exponentials
\begin{align*}%\label{eq:exp_set_all_d}
    E(\Lambda) = \braqMed{\indicator{\Omega}{t} e_\lambda(t) : \lambda \in \Lambda} = \braqMed{\indicator{\Omega}{t} e^{2 \pi i \braa{\lambda, t}} : \lambda \in \Lambda}, \quad t\in \R^d
\end{align*}
in $L^2(\Omega)$, where $\Lambda$ is a set of frequencies in $\R^d$.
\begin{definition}[Spectral set] \label{def:spectral_set}
    Let $\Omega \subset \R^d$ be a bounded subset with $0< \operatorname{mes} \Omega < \infty$. If the set of exponentials $E(\Lambda)$ form an \textsc{orthogonal} basis for $L^2 (\Omega)$ and $\Lambda \subset \R^d$, then $\Omega$ is called a \emph{spectral set} and $\Lambda$ is called a \emph{spectrum} for $\Omega$. We say that $(\Omega, \Lambda)$ is a \emph{spectral pair} \cite{liuUniformityNonUniformGabor2003}.
\end{definition} 

Spectral sets and spectra are useful and interesting. One evident reason follows directly from the basis property. That is, every function $f \in L^2(\Omega)$ has a series expansion given by
\begin{equation*}
    f = \sum_{\lambda \in \Lambda} c_{\lambda} e_{\lambda},
\end{equation*}
with $e_\lambda \in E(\Lambda)$, and constants explicitly given by the inner product $c_{\lambda}=\braa{f,e_{\lambda}}$. Moreover, the orthogonality condition guarantees that the series expansion is \emph{unique}.

Due to the restrictions imposed by the orthogonality requirement, there is limited flexibility in constructing spectral sets, making them highly rigid structures. As a consequence, not many sets $\Omega$ have the property of being spectral. Examples of sets that are not spectral are the \emph{unit triangle} and \emph{unit ball}\footnote[1]{It was later shown that the unit ball for any dimension $d\geq2$ is not a spectral set \cite{iosevichFourierBasesDistance2001}. See also \cite{kolountzakisPackingTilingOrthogonality2000} for an excellent exposition.} for $d=2$, meaning that there exist no suitable spectra $\Lambda$ to make a spectral pair with either of these sets \cite{fugledeCommutingSelfadjointPartial1974}. However, once we have one spectrum for a spectral set, observe that the spectrum is shift-invariant.
%* Also, spectral sets are a rarity as very few sets satisfy all the conditions. Even classifying all spectra for certain sets is not a trivial case. which is why the spectral conjecture of Fuglede is so essential. 
\begin{lemma}\label{lem:spectrum_shift_is_spectrum}
    Let $\Lambda$ be a spectrum for a spectral set $\Omega \subset \R^d$. Then $\Lambda_\alpha = \Lambda + \alpha$ is also a spectrum for $\Omega$ for any vector $\alpha \in \R^d$.
\end{lemma}

\begin{proof}
    Assume that $\Lambda$ is a spectrum for $\Omega \subset \R^d$, and fix $\alpha\in \R^d$. Given $\lambda,\lambda' \in \Lambda$ the inner product factors as
    \begin{align*}
        \braaMed{e_{\lambda+\alpha},e_{\lambda'+\alpha}}_{L^2(\Omega)} =& \int_{\Omega} e_{\lambda+\alpha}(t)\overline{e_{\lambda'+\alpha}(t)} dt\\
        =& \int_{\Omega} e^{2\pi i \braa{\lambda+\alpha,t}} e^{-2 \pi i \braa{\lambda'+\alpha,t}} dt\\
        =& \int_{\Omega} e^{2\pi i  ((\lambda_1+\alpha_1)t_1 + \dots +(\lambda_d+\alpha_d) t_d)} e^{-2\pi i  ((\lambda_1'+\alpha_1) t_1 + \dots +(\lambda_d'+\alpha_d) t_d)} dt\\
        =& \int_{\Omega} e^{2\pi i  ((\lambda_1+\alpha_1) -(\lambda_1'+\alpha_1))t_1} \cdots e^{2\pi i  ((\lambda_d+\alpha_d) -(\lambda_d'+\alpha_d))t_d} dt\\
        =& \int_{\Omega} e^{2\pi i  (\lambda_1 -\lambda_1')t_1} \cdots e^{2\pi i  (\lambda_d -\lambda_d')t_d} dt\\
        =& \braaMed{e_{\lambda},e_{\lambda'}}_{L^2(\Omega)}.
    \end{align*}
    This shows that $E(\Lambda_\alpha)$ is orthogonal since $E(\Lambda)$ is orthogonal. Now, to show completeness assume that $f\in L^2(\Omega)$ satisfies $\braaMed{f,e_{\lambda + \alpha}} = 0$ for all $\lambda\in \Lambda_\alpha$, and observe that
    \begin{align*}
        \braaMed{f,e_{\lambda+\alpha}}_{L^2(\Omega)} =& \int_{\Omega} f(t) e^{-2\pi i  ((\lambda_1+\alpha_1) t_1 + \dots +(\lambda_d+\alpha_d) t_d)} dt\\
        =& \int_{\Omega} f(t) e^{-2\pi i  (\lambda_1+\alpha_1)t_1} \dots e^{-2\pi i  (\lambda_d+\alpha_d)t_d} dt\\
        =& \int_{\Omega} f(t) e^{-2\pi i  \lambda_1 t_1}e^{-2\pi i  \alpha_1 t_1} \dots e^{-2\pi i  \lambda_1 t_d}e^{-2\pi i  \alpha_d t_d} dt\\
        =& \int_{\Omega} f(t) e^{-2\pi i  (\alpha_1t_1 + \dots + \alpha_dt_d) }  e^{-2\pi i  (\lambda_1t_1 + \dots + \lambda_d t_d)}dt\\
        =& \int_{\Omega} f(t) e^{-2\pi i \braa{\alpha, t} }  e^{-2\pi i  \braa{\lambda, t}}dt\\
        =& \braaMed{f e^{-2\pi i  \braa{\alpha, t} },e_{\lambda}}_{L^2(\Omega)}
    \end{align*}
    That is, we must have $\braaMed{f e^{2\pi i  \braa{\alpha, t} },e_{\lambda}} = 0$ for all $\lambda \in \Lambda$. Since $\Lambda$ is a spectrum, we know that 
    \begin{equation*}
        f(t) e^{-2 \pi i \braa{\alpha, t}} = 0
    \end{equation*}
    for almost all $t \in \Omega$. As $e^{-2 \pi i \braa{\alpha, t}} \neq 0$ for all $t\in \Omega$, this implies that $f = 0$ almost everywhere on $\Omega$. This shows that $E(\Lambda_\alpha)$ is complete in $L^2(\Omega)$. %In other words, in both the orthogonality and completeness, $E(\Lambda_\alpha)$ was unaffected by the shift $\alpha \in \R^d$. 
    Since $\alpha$ was arbitrary, this concludes the proof. 
\end{proof}

% As in \cref{chap:tiling},
We will here mainly consider the case when $\Omega$ is the unit cube in $\R^d$ and study the connection between tiling sets and spectra for $\Omega=I^d$.  Before considering spectral sets for the unit cube in higher dimensions, let us first consider the one-dimensional case and observe our limited flexibility in choosing $\Lambda$ here.

\section{The unit cube in dimension one}\label{sec:complx_trig_1d}
    %* What flexibility do we have? Almost nothing at all
    %* Samtidig: Concerned with the structure of the discrete sets \Lambda, which at the same time serve as spectra for I^d (the basis property) and also are sets of vectors \lambda, which makes the translates \lambda + I^d tile \R^d
    \subfile{04_x1__one_dim.tex}

\section{Spectral sets in higher dimensions}\label{sec:spec_higher_dim}
%\section{Spectral sets for $\Omega = [0,1]^2$}
    \subfile{04_x4__higher_dim}
    %* Jorgen pedersen – generelt og spesialtilfelle



\end{document}