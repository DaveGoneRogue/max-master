\documentclass[../thesis.tex]{subfiles}
% Seperate preamble for this subfile. This preamble is loaded last, so it may be used to override various functions.

% Better comment extension for Vscode colors these comments differently
% Normal comment color
% * Important information is highlighted
% ! ALERT
% ? Question
% TODO stuff to do
% // this is strikethrough


\begin{document}

\section{Notation}
    \subfile{02_x1__notation.tex}
%! Notasjon i L^2 og i L^P
%! C(Omega), C([0,1]), C_per(0,1)

\section{Orthonormal Bases}
    \subfile{02_x2__onb.tex}
% ! ONB generelt, men ikke mer enn nødvendig, og da kanskje spesifisert for L^2


\section{Exponential function}
Given $\lambda \in \mathbb{Z}$, the complex exponential of the frequency $\lambda$ is 
\begin{equation}
    e_{\lambda}(t) = e^{2 \pi i \lambda t} = \cos{(2 \pi \lambda t)} + i \sin{(2 \pi \lambda t)}, \quad t\in \mathbb{R}    
\end{equation}
The complex trigonometric system is the sequence
\begin{equation}
    \left\{ e_{\lambda} \right\}_{\lambda\in \mathbb{Z}} = \{ e^{2 \pi i \lambda t} \}_{\lambda \in \mathbb{Z}}    
\end{equation}
where the domain of $e_\lambda$ is the real line (the input $t$ is a real number), and the output is a complex number. Furthermore, let $\mathbb{T}=\{z\in \mathbb{C} : |z|=1\}$ be the unit circle in the complex plane, and note that each point $z\in\mathbb{T}$ is on the form $z=e^{2 \pi i t}$ for $t$ a real number, meaning that for an arbitrary $\lambda \in \mathbb{Z}$ we have
\begin{equation}
    \left|e_{\lambda}(t) \right|= |e^{2 \pi i \lambda t} | = 1 \quad \text{for all } t\in \mathbb{R} 
\end{equation}
Furthermore, observe that $e_\lambda$ is 1-\emph{periodic}, because
\begin{equation}
    e_\lambda(t+1) = e^{2 \pi i \lambda (t+1)} = e^{2 \pi i \lambda t} e^{2 \pi i \lambda} = e^{2 \pi i \lambda t} = e_\lambda(t)
\end{equation}

The 1-periodicity of $e_\lambda$ is important. If we know values of $e_\lambda(t)$ for $t$ on an interval of length $1$, we know the values for all points $t\in \mathbb{R}$. This allows us to restrict our domain from all of $\mathbb{R}$ to just the $t$'s that lie on the interval $[0,1]$, the unit interval.







\end{document}