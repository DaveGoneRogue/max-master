\documentclass[../thesis.tex]{subfiles}
% Separate preamble for this subfile. This preamble is loaded last, so it may be used to override various functions.

% Better comment extension for Vscode colors these comments differently
% Normal comment color
% * Important information is highlighted
% ! ALERT
% ? Question
% TODO stuff to do
% // this is strikethrough

\begin{document}

\begin{definition}[Span]\label{def:span}
    Let $E$ be a subset of a vector space $X$. The \emph{finite linear span}, or \emph{span} for short, is the set of all finite linear combinations of $E$, that is
    \begin{equation*}
        \spn{E}=\braqMed{\sum_{n=1}^N C_n v_n : N>0, v_n \in E, C_i \in \mathbb{F}}.
    \end{equation*}
    We say that $E$ \emph{spans} $X$ if $\spn{E} = X$.
\end{definition}


\begin{definition}[Closed Span]\label{def:closed_span}
    Let $E$ be a subset of a normed space $X$. The \emph{closed linear span} or \emph{closed span} of $E$ is the closure of $\spnclos{E}$, that is
    \begin{equation*}
        \spnclos{E} = \braqMed{y\in X: \exists y_i\in \spn{E} \quad \text{ s.t } \quad y_i \longrightarrow y} \qedhere
    \end{equation*}
\end{definition}


\begin{definition}[Closure]\label{def:closure}
    Let $X$ be an arbitrary set. Given a subset $E \subseteq X$, the \emph{closure} of $E$, denoted $\overline{E}$, is the smallest closed set containing $E$. The closure is expressed as the intersection of all closed sets containing $E$. That is the set
    \begin{equation*}
        \overline{E} = \bigcap \braqMed{F \subseteq X : F \text{ closed, and } E \subseteq F}.
    \end{equation*}
    Furthermore, if $\overline{E} = X$ then we say that $E$ is \emph{dense} in $X$. Also, if $E$ is closed, then $\overline{E}=E$. In this regard, note that any intersection of closed sets is closed, implying that $\overline{E}$ is closed.
\end{definition}
\begin{remark}
    Note that if $E$ is a subset of a metric space $X$, then
    \begin{equation*}
        \overline{E} = \{ y \in X : \exists y_n\in E \quad \text{ s.t }\quad y_n \longrightarrow y\}.
    \end{equation*}
    %proof of the above can be found in CITE: Linmet bok, page 67.
\end{remark}

\begin{definition}[Complete]\label{def:complete}
    Let $E$ be a subset of a normed space $X$. If $\spnclos{E} = X$, that is, $\spn{E}$ being dense in $X$ with respect to the \GenNormX, then we say that $E$ is \emph{complete}, \emph{fundamental}, or \emph{total}.
\end{definition}

%* —————————————————————————————— Importing files ———————————————————————————————
\section{Notation}
    \subfile{02_x1__notation.tex}
%! Notasjon i L^2 og i L^P
%! C(Omega), C([0,1]), C_per(0,1)

\section{Orthonormal Bases}
    \subfile{02_x2__onb.tex}
% ! ONB generelt, men ikke mer enn nødvendig, og da kanskje spesifisert for L^2


\section{Exponential functions}
Given $\lambda \in \R^d$, the complex exponential function $e_{\lambda}: \R^d \rightarrow \C$ with frequency $\lambda$ is 
\begin{equation}
    e_{\lambda}(t) = e^{2 \pi i \inpl{t}} = \cos{(2 \pi \inpl{t})} + i \sin{(2 \pi \inpl{t})}, \quad t\in \mathbb{R}^d.
\end{equation}
Here, $\inpl{t}$ is the euclidean inner product as defined in \cref{def:dot_prod}. Throughout this thesis we denote by $E\brac{\Lambda}$ the exponential system
\begin{align}
    %E\brac{\Lambda} = \braq{e_\lambda}_{\lambda \in \Lambda} = \braq{e^{2\pi i \lambda t } : \lambda \in \Lambda}
    % E\brac{\Lambda} = \braqMed{e_\lambda}_{\lambda \in \Lambda} = \braqMed{e^{2\pi i \inpl{t}} : \lambda \in \Lambda}.
    %*DENNNEE\brac{\Lambda} =& \braqMed{e_\lambda}_{\lambda \in \Lambda} = \braqMed{e^{2\pi i \inpl{t}}}_{\lambda \in \Lambda} = \braqMed{e^{2\pi i \inpl{t}} : \lambda \in \Lambda}.\\
    %E\brac{\Lambda} =& \braq{e_\lambda}_{\lambda \in \Lambda} = \braq{e^{2\pi i \inpl{t}}}_{\lambda \in \Lambda}= \braq{e^{2\pi i \inpl{t}} : \lambda \in \Lambda}.
    E\brac{\Lambda} = \braqMed{e_\lambda(t) : \lambda \in \Lambda} = \braqMed{e^{2\pi i \inpl{t}} : \lambda \in \Lambda}, \quad t\in \R^d.
\end{align}
Throughout the thesis, we will mainly think of $E(\Lambda)$ as a system of functions in $L^2(\Omega)$ for some subset $\Omega \subseteq \R^d$. In this setting, we intuitively understand $e_\lambda$ to be the $e_\lambda$ restricted to $\Omega$, that is $\indicatorNoVar{\Omega}$

\begin{example}
    In the case where $d=1$ and $\Lambda = \Z$ we have what some authors refer to as \emph{the complex trigonometric system} \cite{heilMetricsNormsInner2018} \cite{encyclopediaofmathematicsTrigonometricSystem},
    \begin{align}
        %E\brac{\Z} = \braqMed{e_n}_{n \in \Z} = \braqMed{e^{2\pi i t n} : n \in \Z}, %\quad t\in \R.
        %E\brac{\Z} = \braqMed{e_n : n \in \Z} = \braqMed{e^{2\pi i t n} : n \in \Z}, \quad t\in \R.
        E\brac{\Z} &= \braqMed{e^{2\pi i t n} : n \in \Z}, \quad t\in \R.\\
        %E\brac{\Z} &= \braqMed{e_n : n \in \Z}\\
        %E\brac{\Z} &= \braqMed{e_n(t) : n \in \Z}\quad t\in \R.
    \end{align}
    Note that for an arbitrary $n \in \Z$ we have
    \begin{equation}
        \bralMed{e_{n}(t) }= \bralMed{e^{2 \pi i n t}} = 1 \quad \text{for all } t\in \R.
    \end{equation}
    Furthermore, observe that $e_n$ is 1-\emph{periodic}, because
    \begin{equation}
        e_n(t+1) = e^{2 \pi i n (t+1)} = e^{2 \pi i n t} e^{2 \pi i n} = e^{2 \pi i n t} = e_n(t)
    \end{equation}
    The 1-periodicity of $e_n$ is important. If we know values of $e_n(t)$ for $t$ on an interval of length $1$, we know the values for all points $t\in \mathbb{R}$. For example, this allows us to restrict our domain from all of $\mathbb{R}$ to just the $t$'s that lie on the interval $[0,1]$, the unit interval. 
\end{example}
%Furthermore, let $\mathbb{T}=\{z\in \mathbb{C}: |z|=1\}$ be the unit circle in the complex plane, and note that each point $z\in\mathbb{T}$ is on the form $z=e^{2 \pi it}$ for $t$ a real number 

%* TANKE: noe om boundedness, og at exp lever i L(\Omega) men IKKE i L^2(R) for da går normen av exp til *helvete*
\end{document}


