\documentclass[../thesis.tex]{subfiles}
% Separate preamble for this subfile. This preamble is loaded last, so it may be used to override various functions.

% Better comment extension for Vscode colors these comments differently
% Normal comment color
% * Important information is highlighted
% ! ALERT
% ? Question
% TODO stuff to do
% // this is strikethrough


\begin{document}

\section{Notation}
    \subfile{02_x1__notation.tex}
%! Notasjon i L^2 og i L^P
%! C(Omega), C([0,1]), C_per(0,1)

\section{Orthonormal Bases}
    \subfile{02_x2__onb.tex}
% ! ONB generelt, men ikke mer enn nødvendig, og da kanskje spesifisert for L^2


\section{Exponential functions}
\textcolor{green}{kommentar: Har notasjonskonflikt med notasjon for $e_\lambda= e^{2 \pi i \inpl{t}}$ uten indikator funksjon, og hvor jeg senere egt bruker $e_\lambda=\indicator{\Omega}{t} e^{2 \pi i \inpl{t}}$}
Given $\lambda \in \R^d$, the complex exponential function $e_{\lambda}: \R^d \rightarrow \C$ with frequency $\lambda$ is 
\begin{equation}
    e_{\lambda}(t) = e^{2 \pi i \inpl{t}} = \cos{(2 \pi \inpl{t})} + i \sin{(2 \pi \inpl{t})}, \quad t\in \mathbb{R}^d.
\end{equation}
Here, $\inpl{t}$ is the euclidian inner product as defined in \cref{def:dot_prod}. Throughout this thesis we denote by $E\brac{\Lambda}$ the exponential system
\begin{align}
    %E\brac{\Lambda} = \braq{e_\lambda}_{\lambda \in \Lambda} = \braq{e^{2\pi i \lambda t } : \lambda \in \Lambda}
    % E\brac{\Lambda} = \braqMed{e_\lambda}_{\lambda \in \Lambda} = \braqMed{e^{2\pi i \inpl{t}} : \lambda \in \Lambda}.
    %*DENNNEE\brac{\Lambda} =& \braqMed{e_\lambda}_{\lambda \in \Lambda} = \braqMed{e^{2\pi i \inpl{t}}}_{\lambda \in \Lambda} = \braqMed{e^{2\pi i \inpl{t}} : \lambda \in \Lambda}.\\
    %E\brac{\Lambda} =& \braq{e_\lambda}_{\lambda \in \Lambda} = \braq{e^{2\pi i \inpl{t}}}_{\lambda \in \Lambda}= \braq{e^{2\pi i \inpl{t}} : \lambda \in \Lambda}.
    E\brac{\Lambda} = \braqMed{e_\lambda(t) : \lambda \in \Lambda} = \braqMed{e^{2\pi i \inpl{t}} : \lambda \in \Lambda}, \quad t\in \R.
\end{align}
\begin{example}
    In the case where $d=1$ and $\Lambda = \Z$ we have what some authors refer to as \emph{the complex trigonometric system} \cite{heilMetricsNormsInner2018} \cite{encyclopediaofmathematicsTrigonometricSystem},
    \begin{align}
        %E\brac{\Z} = \braqMed{e_\lambda}_{\lambda \in \Z} = \braqMed{e^{2\pi i t \lambda} : \lambda \in \Z}, %\quad t\in \R.
        %E\brac{\Z} = \braqMed{e_\lambda : \lambda \in \Z} = \braqMed{e^{2\pi i t \lambda} : \lambda \in \Z}, \quad t\in \R.
        E\brac{\Z} &= \braqMed{e^{2\pi i t \lambda} : \lambda \in \Z}, \quad t\in \R.\\
        %E\brac{\Z} &= \braqMed{e_\lambda : \lambda \in \Z}\\
        E\brac{\Z} &= \braqMed{e_\lambda(t) : \lambda \in \Z}\quad t\in \R.
    \end{align}
    Note that for an arbitrary $\lambda \in \Z$ we have
    \begin{equation}
        \bralMed{e_{\lambda}(t) }= \bralMed{e^{2 \pi i \lambda t}} = 1 \quad \text{for all } t\in \R.
    \end{equation}
    Furthermore, observe that $e_\lambda$ is 1-\emph{periodic}, because
    \begin{equation}
        e_\lambda(t+1) = e^{2 \pi i \lambda (t+1)} = e^{2 \pi i \lambda t} e^{2 \pi i \lambda} = e^{2 \pi i \lambda t} = e_\lambda(t)
    \end{equation}
    The 1-periodicity of $e_\lambda$ is important. If we know values of $e_\lambda(t)$ for $t$ on an interval of length $1$, we know the values for all points $t\in \mathbb{R}$. For example, this allows us to restrict our domain from all of $\mathbb{R}$ to just the $t$'s that lie on the interval $[0,1]$, the unit interval. 
\end{example}
%Furthermore, let $\mathbb{T}=\{z\in \mathbb{C}: |z|=1\}$ be the unit circle in the complex plane, and note that each point $z\in\mathbb{T}$ is on the form $z=e^{2 \pi it}$ for $t$ a real number 

%* TANKE: noe om boundedness, og at exp lever i L(\Omega) men IKKE i L^2(R) for da går normen av exp til *helvete*
\end{document}