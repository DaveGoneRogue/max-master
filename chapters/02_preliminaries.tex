\documentclass[../thesis.tex]{subfiles}
% Separate preamble for this subfile. This preamble is loaded last, so one can override various functions before \begin{document}

% Better comment extension for Vscode colors these comments differently
% Normal comment color
% * Important information
% ! ALERT
% ? Question
% TODO stuff to do
% // This is strikethrough


\begin{document}

\SigridChange{Intro text for what the contents of this chapter is about. (and reference to main sources?) }
The main sources for this chapter are different sections from both \cite{heilMetricsNormsInner2018} and \cite{heilIntroductionRealAnalysis2019}.
\textbf{alternative} For more details on this chapter, the reader is referred to chapters two, three, and four of \cite{heilMetricsNormsInner2018}.

Perhaps in its own section labeled notation, maybe we can fill in some more here
\textcolor{orange}{
%* —————————————————————————————————————— Indicator function  ——————————————————————————————————————
\begin{definition}(Indicator function)\label{def:indicator}
    Let $E$ be a subset of a set $X$. The \emph{indicator function}, also known as the \emph{characteristic function} of $E$, is a function $\indicator{E}{t}: X \rightarrow \braq{0,1}$ where
    \begin{equation*}
        \indicator{E}{t}  = 
        \begin{cases} 
            1, &  t\in E,\\
            0, &  t \notin E.
        \end{cases}
        \qedhere
    \end{equation*}
\end{definition}}

\textcolor{orange}{
\begin{definition}\label{def:dot_prod}
    Let $t=(t_1,\dots t_d)$ and $\lambda=(\lambda_1, \dots, \lambda_d)$ be two vectors in $\R^d$. The inner product, denoted $\langle \cdot, \cdot \rangle$, between $t$ and $\lambda$ is
    \begin{equation*}
        \langle t, \lambda \rangle = \sum_{n=1}^d t_n \lambda_n = t_1\lambda_1 + \dots + t_d\lambda_d,
    \end{equation*}
    also known as the dot product. 
\end{definition}
}

\section{section title / results on ... / Further results on function spaces (and move section after 2.2) }
    \subfile{02_x1__asec.tex}

\section{Vector spaces / Vector spaces of functions}
    \subfile{02_x2__vecspace.tex}

\section{Orthonormal Bases}
    \subfile{02_x3__onb.tex}

\section{Exponential functions}
    \subfile{02_x4__exponential_function.tex}
    %* TANKE: noe om boundedness, og at exp lever i L(\Omega) men IKKE i L^2(R) for da går normen av exp til *helvete*


\end{document}


