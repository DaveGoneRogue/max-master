\documentclass[../thesis.tex]{subfiles}
% Separate preamble for this subfile. This preamble is loaded last, so one can override various functions before \begin{document}

% Better comment extension for Vscode colors these comments differently
% Normal comment color
% * Important information
% ! ALERT
% ? Question
% TODO stuff to do
% // this is strikethrough


\begin{document}

\begin{definition}[Span]\label{def:span}
    Let $E$ be a subset of a vector space $X$. The \emph{finite linear span}, or \emph{span} for short, is the set of all finite linear combinations of $E$, that is
    \begin{equation*}
        \spn{E}=\braqMed{\sum_{n=1}^N C_n v_n : N>0, v_n \in E, C_i \in \mathbb{F}}.
    \end{equation*}
    We say that $E$ \emph{spans} $X$ if $\spn{E} = X$.
\end{definition}

\begin{definition}[Closure]\label{def:closure}
    Let $X$ be an arbitrary set. Given a subset $E \subseteq X$, the \emph{closure} of $E$, denoted $\overline{E}$, is the smallest closed set containing $E$. The closure is expressed as the intersection of all closed sets containing $E$. That is the set
    \begin{equation*}
        \overline{E} = \bigcap \braqMed{F \subseteq X : F \text{ closed, and } E \subseteq F}. \qedhere
    \end{equation*}
\end{definition}
Furthermore, if $\overline{E} = X$ then we say that $E$ is \emph{dense} in $X$. Also, if $E$ is closed, then $\overline{E}=E$. In this regard, note that any intersection of closed sets is closed, implying that $\overline{E}$ is closed.
\begin{remark}
    Note that if $E$ is a subset of a metric space $X$, then
    \begin{equation*}
        \overline{E} = \{ y \in X : \exists \space \bracMed{y_n}_{n\geq 1} \in E \quad \text{ s.t }\quad y_n \longrightarrow y\}.
    \end{equation*}
    %proof of the above can be found in CITE: Linmet bok, page 67.
\end{remark}

\begin{definition}[Closed Span]\label{def:closed_span}
    Let $E$ be a subset of a normed space $X$. The \emph{closed linear span} or \emph{closed span} of $E$ is the closure of $\spnclos{E}$, that is
    \begin{equation*}
        \spnclos{E} = \braqMed{y\in X: \exists \space y_i\in \spn{E} \quad \text{ s.t } \quad y_i \longrightarrow y}. \qedhere
    \end{equation*}
\end{definition}

\begin{definition}[Complete]\label{def:complete}
    Let $E$ be a subset of a normed space $X$. If $\spnclos{E} = X$, that is, $\spn{E}$ being dense in $X$ with respect to the \GenNormX, then we say that $E$ is \emph{complete}.  %, \emph{fundamental}, or \emph{total}.
\end{definition}

%* —————————————————————————————— Importing files ———————————————————————————————
\section{Notation}
    \subfile{02_x1__notation.tex}

\section{Orthonormal Bases}
    \subfile{02_x2__onb.tex}
% ! ONB generelt, men ikke mer enn nødvendig, og da kanskje spesifisert for L^2


\section{Exponential functions}
    \subfile{02_x3__exponential_function.tex}

%* TANKE: noe om boundedness, og at exp lever i L(\Omega) men IKKE i L^2(R) for da går normen av exp til *helvete*
\end{document}


