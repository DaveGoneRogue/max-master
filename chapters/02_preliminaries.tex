\documentclass[../thesis.tex]{subfiles}
% Separate preamble for this subfile. This preamble is loaded last, so one can override various functions before \begin{document}

% Better comment extension for Vscode colors these comments differently
% Normal comment color
% * Important information
% ! ALERT
% ? Question
% TODO stuff to do
% // This is strikethrough


\begin{document}
We commence this thesis by introducing a \SigridChange{selection/assortment} of fundamental notations, definitions, and results that will be employed throughout. Following this, a concise overview of key results regarding vector spaces and vector spaces of functions is presented. It is important to note that these sections do not delve deeply into these topics; for thorough coverage, the reader is referred to the primary sources of this chapter \cite{heilMetricsNormsInner2018,heilIntroductionRealAnalysis2019}. The chapter concludes with a short discussion of the terminology and notation employed alongside the exponential and indicator functions.

To begin, let $\N$ denote the set of positive integers $\braq{1,2,3, \dots}$, $\Z$ denote the set of all integers $\braq{\dots, -1,0,1, \dots}$, and $\R$ and $\C$ to denote the set of all real and complex numbers respectively. Furthermore, we denote by $I^d = \bras{0,1}^d$ the \emph{$d$-dimensionial unit cube} in $\R^d$, sometimes referred to as the \emph{unit hypercube} or simply the \emph{unit cube}. These terms will be used interchangeably, and unless the dimension is specified in the context, it will mean the $d$-dimensional cube. Note that the letter $d$ always refers to the dimension and \emph{vice versa}.

Last, we note that the statement is found somewhere in the text if we write \namecref{conj:keller_tiling} instead of conjecture, and similar for theorems, lemmas, and \emph{et cetera}. From the context the reference appears in, one will also find the corresponding label number.

\section{Results on vector spaces}  %! Closure from set theory. The rest are results of vector spaces
    \subfile{02_x1__asec.tex}

\section{Vector spaces of functions}  %* Functions also form vector spaces
    \subfile{02_x2__vecspace.tex}

\section{Orthonormal Bases}
    \subfile{02_x3__onb.tex}

\section{Exponential functions}
    \subfile{02_x4__exponential_function.tex}

\section{The indicator function and the zero-set}\label{sec:indicator_zero_set}
    \subfile{02_x5__indicator.tex}

\end{document}


