\documentclass[../thesis.tex]{subfiles}
% Seperate preamble for this subfile. This preamble is loaded last, so it may be used to override various functions.

% Better comment extension for Vscode colors these comments differently
% Normal comment color
% * Important information is highlighted
% ! ALERT
% ? Question
% TODO stuff to do
% // this is strikethrough


\begin{document}

% Begynne med L^2 og L^p
% Så ta for seg C([0,1]) og C_per([0,1])
% Hvordan flette inn C(\Omega) ??
\begin{definition}[Lebesgue Space]
    Let $E$ be a measurable subset of $\R^d$. Given an index $1 \leq p < \infty$ and a measurable function $f:E\rightarrow \C$, we say that $f$ is \emph{$p$-integrable} if the Lebesgue integral of $|f|^p$ is finite, or equivalent that the $\|\cdot\|_p$-norm of $f$ is finite. That is,
    \begin{equation*}
        \int_{E} |f(t)|^p dt < \infty,
    \end{equation*}
    or 
    \begin{equation*}
        \|f\|_{L^p(E)} = \brac{\int_{E} |f(t)|^p dt}^{1/p}< \infty,
    \end{equation*}
    where $dt= dt_1 \dots dt_d$. Conversly, $\|f\|_p = \infty$ then $f$ is \emph{not} $p$-integrable. We define the \emph{Lebesgue space} $L^p(E)$ as the set of all $p$-integrable functions on $E$. %The \emph{Lebesgue space} $L^p(E)$ is the set of all $p$-integrable functions on $E$.
\end{definition}
% Heil REal analysis book page 269, citation?

Of all the $L^p$ spaces, the case of $p=2$ is noteworthy since $L^2(E)$ is the only space that has its norm induced from the inner product.
In addition, from the Riesz-Fischer Theorem \cite[p.~279]{heilIntroductionRealAnalysis2019} all $L^p$ spaces are complete with respect to its $\|\cdot\|_p$-norm. Hence, $L^2$ is a Hilbert space. The inner product on $L^2(E)$ for two functions $f,g\in L^2(E)$ is given by
\begin{equation}
    \langle f, g\rangle_{L^2(E)} = \int_{E} f(t)\overline{g(t)} dt,
\end{equation}
and the induced norm is 
\begin{equation}
    \|f\|_{L^2(E)}^2 = \langle f, f\rangle_{L^2(E)} = \int_{E} |f(t)|^2 dt.
\end{equation}




Introduce

Let
\begin{equation}
    C([a,b]) = \{f:[a,b] \rightarrow \mathbb{C}: f \text{ is continuous on } [a,b]\}.
\end{equation}
denote the set of all continuous, scalar-valued functions whose domain is a closed and bounded interval.


Let
\begin{equation*}
C_{\text {per }}[a,b]=\{f \in C[a,b]: f(a)=f(b)\}.
\end{equation*}
denote the set of all periodic, continuous, real-valued, functions on the interval $[a,b]$:


\begin{definition}\label{eq:dot_prod}
    Let $t=(t_1,\dots t_d)$ and $\lambda=(\lambda_1, \dots, \lambda_d)$ be two vectors in $\R^d$. The inner product, denoted $\langle \cdot, \cdot \rangle$, between $t$ and $\lambda$ is the following 
    \begin{equation}
        \langle t, \lambda \rangle = \sum_{n=1}^d t_n \lambda_n = t_1\lambda_1 + \dots + t_d\lambda_d,
    \end{equation}
    also known as the dot product. 
\end{definition}





\end{document}