\documentclass[../thesis.tex]{subfiles}
% Separate preamble for this subfile. This preamble is loaded last, so it may be used to override various functions.

% Better comment extension for Vscode colors these comments differently
% Normal comment color
% * Important information is highlighted
% ! ALERT
% ? Question
% TODO stuff to do
% // this is strikethrough


\begin{document}

% Begynne med L^2 og L^p
% Så ta for seg C([0,1]) og C_per([0,1])
% Hvordan flette inn C(\Omega) ??
\begin{definition}[Lebesgue Space]
    Let $E$ be a measurable subset of $\R^d$. Given $1 \leq p < \infty$ and a measurable function $f:E\rightarrow \C$, we say that $f$ is \emph{$p$-integrable} if the Lebesgue integral of $|f|^p$ is finite, that is
    \begin{equation*}
        \int_{E} |f(t)|^p dt < \infty,
    \end{equation*}
    where $dt= dt_1 \dots dt_d$. We define the \emph{Lebesgue space} $L^p(E)$ as the set of all $p$-integrable functions on $E$, and equip it with the norm
    \begin{equation*}
        \|f\|_{L^p(E)} = \brac{\int_{E} |f(t)|^p dt}^{1/p}. \qedhere
    \end{equation*}
\end{definition}
% Heil REal analysis book page 269, citation?

Of all the $L^p$ spaces, the case $p=2$ is special since $L^2(E)$ is the only space whose norm is induced from the inner product
\begin{equation}
    \langle f, g\rangle_{L^2(E)} = \int_{E} f(t)\overline{g(t)} dt.
\end{equation}
for two functions $f,g\in L^2(E)$. In addition, from the Riesz-Fischer Theorem \cite[p.~279]{heilIntroductionRealAnalysis2019} all $L^p$ spaces are complete with respect to the \LPnorm. Hence, $L^2(E)$ is a Hilbert space. 


\colorbox{BurntOrange}{Overgang}

%* —————————————————————————————————————— C Cont. FUNCTIONS ——————————————————————————————————————
Let
\begin{equation}
    C([a,b]) = \{f:[a,b] \rightarrow \mathbb{C}: f \text{ is continuous on } [a,b]\}
\end{equation}
denote the set of all continuous, scalar-valued functions whose domain is the closed and bounded interval $\bras{a,b}$.

\begin{lemma}\label{lem:c_dense_L2}
    The set of all continuous functions $C([a,b])$ is dense in $L^2([a,b])$ with respect to the \Ltwonorm \space \cite[p.~326]{rudinPrinciplesMathematicalAnalysis20}.
\end{lemma}

\begin{proof}
    $$\dots$$
\end{proof}

%* —————————————————————————————————————— C PERIODIC FUNCTIONS ——————————————————————————————————————
As we advance, let
\begin{equation*}
    \Cper([a,b])=\{f \in C[a,b]: f(a)=f(b)\}
\end{equation*}
denote the set of all periodic, continuous, and real-valued functions on the interval $[a,b]$. Similar to \cref{lem:c_dense_L2} we will show the following

\textcolor{green}{kommentar: a,b ?}
\begin{lemma}\label{lem:c_per_dense_c_and_dense_L2}
    The set of all periodic functions $\Cper([0,1])$ is dense in $L^2([0,1])$ with respect to the \Ltwonorm.
\end{lemma}

\begin{proof}
    \begin{figure} %!! Lag figur
        \centering
        \includegraphics [width=0.45\textwidth]{ntnu.png}
        \caption{Graph}
        \label{fig:g_periodic_close_to_f}
    \end{figure}
    Let $f \in C([0,1])$, $\varepsilon>0$ and $\delta = \varepsilon^2/(8\|f\|_\infty)$. Our goal is to construct a piecewise continuous function $g$ which is identical to $f$ on the interval $[0+\delta,1-\delta ]$ and which attains the value $g = 0$ on the endpoints, making $g\in \Cper$. We will see that this will not change the \Ltwonorm by a negligible amount and only at the expense of two intervals of length $\delta$. In other words, allowing us to approximate any continuous function as closely as we want by a modified (periodic) version of the original function. We construct $g$ in the following way
    \begin{equation*}
        g(t) = 
        \begin{cases} 0, &  t=0,\\  
            \text{linear}, &  0<t<0+\delta,\\ 
            f(t), & 0+\delta \leq t \leq 1-\delta,\\ 
            \text{linear}, &  1-\delta <t<1,\\ 
            0, &  t=1,
        \end{cases}
    \end{equation*}
    % kan lage formel for h på ax+b formel: $ -g(1-\delta) / \delta \cdot x + g(1-\delta)/ \delta $ <- dette er høyre side. $dy/dx \cdot x + b$
    As shown in \cref{fig:g_periodic_close_to_f}, observe that for all $t$ on the interval $[0+\delta, 1-\delta]$ we have $|f(t)-g(t)|= 0$, and for all other values of $t$ we know that
    \begin{equation*}
        |g(t)| \leq \|g\|_{\infty} = \sup_{t\in[0,1]} |g(t)| = \sup_{t\in[0+\delta, 1-\delta]} |g(t)| = \sup_{t\in[0+\delta, 1-\delta]} |f(t)| \leq \sup_{t\in[0, 1]} |f(t)| =\| f\|_{\infty}
    \end{equation*}
    with equality in the third step using the fact that $g$ is linear and goes towards zero as it approaches $0$ or $1$, and therefore attains its greatest value on the interval $[0+\delta,1-\delta]$ where it is equal to $f$. Using this, as well as the triangle inequality, we have that
    \begin{equation*}
        \left|f(t)-g(t) \right| \leq |f(t)| + |-g(t)| \leq \|f \|_{\infty} + \|f \|_{\infty} = 2 \|f \|_{\infty}
    \end{equation*}
    for all $t$ on the intervals $[0, 0+\delta]$ and $[1-\delta,1]$. Now,
    \begin{align*}
        \| f-g \|_{L^2}^2 &=  \int_0^{0+\delta} \left|f(t)-g(t) \right|^2dt + \int_{0+\delta}^{1-\delta} \left|f(t)-g(t) \right|^2dt +\int_{1-\delta}^{1} \left|f(t)-g(t) \right|^2dt\\ 
        &\leq \int_0^{0+\delta} (2 \| f\|_\infty)^2dt + 0 +\int_{1-\delta}^{1} (2 \| f\|_\infty)^2dt\\
        &=  4 \delta \| f\|_\infty^2 + 4 \delta \| f\|_\infty^2\\ 
        &= \varepsilon^2
    \end{align*}
    This shows that $\Cper([0,1])$ is dense in $C([0,1])$. Since $C([0,1])$ is dense in $L^2([0,1])$ from \cref{lem:c_dense_L2}, it follows that $\Cper([0,1])$ is dense in $L^2([0,1])$, which finalizes our proof \cite[p.~228]{heilMetricsNormsInner2018}.
\end{proof}

%* —————————————————————————————————————— Indicator function  ——————————————————————————————————————
\begin{definition}(Indicator function)\label{def:indicator}
    Let $A$ be a subset of a set $X$. The \emph{indicator function}, also known as the \emph{characteristic function}, is a function $\indicator{E}{t}: X \rightarrow \braq{0,1}$ where
    \begin{equation*}
        \indicator{E}{t}  = 
        \begin{cases} 
            1, &  t\in X,\\
            0, &  t \notin X.
        \end{cases}
        \qedhere
    \end{equation*}
\end{definition}


\begin{definition}\label{def:dot_prod}
    Let $t=(t_1,\dots t_d)$ and $\lambda=(\lambda_1, \dots, \lambda_d)$ be two vectors in $\R^d$. The inner product, denoted $\langle \cdot, \cdot \rangle$, between $t$ and $\lambda$ is the following 
    \begin{equation*}
        \langle t, \lambda \rangle = \sum_{n=1}^d t_n \lambda_n = t_1\lambda_1 + \dots + t_d\lambda_d,
    \end{equation*}
    also known as the dot product. 
\end{definition}


\end{document}