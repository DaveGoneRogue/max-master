\documentclass[../thesis.tex]{subfiles}
% Separate preamble for this subfile. This preamble is loaded last, so it may be used to override various functions.

% Better comment extension for Vscode colors these comments differently
% Normal comment color
% * Important information is highlighted
% ! ALERT
% ? Question
% TODO stuff to do
% // this is strikethrough


\begin{document}

% Begynne med L^2 og L^p
% Så ta for seg C([0,1]) og C_per([0,1])
% Hvordan flette inn C(\Omega) ??
\begin{definition}[Lebesgue Space]
    Let $E$ be a measurable subset of $\R^d$. Given $1 \leq p < \infty$ and a measurable function $f:E\rightarrow \C$, we say that $f$ is \emph{$p$-integrable} if the Lebesgue integral of $|f|^p$ is finite, that is
    \begin{equation*}
        \int_{E} |f(t)|^p dt < \infty,
    \end{equation*}
    where $dt= dt_1 \dots dt_d$. We define the \emph{Lebesgue space} $L^p(E)$ as the set of all $p$-integrable functions on $E$, and equip it with the norm
    \begin{equation*}
        \|f\|_{L^p(E)} = \brac{\int_{E} |f(t)|^p dt}^{1/p}.
    \end{equation*}
\end{definition}
% Heil REal analysis book page 269, citation?

Of all the $L^p$ spaces, the case $p=2$ is special since $L^2(E)$ is the only space whose norm is induced from the inner product
\begin{equation}
    \langle f, g\rangle_{L^2(E)} = \int_{E} f(t)\overline{g(t)} dt.
\end{equation}
for two functions $f,g\in L^2(E)$. In addition, from the Riesz-Fischer Theorem \cite[p.~279]{heilIntroductionRealAnalysis2019} all $L^p$ spaces are complete with respect to the \LPnorm. Hence, $L^2(E)$ is a Hilbert space. 


\colorbox{BurntOrange}{Overgang}


Let
\begin{equation}
    C([a,b]) = \{f:[a,b] \rightarrow \mathbb{C}: f \text{ is continuous on } [a,b]\}
\end{equation}
denote the set of all continuous, scalar-valued functions whose domain is the closed and bounded interval $\bras{0,1}$.

\begin{lemma}\label{lem:c_dense_L2}
    The set of all continius functions $C([0,1])$ is dense in $L^2([0,1])$ with respect to the \Ltwonorm \cite[p.~326]{rudinPrinciplesMathematicalAnalysis20}.
\end{lemma}

\begin{proof}
    $$\dots$$
\end{proof}

As we advance, let
\begin{equation*}
C_{\text {per }}[a,b]=\{f \in C[a,b]: f(a)=f(b)\}
\end{equation*}
denote the set of all periodic, continuous, and real-valued functions on the interval $[a,b]$. Similar to \cref{lem:c_dense_L2} we will show the following

\begin{lemma}
    The set of all periodic functions $C_{\text {per}}[0,1]$ is dense in $L^2([0,1])$ with respect to the \Ltwonorm.
\end{lemma}

\begin{proof}\cite[p.~228]{heilMetricsNormsInner2018}
    Let $f \in C[0,1]$, $\delta = \varepsilon^2/(8\|f\|_\infty)$, and construct $g$ such that $g\in C_{\text{per}}$ in the following way %We will construct a function $g$ that belongs to $C_{\text{per}}[0,1]$. Let  $\delta = \varepsilon^2/(8\|f\|_\infty)$, and $g$  be given by
%\begin{equation*}
%    \| g-h \|_{L^2([0,1])}^2 = \int_0^1 \left|g(t)-h(t) \right|^2dt < \varepsilon^2.
%\end{equation*}
%Let  $\delta = \varepsilon^2/(8\|g\|_\infty)$, and $h$  be given by
\begin{figure}
    \centering
    \includegraphics [width=0.45\textwidth]{ntnu.png}
    \caption{Graph}
    \label{fig:cantor_2d}
\end{figure}

\begin{equation*}
    g(t) = 
    \begin{cases} 0, &  t=0,\\  
        \text{linear}, &  0<t<0+\delta,\\ 
        f(t), & 0+\delta \leq t \leq 1-\delta,\\ 
        \text{linear}, &  1-\delta <t<1,\\ 
        0, &  t=1,
    \end{cases}
\end{equation*}
% kan lage formel for h på ax+b formel: $ -g(1-\delta) / \delta \cdot x + g(1-\delta)/ \delta $ <- dette er høyre side. $dy/dx \cdot x + b$
Note that on the interval $[0+\delta, 1-\delta]$ we have $f(t)-g(t)= 0$ for all $t$, and for all other values of $t$ the 'distance' is at worst 
\begin{equation*}
    \left|f(t)-g(t) \right| \leq |f(t)| + |-g(t)| \leq \|f \|_{\infty} + \|f \|_{\infty} = 2 \|f \|_{\infty}
\end{equation*}
by first using the triangle inequality and then that
\begin{equation*}
    |g(t)| \leq \|g\|_{\infty} = \sup_{t\in[0,1]} |g(t)| = \sup_{t\in[0+\delta, 1-\delta]} |g(t)| = \sup_{t\in[0+\delta, 1-\delta]} |f(t)| \leq \sup_{t\in[0, 1]} |f(t)| =\| f\|_{\infty}
\end{equation*}
with equality in the second step from the fact that $h$ is linear and goes towards zero as it approaches $0$ or $1$, and therefore attains its greatest value on the interval $[0+\delta,1-\delta]$. Now,
\begin{align*}
    \| f-g \|_{L^2}^2 &=  \int_0^{0+\delta} \left|f(t)-g(t) \right|^2dt + \int_{0+\delta}^{1-\delta} \left|f(t)-g(t) \right|^2dt +\int_{1-\delta}^{1} \left|f(t)-g(t) \right|^2dt\\ 
    &\leq \int_0^{0+\delta} (2 \| f\|_\infty)^2dt + 0 +\int_{1-\delta}^{1} (2 \| f\|_\infty)^2dt\\
    &=  4 \delta \| f\|_\infty^2 + 4 \delta \| f\|_\infty^2\\ 
    &= \varepsilon^2
\end{align*}

closing remarks on the span of periodic polynomials and approximations to continuous functions
\end{proof}



\begin{definition}\label{def:dot_prod}
    Let $t=(t_1,\dots t_d)$ and $\lambda=(\lambda_1, \dots, \lambda_d)$ be two vectors in $\R^d$. The inner product, denoted $\langle \cdot, \cdot \rangle$, between $t$ and $\lambda$ is the following 
    \begin{equation*}
        \langle t, \lambda \rangle = \sum_{n=1}^d t_n \lambda_n = t_1\lambda_1 + \dots + t_d\lambda_d,
    \end{equation*}
    also known as the dot product. 
\end{definition}


\begin{definition}[Span]\label{def:span}
    Let $E$ be a subset of a vector space $X$. The \emph{finite linear span}, or \emph{span} for short, is the set of all finite linear combinations of $E$, that is
    \begin{equation*}
        \spn{E}=\braqMed{\sum_{n=1}^N C_n v_n : N>0, v_n \in E, C_i \in \mathbb{F}}.
    \end{equation*}
    We say that $E$ \emph{spans} $X$ if $\spn{E} = X$.
\end{definition}


\begin{definition}[Closed Span]\label{def:closed_span}
    Let $E$ be a subset of a normed space $X$. The \emph{closed linear span} or \emph{closed span} of $E$ is the closure of $\spnclos{E}$, that is
    \begin{equation*}
        \spnclos{E} = \braqMed{y\in X: \exists y_i\in \spn{E} \quad \text{ s.t } \quad y_i \longrightarrow y} \qedhere
    \end{equation*}
\end{definition}


\begin{definition}[Closure]\label{def:closure}
    Let $X$ be an arbitrary set. Given a subset $E \subseteq X$, the \emph{closure} of $E$, denoted $\overline{E}$, is the smallest closed set containing $E$. The closure is expressed as the intersection of all closed sets containing $E$. That is the set
    \begin{equation*}
        \overline{E} = \bigcap \braqMed{F \subseteq X : F \text{ closed, and } E \subseteq F}.
    \end{equation*}
    Furthermore, if $\overline{E} = X$ then we say that $E$ is \emph{dense} in $X$. Also, if $E$ is closed, then $\overline{E}=E$. In this regard, also note that any intersection of closed sets is closed, implying that $\overline{E}$ is closed.
\end{definition}
\begin{remark}
    Note that if $E$ is a subset of a metric space $X$, then
    \begin{equation*}
        \overline{E} = \{ y \in X : \exists y_n\in E \quad \text{ s.t }\quad y_n \longrightarrow y\}.
    \end{equation*}
    %proof of the above can be found in CITE: Linmet bok, page 67.
\end{remark}


\end{document}