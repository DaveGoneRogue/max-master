\documentclass[../thesis.tex]{subfiles}
% Seperate preamble for this subfile. This preamble is loaded last, so it may be used to override various functions.

% Better comment extension for Vscode colors these comments differently
% Normal comment color
% * Important information is highlighted
% ! ALERT
% ? Question
% TODO stuff to do
% // this is strikethrough


\begin{document}

% Begynne med L^2 og L^p
% Så ta for seg C([0,1]) og C_per([0,1])
% Hvordan flette inn C(\Omega) ??

%? Sjekk hvordan denne er introdusert i HEIL
Let $C([0,1])$ denote the set of all continuous, scalar-valued functions whose domain is $[0,1]$:
\begin{equation}
    C([0,1]) = \{f:[0,1] \rightarrow \mathbb{C}: f \text{ is continuous on } [0,1]\}.
\end{equation}

Let
\begin{equation}
    C([0,1]) = \{f:[0,1] \rightarrow \mathbb{C}: f \text{ is continuous on } [0,1]\}.
\end{equation}
denote the set of all continuous, scalar-valued functions whose domain is $[0,1]$:



Let
\begin{equation*}
C_{\text {per }}[0,1]=\{f \in C[0,1]: f(0)=f(1)\}.
\end{equation*}
denote the set of all periodic, continuous, real-valued, functions on the interval $[0,1]$:



\begin{definition}\label{eq:dot_prod}
    Let $t=(t_1,\dots t_d)$ and $\lambda=(\lambda_1, \dots, \lambda_d)$ be two vectors in $\R^d$. The inner product, denoted $\langle \cdot, \cdot \rangle$, between $t$ and $\lambda$ is the following 
    \begin{equation}
        \langle t, \lambda \rangle = \sum_{n=1}^d t_n \lambda_n = t_1\lambda_1 + \dots + t_d\lambda_d,
    \end{equation}
    also known as the dot product. 
\end{definition}





\end{document}