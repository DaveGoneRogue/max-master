\documentclass[../thesis.tex]{subfiles}
% Seperate preamble for this subfile. This preamble is loaded last, so it may be used to override various functions.

% Better comment extension for Vscode colors these comments differently
% Normal comment color
% * Important information is highlighted
% ! ALERT
% ? Question
% TODO stuff to do
% // this is strikethrough


\begin{document}
Let $I^2$ denote the unit square $[0,1]^2$. Going forward, our goal is to show that $\left\{ e^{2\pi i \langle \lambda,t  \rangle } : \lambda \in \mathbb{Z}^2\right\}$ is an ONB for $L^2{(I^2)}$. Note that $\lambda=(\lambda_1,\lambda_2)$ and $t=(x,y)$. We begin with orthogonality. Since the complex expontential functions are continuous and hence Lebesgue integrable on $I^2$ we have
\begin{align*} 
    \langle e_\lambda,e_{\lambda'} \rangle_{L^2} &= \int_0^1\int_0^1 e_{\lambda}(t) \overline{e_{\lambda'}(t)} dy dx\\ 
    &= \int_0^1\int_0^1 e^{2\pi i \langle \lambda,t\rangle } e^{-2\pi i  \langle \lambda',t\rangle} dy dx\\ 
    &= \int_0^1\int_0^1 e^{2\pi i  (\lambda_1x + \lambda_2 y)} e^{-2\pi i  (\lambda_1' x + \lambda_2' y)} dy dx\\ 
    &= \int_0^1\int_0^1 e^{2\pi i  (\lambda_1- \lambda_1')x} e^{2\pi i  (\lambda_2 - \lambda_2')y} dy dx\\ 
    &= \int_0^1e^{2\pi i  (\lambda_1- \lambda_1')x}\int_0^1  e^{2\pi i  (\lambda_2 - \lambda_2')y} dy dx
\end{align*}
%using \cref{eq:dot_prod} and rearanging. 
Let $\lambda_2 \neq \lambda_2'$. Observe that the inner integral can be solved with the exact same substitution as for 1D with $u = 2 \pi i (\lambda_2 - \lambda_2')y$, which we have already shown is equal to zero. In other words, the inner integral is zero, and we have $\langle e_\lambda,e_{\lambda'} \rangle_{L^2} = 0$. Let $\lambda_2 = \lambda_2'$. Solving the inner integral yields
\begin{align*}
    \int_0^1  e^{2\pi i  (\lambda_2 - \lambda_2)y} dy&= \int_0^1 e^0 dy = y \big| _0^1 = 1.
\end{align*}

We are then left with only the outer integral. Let $\lambda_1 \neq \lambda_1'$ and observe that this is the same integral as for the case of $\lambda_2 \neq \lambda_2'$. Using an equivalent substitution yields the same result, and we have that $\langle e_\lambda,e_{\lambda'} \rangle_{L^2} = 0$. Let $\lambda_1 = \lambda_1'$. Observe that we then have the case $\langle e_\lambda,e_{\lambda} \rangle_{L^2} = \| e_\lambda \|^2_{L^2}$, and that 


which shows that $\left\{ e^{2\pi i \langle \lambda,x  \rangle } : \lambda \in \mathbb{Z}^2\right\}$ is not only orthogonal, it is also orthonormal.


Let $f\in L^2(I^2)$ and assume that $f$ is orthogonal to $\spn{\left\{ e^{2\pi i \langle \lambda,x  \rangle} : \lambda \in \mathbb{Z}^2\right\}}$. We then have
\begin{align*}
    \langle e_\lambda, f \rangle_{L^2(I^2)} &= \int_0^1 \int_0^1 e_\lambda(x,y) \overline{f(x,y)} dydx \\   
    &= \int_0^1 \int_0^1 e^{2\pi i  (\lambda_1x + \lambda_2 y)} \overline{f(x,y)} dydx \\
    &= \int_0^1 e^{2 \pi i \lambda_1 x}\int_0^1 e^{2 \pi i \lambda_2 y} \overline{f(x,y)} dydx \\
    &= 0
\end{align*}

%! SIGRID: NEi Dette er ikke rett
for all $\brac{\lambda_1,\lambda_2} \in \Z$. To see this, 


fix $\lambda_2$ and observe when the integral is equal to zero for all $\lambda_1\in \Z$, which can only happen if the inner integral is zero. That is 
\begin{equation*}
    0=\int_0^1 e^{2 \pi i \lambda_2 y} \overline{f(x,y)} dy  =: F(x)
\end{equation*}
for almost all $x\in[0,1]$. This is because the only function satisfying 
\begin{equation*}
    \int_0^1 e^{2 \pi i \lambda_1 x} F(x)dx = \langle e_{\lambda_1}, \overline{F(x)}\rangle_{L^2(I)} = 0
\end{equation*}
for all $\lambda_1 \in \Z$ is $F(x)=0$, since $\brac{I,\Z}$ is a spectral pair. Note that $F(x)\in L^2(I)$ since
\begin{align*}
    \|F(x)\|_{L^2(I^2)}^2 =&  \int_0^1 \int_0^1 F(x) \overline{F(x)} dydx
    = \int_0^1 | F(x)|^2 \int_0^1 dy dx = \int_0^1 | F(x)|^2 dx\\
    =& \int_0^1 \left| \int_0^1 e^{2 \pi i \lambda_2 y} \overline{f(x,y)} dy \right|^2 dx 
    = \int_0^1 \left| \langle e^{2 \pi i \lambda_2 y} , f(x,y) \rangle_{L^2(I)} \right|^2 dx\\
    \leq & \int_0^1 \left( \|e^{2 \pi i \lambda_2 y} \|_{L^2(I)} \|f(x,y) \|_{L^2(I)} \right)^2dx
    = \int_0^1 | 1|^2 \|f(x,y) \|_{L^2(I)}^2 dx\\
    = & \|f(x,y) \|_{L^2(I^2)}^2
\end{align*}

%? Sigrid: STRYK
% where we have used Cauchy-Schwartz in the first inequality, and that the exponential has $\|\cdot\| = 1$ as shown in \cref{eq:exp_norm_one}. Note that $\|f(x,y) \|_{L^2(I^2)}^2 =\int_0^1 \int_0^1 |\overline{f(x,y)}|^2 dy dx$ gives the final equality, and the final result that $\|F(x)\|_{L^2(I^2)}^2 \leq \|f(x,y) \|_{L^2(I^2)}^2$


%! Dette er og feil
%As $\lambda_2$ was arbitrarily fixed, the only function satifying $F(x) = \int_0^1 e^{2 \pi i \lambda_2 y} \overline{f(x,y)} dy = 0 $ for all $\lambda_2 \in \Z$ is $f(x,y)=0$, since $\brac{I,\Z}$ is a spectral pair. As $f(x,y)=0$ is the only solution for all $\lambda \in \Z$, it means that $\left\{ e^{2\pi i \langle \lambda,x  \rangle } : \lambda \in \mathbb{Z}^2\right\}$ is complete. 


\end{document}