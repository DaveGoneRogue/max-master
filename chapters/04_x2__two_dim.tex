\documentclass[../thesis.tex]{subfiles}
% Seperate preamble for this subfile. This preamble is loaded last, so it may be used to override various functions.

% Better comment extension for Vscode colors these comments differently
% Normal comment color
% * Important information is highlighted
% ! ALERT
% ? Question
% TODO stuff to do
% // this is strikethrough


\begin{document}

% ? Intro overgang, noe for å introdusere d-dimensjoner. 
% ? Before expanding up to higher dimensions, let us form the following defenitions.


\begin{definition}[Spectral set \cite{liuUniformityNonUniformGabor2003}] \label{def:spectral_set}
    Let $\Omega \subseteq \mathbb{R}^d$ be a measurable subset with positive and finite (bounded) measure, i.e $0< \operatorname{mes} \Omega < \infty$, and let $e_{\lambda}$ be an exponential function on $\Omega$ given by $e_{\lambda}(t) = e^{2\pi i \langle \lambda,t  \rangle }$. If the set of exponentials $\{ e^{2\pi i \langle \lambda,t  \rangle } : \lambda \in \Lambda\}$  form an \textsc{orthogonal} basis for $L^2 (\Omega)$ when $\lambda$ ranges over some subset $\Lambda$ in $\mathbb{R}^d$, then $\Omega$ is called a \emph{spectral set} and $\Lambda$ is called a \emph{spectrum} for $\Omega$. We say that $(\Omega, \Lambda)$ is a \emph{spectral pair}. 
\end{definition} 

% ! Here is incosistent spacing

Note, even though the exponential function $e^{2\pi i \langle \lambda,t  \rangle }$ is defined over $\mathbb{C}$, the inner product is defined over $\R^d$ as $t\in \Omega\subset \R^d$ and $\lambda \in \Lambda \subset \R^d$, and can be calculated using \cref{eq:dot_prod}.

% ! Note that a spectral set $\Omega$ may have more than one spectrum.
% ! Spectral pairs are connected to tilings.

In other words we have shown that $\brac{\Omega,\mathbb{Z}}$ is a spectral pair. This follows directly from \cref{def:spectral_set} with $\Omega = [0,1]$, $\Lambda = \Z $, $d=1$, and in conjunction with the proof that $\{ e^{2\pi i \lambda t} : \lambda \in \mathbb{Z} \}$ is an orthonormal basis for $L^2([0,1])$. As shown in \cite{taoIntroductionMeasureTheory2011}, closed sets are measurable, in our case with positive and finite measure $\operatorname{mes}(\Omega) = 1$. % * using Lebesgue measure, state this?


% TODO
% Todo: notation switch from $\{ e_\lambda \}_{\lambda \in \Z}$ to $\{e^{2 \pi i \dots \lambda \dots} : \lambda \in \Z \}$, go back and change
% Todo: sjekke spacing, for nå er det ikke konsistent: Ny linje, ny linje med spacing, ny lince med spacing etterfulgt av theorem enviroment, etc


% ! Def tiling ??

%\begin{definition}[Tiling set]
%    Let $\Omega \subset \mathbb{R}^d$ be a subset with nonzero measure, and consider a set $T \subseteq \mathbb{R}^d$. If the set of translates ${}$$\{\Omega+l: l\in T\}$ cover $\mathbb{R^d}$ up to measure zero, and if all intersections $(\Omega+l) \cap (\Omega+l')$  for $l\neq l'$ in $L$ have measure zero, then $\Omega$ is called a \emph{tile}, and $T$ is called a \emph{tiling set} for $\Omega$. We say that $(\Omega, T)$ is a \emph{tiling pair}. 
%\end{definition}

%In other words, the shifts $\Omega + T$ constitute a \emph{measuredisjoint covering} of $\mathbb{R}^d$, and we can say that $\Omega$ \emph{tiles} $\mathbb{R}^d$ \emph{by translation}, or that $\Omega+T$ is a \emph{tiling} of $\mathbb{R}^d$.


% Fuglede’s spectral set conjecture. Let $\Omega$􏱁 be a set in $\mathbb{R}^d$ with positive and finite Lebesgue measure. Then 􏱁$\Omega$ is a spectral set if and only if $\Omega$􏱁 tiles $\mathbb{R}^d$ by translation.

% An equivalent restatment of Fuglede was presented in JorgenPedersen. 

% Let $\Omega \subset \mathbb{R}^d$ have positive and finite Lebesgue measure. Then $\Omega$ is a spectral set if and only if $\Omega$ is a tile, i.e., there exists a set $\Lambda$ so that $(\Omega, \Lambda)$ is a spectral pair if and only if there exists a set $\Lambda^{\prime}$ so that $\left(\Omega, \Lambda^{\prime}\right)$ is a tiling pair.

%with the following dual conjectures.

% Conjecture 1.3: Let $\Lambda \subset \mathbb{R}^d$. Then $\Lambda$ is a spectrum if and only if $\Lambda$ is a tiling set, i.e., there exists a set $\Omega$ so that $(\Omega, \Lambda)$ is a spectral pair if and only if there exists a set $\Omega^{\prime}$ so that $\left(\Omega^{\prime}, \Lambda\right)$ is a tiling pair.

% Conjecture 1.4: Let $\Lambda \subset \mathbb{R}^d$. Then $\left(I^d, \Lambda\right)$ is a spectral pair if and only if $\left(I^d, \Lambda\right)$ is a tiling pair.

%When  $\Omega = I^d$ the connection between tiles and spectrum is more direct than for other examples of sets $\Omega$. As shown in sigrid_note:lagarias_reeds_wang and Spectral and tiling properties of the unit cube_ALEX IOSEVICH AND STEEN PEDERSEN it is possible to classify all spectra by showing that $\Lambda$ is a spectrum for the unit cube $I^d$, if and only if $I^d$ tiles $\mathbb{R}^d$ by $\Lambda$-translates.


Let $I^2$ denote the unit square $[0,1]^2$. Going forward, our goal is to show that $\left\{ e^{2\pi i \langle \lambda,t  \rangle } : \lambda \in \mathbb{Z}^2\right\}$ is an ONB for $L^2{(I^2)}$. Note that $\lambda=(\lambda_1,\lambda_2)$ and $t=(x,y)$. We begin with the orthogonality. Since the complex expontential functions are continuous and hence Lebesgue integrable on $I^2$ we have the following
\begin{align*} 
    \langle e_\lambda,e_{\lambda'} \rangle_{L^2} &= \int_0^1\int_0^1 e_{\lambda}(t) \overline{e_{\lambda'}(t)} dy dx\\ 
    &= \int_0^1\int_0^1 e^{2\pi i \langle \lambda,t\rangle } e^{-2\pi i  \langle \lambda',t\rangle} dy dx\\ 
    &= \int_0^1\int_0^1 e^{2\pi i  (\lambda_1x + \lambda_2 y)} e^{-2\pi i  (\lambda_1' x + \lambda_2' y)} dy dx\\ 
    &= \int_0^1\int_0^1 e^{2\pi i  (\lambda_1- \lambda_1')x} e^{2\pi i  (\lambda_2 - \lambda_2')y} dy dx\\ 
    &= \int_0^1e^{2\pi i  (\lambda_1- \lambda_1')x}\int_0^1  e^{2\pi i  (\lambda_2 - \lambda_2')y} dy dx
\end{align*}

using \cref{eq:dot_prod} and rearanging. Let $\lambda_2 \neq \lambda_2'$. Observe that the final inner integral can be solved with the exact same substitution as for 1D with $u = 2 \pi i (\lambda_2 - \lambda_2')y$, which we have already shown is equal to zero. In other words, the inner integral is zero, and we have $\langle e_\lambda,e_{\lambda'} \rangle_{L^2} = 0$. Let $\lambda_2 = \lambda_2'$. Solving the inner integral yields
\begin{align*}
    \int_0^1  e^{2\pi i  (\lambda_2 - \lambda_2)y} dy&= \int_0^1 e^0 dy = y \big| _0^1 = 1.
\end{align*}

We are then left with only the outer integral. Let $\lambda_1 \neq \lambda_1'$ and observe that this is essentially the same integral as for the case of $\lambda_2 \neq \lambda_2'$. Using an equivalent substitution yields the same result, and we have that $\langle e_\lambda,e_{\lambda'} \rangle_{L^2} = 0$. Let $\lambda_1 = \lambda_1'$. Observe that we then have the case $\langle e_\lambda,e_{\lambda} \rangle_{L^2} = \| e_\lambda \|^2_{L^2}$, and that 
\begin{align*}
    \langle e_\lambda, e_\lambda\rangle_{L^2} &= \int_0^1 \int_0^1 e_\lambda \overline{e_\lambda} dydx = \int_0^1 \int_0^1 |e_\lambda|^2 dydx \\
    &=\int_0^1\int_0^1 |e^{2 \pi i \lambda t}|^2dydx = \int_0^1\int_0^1 |1|^2 dydx  \\
    &= 1
\end{align*}

which shows that $\left\{ e^{2\pi i \langle \lambda,x  \rangle } : \lambda \in \mathbb{Z}^2\right\}$ is not only orthogonal, it is also orthonormal.


Let $f\in L^2(I^2)$ and assume that $f$ is it is orthogonal to $\spn{\left\{ e^{2\pi i \langle \lambda,x  \rangle} : \lambda \in \mathbb{Z}^2\right\}}$. We then have
\begin{align*}
    \langle e_\lambda, f \rangle_{L^2(I^2)} &= \int_0^1 \int_0^1 e_\lambda(x,y) \overline{f(x,y)} dydx \\   
    &= \int_0^1 \int_0^1 e^{2\pi i  (\lambda_1x + \lambda_2 y)} \overline{f(x,y)} dydx \\
    &= \int_0^1 e^{2 \pi i \lambda_1 x}\int_0^1 e^{2 \pi i \lambda_2 y} \overline{f(x,y)} dydx \\
    &= 0
\end{align*}
for all $\brac{\lambda_1,\lambda_2} \in \Z$. To see this, fix $\lambda_2$ and observe when the integral is equal to zero for all $\lambda_1\in \Z$, which can only happen if the inner integral is zero. That is 
\begin{equation*}
    0=\int_0^1 e^{2 \pi i \lambda_2 y} \overline{f(x,y)} dy  =: F(x)
\end{equation*}
for almost all $x\in[0,1]$. This is because the only function satisfying 
\begin{equation*}
    \int_0^1 e^{2 \pi i \lambda_1 x} F(x)dx = \langle e_{\lambda_1}, \overline{F(x)}\rangle_{L^2(I^2)} = 0
\end{equation*}
for all $\lambda_1 \in \Z$ is $F(x)=0$, since $\brac{I,\Z}$ is a spectral pair. Note that $F(x)\in L^2(I^2)$ since
\begin{align*}
    \|F(x)\|_{L^2(I^2)}^2 =&  \int_0^1 \int_0^1 F(x) \overline{F(x)} dydx
    = \int_0^1 | F(x)|^2 \int_0^1 dy dx = \int_0^1 | F(x)|^2 dx \\
    =& \int_0^1 \left| \int_0^1 e^{2 \pi i \lambda_2 y} \overline{f(x,y)} dy \right|^2 dx 
    = \int_0^1 \left| \langle e^{2 \pi i \lambda_2 y} , f(x,y) \rangle_{L^2(I)} \right|^2 dx\\
    \leq & \int_0^1 \left( \|e^{2 \pi i \lambda_2 y} \|_{L^2(I)} \|f(x,y) \|_{L^2(I)} \right)^2dx
    = \int_0^1 | 1|^2 \|f(x,y) \|_{L^2(I)}^2 dx \\
    = & \|f(x,y) \|_{L^2(I^2)}^2
\end{align*}
where we have used Cauchy-Schwartz in the first inequality, and that the exponential has $\|\cdot\| = 1$ as shown in \cref{eq:exp_norm_one}. Note that $\|f(x,y) \|_{L^2(I^2)}^2 =\int_0^1 \int_0^1 |\overline{f(x,y)}|^2 dy dx$ gives the final equality, and the final result that $\|F(x)\|_{L^2(I^2)}^2 \leq \|f(x,y) \|_{L^2(I^2)}^2$

As $\lambda_2$ was arbitrarily fixed, the only function satifying $F(x) = \int_0^1 e^{2 \pi i \lambda_2 y} \overline{f(x,y)} dy = 0 $ for all $\lambda_2 \in \Z$ is $f(x,y)=0$, since $\brac{I,\Z}$ is a spectral pair. As $f(x,y)=0$ is the only solution for all $\lambda \in \Z$, it means that $\left\{ e^{2\pi i \langle \lambda,x  \rangle } : \lambda \in \mathbb{Z}^2\right\}$ is complete. 


\end{document}