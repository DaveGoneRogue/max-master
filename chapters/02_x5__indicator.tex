\documentclass[../thesis.tex]{subfiles}
% Separate preamble for this subfile. This preamble is loaded last, so one can override various functions before \begin{document}

% Better comment extension for Vscode colors these comments differently
% Normal comment color
% * Important information
% ! ALERT
% ? Question
% TODO stuff to do
% // This is strikethrough


\begin{document}
\textcolor{orange}{  %! Here
Throughout the theses, we use the following terminology and notation for the inner product and the indicator function.
\begin{definition}\label{def:dot_prod}
    Let $t=(t_1,\dots t_d)$ and $\lambda=(\lambda_1, \dots, \lambda_d)$ be two vectors in $\R^d$. The inner product, denoted $\braa{\cdot, \cdot}$, between $t$ and $\lambda$ is
    \begin{equation*}
        \braaMed{t, \lambda} = \sum_{n=1}^d t_n \lambda_n = t_1\lambda_1 + \dots + t_d\lambda_d,
    \end{equation*}
    also known as the dot product. 
\end{definition}
} %! Here

\textcolor{orange}{  %! Here
We define the indicator function as follows. 
\begin{definition}(Indicator function)\label{def:indicator}
    Let $E$ be a subset of a set $X$. The \emph{indicator function}, also known as the \emph{characteristic function} of $E$, is a function $\indicator{E}{t}: X \rightarrow \braq{0,1}$ where
    \begin{equation*}
        \indicator{E}{t}  = 
        \begin{cases} 
            1, &  t\in E,\\
            0, &  t \notin E.
        \end{cases}
        \qedhere
    \end{equation*}
\end{definition}
} %! Here

%* Recall: Heine Borel gives us that compact is equivalent to being closed and bounded for all subsets of R.
Now, given a compact subset $\Omega\subset \R^d$, consider the $d$-dimensional Fourier transform 
\begin{equation*}
    \hat{f}(z) = \int_{\R^d} f(t)  e^{-2\pi i \braa{z,t}} dt,
\end{equation*}
where $z = (z_1, \dots, z_j) \in \C^d$ and $t = (t_1, \dots, t_j) \in \R^d$, of the indicator function
\begin{equation*}
    \indicatorFourier{\Omega}{z} = \int_{\R^d} \indicator{\Omega}{t}  e^{-2\pi i \braa{z,t}} dt.
\end{equation*} 
Note that $\indicatorFourier{\Omega}{z}$ is an entire function for any $z\in \C^d$ since $\Omega$ is compact. % Think the case where JP says: "It is defined for any real vector ($Z\in \R^d$), because $\Omega$ has finite measure and that the absolute value of $e^{-2i\pi \braa{z,t}}$ is one." Two reasons for that formulation. First, He considers real vectors $z$ (although they are of a complex value function). Secondly, we must thus consider the absolute value of the exponential, which is real. 
Furthermore, we use the following notation to denote the \emph{zero-set} of $\indicatorFourier{\Omega}{z}$,
\begin{equation*}
    \Zstroke_{\Omega} = \braqMed{ z \in \C^d \setminus \braqMed{0} : \indicatorFourier{\Omega}{z} = 0}.
\end{equation*}

Before showing a useful result, we can, without loss of generality and ease of notation, redefine our function to be
\begin{equation*}
    \indicatorFourier{\Omega}{z} = \int_{\R^d} \indicator{\Omega}{t}  e^{-2\pi i \braa{z,t}} dt = \int_{\Omega} e^{-2\pi i \braa{z,t}} dt := F_{\Omega}(z)
\end{equation*}

\begin{lemma}
    Let $\Omega$ be BLALBA in $\R^d$. A set $\Lambda$ gives an orthoganl set of exponentials $E(\Lambda)$ in $L^2(\Omega)$ if and only if
    $\Lambda - \Lambda \subseteq \Zstroke_{\Omega} \cup \braq{0}$


    \SigridComment{Se på non zero elements of direkte undermengde}
\end{lemma}

\end{document}