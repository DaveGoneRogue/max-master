\documentclass[../thesis.tex]{subfiles}
% Separate preamble for this subfile. This preamble is loaded last, so one can override various functions before \begin{document}

% Better comment extension for Vscode colors these comments differently
% Normal comment color
% * Important information
% ! ALERT
% ? Question
% TODO stuff to do
% // This is strikethrough


\begin{document}
%We define the indicator function as follows. 
\begin{definition}[Indicator function]\label{def:indicator}
    Let $E$ be a subset of a set $X$. The \emph{indicator function}, also known as the \emph{characteristic function}, of $E$ is a function $\indicatorNoVar{E}: X \rightarrow \braq{0,1}$ where
    \begin{equation*}
        \indicator{E}{t}  = 
        \begin{cases} 
            1, &  t\in E,\\
            0, &  t \notin E.
        \end{cases}
        \qedhere
    \end{equation*}
\end{definition}

%* Recall: Heine Borel gives us that compact is equivalent to being closed and bounded for all subsets of R.
% Now, given a
Given a bounded subset $\Omega\subset \R^d$ with finite and positive measure, consider the $d$-dimensional Fourier transform given by
\begin{equation*}
    \hat{f}(z) = \int_{\R^d} f(t)  e^{-2\pi i \braa{z,t}} dt, \quad z,t\in \R^d,
\end{equation*}
of the indicator function. That is, 
\begin{equation*}
    \indicatorFourier{\Omega}{z} = \int_{\R^d} \indicator{\Omega}{t}  e^{-2\pi i \braa{z,t}} dt, \quad z,t\in \R^d,
\end{equation*} 
which we redefine into the following function $F_{\Omega}(z)$;
\begin{equation}\label{eq:f_omega}
    \indicatorFourier{\Omega}{z} = \int_{\R^d} \indicator{\Omega}{t}  e^{-2\pi i \braa{z,t}} dt = \int_{\Omega} e^{-2\pi i \braa{z,t}} dt := F_{\Omega}(z).
\end{equation}
Since $\Omega$ has finite measure and the exponential has absolute value one, we have that $F_{\Omega}(z)$ is defined for any $z \in \R^d$. Note that the authors \cite{jorgensenSpectralPairsCartesian2001} define $F_{\Omega}$ without the minus sign in the exponent. Additionally, we introduce the following notation to refer to the \emph{set of real zeroes}, or simply the \emph{zero-set}, of $F_{\Omega}(z)$;
%Additionally, we introduce the following notation to refer to the \emph{set of real zeroes} of $F_{\Omega}(z)$, which we will simply name the \emph{zero-set} as all our results apply to $\R$;
\begin{equation*}
    \Zstroke_{\Omega} = \braqMed{ z  = (z_1, \dots z_d) \in \R^d \setminus \braqMed{0} : F_{\Omega}(z)= 0}.
\end{equation*}
To conclude this chapter, we show a simple, yet useful, relationship between orthogonality and the zero-set of $F_{\Omega}(z)$ \cite{lagariasOrthonormalBasesExponentials2000,jorgensenSpectralPairsCartesian2001}. %! nevertheless, instead of yet


\begin{lemma}\label{lem:zero_set_orthoganl_general}
    Let $\Omega \subset \R^d$ be a bounded subset with $0< \mes{\Omega}<\infty$, and consider the set
    \begin{equation}\label{eq:zero_set_orthogonal}
        \Lambda - \Lambda = \braqMed{\lambda-\lambda' : \lambda,\lambda' \in \Lambda}.
    \end{equation}
    for some subset $\Lambda\subset \R^d$. Then $\Lambda$ gives an orthogonal set of exponentials $E(\Lambda)$ if and only if
    \begin{equation}\label{eq:zero_set_inclusion_original}
        \Lambda - \Lambda \subseteq \Zstroke_{\Omega} \cup \braq{0}.
    \end{equation}
\end{lemma}

% of two distinct elements $\lambda,\lambda'\in \Lambda$          for $z\neq0$
\begin{proof}
    As a preliminary step to the proof, observe the following equality between the inner product $\braaMed{e_{\lambda},e_{\lambda'} }_{L^2(\Omega)}$ and $F_{\Omega}(z)$, under the assumption that $\lambda,\lambda'\in \Lambda$ are distinct elements, i.e., $\lambda \neq \lambda'$. Equivalently one can assume $z = \lambda - \lambda' \neq0$ and show the equality the other way.  % (since $\lambda - \lambda' = 0$ implies $\lambda = \lambda'$). 
    \begin{align*}
        \braaMed{e_{\lambda},e_{\lambda'} }_{L^2(\Omega)} =& \int_{\Omega} e_{\lambda}(t)\overline{e_{\lambda'}(t)} dt\\
        =& \int_{\Omega} e^{2\pi i \braa{\lambda,t}} e^{-2 \pi i \braa{\lambda',t}} dt\\
        =& \int_{\Omega} e^{2\pi i  (\lambda_1t_1 + \dots +\lambda_d t_d)} e^{-2\pi i  (\lambda_1' t_1 + \dots +\lambda_d' t_d)} dt\\
        =& \int_{\Omega} e^{2\pi i  (\lambda_1 -\lambda_1')t_1} \cdots e^{2\pi i  (\lambda_d -\lambda_d')t_d} dt\\
        %=& \int_{\Omega} e^{-2\pi i  (\lambda_1'-\lambda_1)t_1} \cdots e^{-2\pi i  (\lambda_d'-\lambda_d)t_d} dt\\
        =&\int_{\Omega} e^{-2 \pi i \braa{(\lambda'-\lambda),t}} dt\\
        =& F_{\Omega} (\lambda'-\lambda)
    \end{align*}
    First, assume orthogonality, meaning that $\braaMed{e_{\lambda},e_{\lambda'} }_{L^2(\Omega)} = 0 $ if $\lambda \neq\lambda'$. Clearly $F_{\Omega}(\lambda'-\lambda) = 0$ from the above equality. This shows that all distinct pairs $\lambda,\lambda'\in\Lambda$ are containted in the zero-set $\Zstroke_{\Omega}$, and if $\lambda = \lambda'$ we have
    \begin{equation}\label{eq:connection_lamb_equal_lamb}
        \braaMed{e_{\lambda},e_{\lambda} }_{L^2(\Omega)} = F_{\Omega}(0) = \brac{\mesMed{\Omega}}^d \neq 0.
    \end{equation}
    Hence $\lambda'-\lambda \in \Zstroke_{\Omega} \cup \braq{0}$ for all elements $\lambda,\lambda'\in \Lambda$, and we have shown \labelcref{eq:zero_set_inclusion_original}. Conversely, if we assume \labelcref{eq:zero_set_inclusion_original}, we can readily observe that $\lambda'-\lambda \in \Zstroke_{\Omega}$ imply both that $\lambda'-\lambda\neq 0$ and $F_{\Omega} (\lambda'-\lambda) = 0$. Hence from the equality above, $\braaMed{e_{\lambda},e_{\lambda'} }_{L^2(\Omega)} = 0 $ for all distinct elements $\lambda,\lambda'\in \Lambda$. If $\lambda-\lambda'\in \braq{0}$, then we have $\lambda=\lambda'$, which \labelcref{eq:connection_lamb_equal_lamb} shows is non-zero and this completes the proof.  %, and we have shown orthogonality.
\end{proof}

%\begin{remark}\label{rem:key_insight}
    %The important observation from 
    %  The key insight derived from \cref{lem:zero_set_orthoganl_general} is that orthogonality is equivalent to the \emph{non-zero} elements of $\Lambda - \Lambda$ being contained in the zero-set $\Zstroke_{\Omega}$.  % of $F_{\Omega}$.  % , meaning
    %  That is to say: $\lambda,\lambda'$ are distinct, i.e. $\lambda\neq\lambda'$; if and only if $\lambda-\lambda'\neq 0$ and $\lambda-\lambda' \in \Zstroke_{\Omega}$
    % they are in $(\Lambda - \Lambda)\setminus\braq{0}\subseteq \Zstroke_{\Omega}$.
    %
    %* Trenger ikke si så mye om dette nå, tar det opp i kapittel 4.3
    % \begin{equation*}
    %     \brac{\lambda - \lambda'} \in \Zstroke_{\Omega} \quad \text{ for all } \quad \lambda, \lambda'\in \Lambda \quad \text{ where } \quad  \lambda \neq % \lambda', 
    % \end{equation*}
    % \begin{equation*}
    %     \Lambda - \Lambda \subseteq \Zstroke_{\Omega} \quad \text{ for } \lambda, \lambda'\in \Lambda, \quad \lambda \neq \lambda', 
    % \end{equation*}
    % \SigridComment{Not $\subseteq$ here, since we cannot have equality unless we also have the elements $\lambda,\lambda'$ that are zero.}
    % \SigridComment{its the inclusion}
%\end{remark}

%* Dimension one, defined below by the smallest positive zero of F_F_omega
%* Remark. Since $\Omega$ is a bounded set, the function F_omega(z can be defined for all complex z and is an entire function. this guarantees that its zeroes are a discrete set
\begin{remark}
    Additionally, \cref{lem:zero_set_orthoganl_general} also implies that elements of $\Lambda$ cannot be too close to each other. To show this, first note that a \emph{ball} $\mathrm{B}$ of radius $\epsilon$ centered at $a\in\R^d$ can be expressed as 
    \begin{equation*}
        \mathrm{B}(a ; \epsilon) = \braqMed{x\in \R^d : \branMed{x-a}_2 < \epsilon},
    \end{equation*}
    Recall that $0<\mes{\Omega}<\infty$, and that $F_{\Omega}(0)=\brac{\mes{\Omega}}^d $. Since $F_{\Omega}$ is defined for any $z \in \R^d$, this implies that a ball centered at zero with radius $R$, specifically $\mathrm{B}(0; R)$, includes no element of the zero-set $\Zstroke_{\Omega}$. Hence, for all other elements $\lambda-\lambda'\in \Zstroke_{\Omega}$, that is all distinct elements $\lambda,\lambda'\in \Lambda$, we would have $\bran{\lambda-\lambda'}_2\geq R$.
\end{remark}

\end{document}