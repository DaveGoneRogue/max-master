\documentclass[../thesis.tex]{subfiles}
% Separate preamble for this subfile. This preamble is loaded last, so one can override various functions before \begin{document}

% Better comment extension for Vscode colors these comments differently
% Normal comment color
% * Important information
% ! ALERT
% ? Question
% TODO stuff to do
% // This is strikethrough


\begin{document}
\textcolor{orange}{  %! Here
Throughout the theses, we use the following terminology and notation for the inner product and the indicator function.
\begin{definition}\label{def:dot_prod}
    Let $\lambda=(\lambda_1, \dots, \lambda_d)$ and $t=(t_1,\dots t_d)$ be two vectors in $\R^d$. The inner product, denoted $\braa{\cdot, \cdot}$, between $\lambda$ and $t$ is
    \begin{equation*}
        \braaMed{\lambda,t} = \sum_{n=1}^d \lambda_n t_n= \lambda_1 t_1 + \dots + \lambda_d t_d,
    \end{equation*}
    also known as the dot product. 
\end{definition}
} %! Here

\textcolor{orange}{  %! Here
%We define the indicator function as follows. 
\begin{definition}(Indicator function)\label{def:indicator}
    Let $E$ be a subset of a set $X$. The \emph{indicator function}, also known as the \emph{characteristic function} of $E$, is a function $\indicator{E}{t}: X \rightarrow \braq{0,1}$ where
    \begin{equation*}
        \indicator{E}{t}  = 
        \begin{cases} 
            1, &  t\in E,\\
            0, &  t \notin E.
        \end{cases}
        \qedhere
    \end{equation*}
\end{definition}
} %! Here

%* Recall: Heine Borel gives us that compact is equivalent to being closed and bounded for all subsets of R.
Now, given a bounded subset $\Omega\subset \R^d$ with finite and positive measure, consider the $d$-dimensional Fourier transform given by
\begin{equation*}
    \hat{f}(z) = \int_{\R^d} f(t)  e^{-2\pi i \braa{z,t}} dt, \quad z,t\in \R^d,
\end{equation*}
of the indicator function. That is, 
\begin{equation*}
    \indicatorFourier{\Omega}{z} = \int_{\R^d} \indicator{\Omega}{t}  e^{-2\pi i \braa{z,t}} dt, \quad z,t\in \R^d,
\end{equation*} 
which we redefine into the following function $F_{\Omega}(z)$;
\begin{equation*}
    \indicatorFourier{\Omega}{z} = \int_{\R^d} \indicator{\Omega}{t}  e^{-2\pi i \braa{z,t}} dt = \int_{\Omega} e^{-2\pi i \braa{z,t}} dt := F_{\Omega}(z).
\end{equation*}
Since $\Omega$ has finite measure and the exponential has absolute value one, we have that $F_{\Omega}(z)$ is defined for any vector $z = (z_1, \dots z_d)\in \R^d$. Additionally, we introduce the following notation to refer to the \emph{set of real zeroes}, or simply the \emph{zero-set}, of $F_{\Omega}(z)$;
%Additionally, we introduce the following notation to refer to the \emph{set of real zeroes} of $F_{\Omega}(z)$, which we will simply name the \emph{zero-set} as all our results apply to $\R$;
\begin{equation*}
    \Zstroke_{\Omega} = \braqMed{ z \in \R^d \setminus \braqMed{0} : F_{\Omega}(z)= 0}.
\end{equation*}
To conclude this chapter, we show a simple, yet useful, relationship between orthogonality and the zero-set of $F_{\Omega}(z)$ \cite{lagariasOrthonormalBasesExponentials2000,jorgensenSpectralPairsCartesian2001}. %! nevertheless, instead of yet
\begin{lemma}
    Let $\Omega \subset \R^d$ be a bounded subset with $0< \mes{\Omega}<\infty$, and consider the set
    \begin{equation}\label{eq:zero_set_orthogonal}
        \Lambda - \Lambda = \braqMed{\lambda-\lambda' : \lambda,\lambda' \in \Lambda}.
    \end{equation}
    for some subset $\Lambda\subset \R^d$. Then $\Lambda$ gives an orthogonal set of exponentials $E(\Lambda)$ if and only if
    \begin{equation*}
        \Lambda - \Lambda \subseteq \Zstroke_{\Omega} \cup \braq{0}.
    \end{equation*}
\end{lemma}
\begin{proof}
    Before we begin the proof, observe the following connection between $F_{\Omega}(z)$ and the inner product $\braaMed{e_{\lambda},e_{\lambda'} }$ of two elements $\lambda,\lambda'\in \Lambda$ under the assumption that $\lambda \neq \lambda'$.
    \begin{align*}
        \braaMed{e_{\lambda},e_{\lambda'} }_{L^2(\Omega)} =& \int_{\Omega} e_{\lambda}(t)\overline{e_{\lambda'}(t)} dt\\
        =& \int_{\Omega} e^{2\pi i \braa{\lambda,t}} e^{-2 \pi i \braa{\lambda',t}} dt\\
        =& \int_{\Omega} e^{2\pi i  (\lambda_1t_1 + \dots +\lambda_d t_d)} e^{-2\pi i  (\lambda_1' t_1 + \dots +\lambda_d' t_d)} dt\\
        =& \int_{\Omega} e^{2\pi i  (\lambda_1 -\lambda_1')t_1} \cdots e^{2\pi i  (\lambda_d -\lambda_d')t_d} dt\\
        %=& \int_{\Omega} e^{-2\pi i  (\lambda_1'-\lambda_1)t_1} \cdots e^{-2\pi i  (\lambda_d'-\lambda_d)t_d} dt\\
        =&\int_{\Omega} e^{-2 \pi i \braa{(\lambda'-\lambda),t}} dt\\
        =& F_{\Omega} (\lambda'-\lambda),
    \end{align*}
    First, assume orthogonality, which means we have $\braaMed{e_{\lambda},e_{\lambda'} } = 0 $ if $\lambda$ and $\lambda'$ are distinct elements. Clearly $F_{\Omega}(\lambda'-\lambda) = 0$ from the above calculation. This tells us that $\lambda-\lambda'\in \Zstroke_{\Omega}$. If $\lambda = \lambda'$ we would have 
    \begin{equation*}
        \braaMed{e_{\lambda},e_{\lambda} }_{L^2(\Omega)} = \int_{\Omega} 1 dt = \prod^d \mes{\Omega} \neq 0
    \end{equation*}
    Hence we have that $\lambda-\lambda' \in \Zstroke_{\Omega} \cup \braq{0}$ for all elements $\lambda,\lambda'\in \Lambda$
    CONVERSE. 
    
\end{proof}


%! sjekke opp om z er i C over og om vi må spesifisere settet av reele nuller. 
\SigridComment{Må bytte fra C til R notasjon i argumentasjonen, dvs, ikke entire etc.} %! VIKTIG
NOE OM AT DEN ER VELDEFINERT FOR ALLE SET OMGEGA NÅR VI HAR DENNE FOURIERTRANSFORMASJONEN.
Note that $\indicatorFourier{\Omega}{z}$ is an entire function for any $z\in \C^d$ since $\Omega$ is compact. % Think the case where JP says: "It is defined for any real vector ($Z\in \R^d$), because $\Omega$ has finite measure and that the absolute value of $e^{-2i\pi \braa{z,t}}$ is one." Two reasons for that formulation. First, He considers real vectors $z$ (although they are of a complex value function). Secondly, we must thus consider the absolute value of the exponential, which is real. 

Furthermore, we use the following notation to denote the \emph{zero-set} of $\indicatorFourier{\Omega}{z}$,
\begin{equation*}
    \Zstroke_{\Omega} = \braqMed{ z \in \R^d \setminus \braqMed{0} : \indicatorFourier{\Omega}{z} = 0}.
\end{equation*}

Before showing a useful result, we can, without loss of generality and ease of notation, redefine our function to be
\begin{equation*}
    \indicatorFourier{\Omega}{z} = \int_{\R^d} \indicator{\Omega}{t}  e^{-2\pi i \braa{z,t}} dt = \int_{\Omega} e^{-2\pi i \braa{z,t}} dt := F_{\Omega}(z)
\end{equation*}

\begin{lemma}
    Let $\Omega$ be BLALBA in $\R^d$. A set $\Lambda$ gives an orthoganl set of exponentials $E(\Lambda)$ in $L^2(\Omega)$ if and only if
    $\Lambda - \Lambda \subseteq \Zstroke_{\Omega} \cup \braq{0}$


    \SigridComment{Se på non zero elements of direkte undermengde}
\end{lemma}

\end{document}