\documentclass[../thesis.tex]{subfiles}
% Separate preamble for this subfile. This preamble is loaded last, so one can override various functions before \begin{document}

% Better comment extension for Vscode colors these comments differently
% Normal comment color
% * Important information
% ! ALERT
% ? Question
% TODO stuff to do
% // this is strikethrough


\begin{document}



%! Generelt intro ish til det under
\section{Tiling sets for the n-cube}


% ! Def tiling ??

%\begin{definition}[Tiling set]
%    Let $\Omega \subset \mathbb{R}^d$ be a subset with nonzero measure, and consider a set $T \subseteq \mathbb{R}^d$. If the set of translates ${}$$\{\Omega+l: l\in T\}$ cover $\mathbb{R^d}$ up to measure zero, and if all intersections $(\Omega+l) \cap (\Omega+l')$  for $l\neq l'$ in $L$ have measure zero, then $\Omega$ is called a \emph{tile}, and $T$ is called a \emph{tiling set} for $\Omega$. We say that $(\Omega, T)$ is a \emph{tiling pair}. 
%\end{definition}

%In other words, the shifts $\Omega + T$ constitute a \emph{measuredisjoint covering} of $\mathbb{R}^d$, and we can say that $\Omega$ \emph{tiles} $\mathbb{R}^d$ \emph{by translation}, or that $\Omega+T$ is a \emph{tiling} of $\mathbb{R}^d$.


% Fuglede’s spectral set conjecture. Let $\Omega$􏱁 be a set in $\mathbb{R}^d$ with positive and finite Lebesgue measure. Then 􏱁$\Omega$ is a spectral set if and only if $\Omega$􏱁 tiles $\mathbb{R}^d$ by translation.

% An equivalent restatment of Fuglede was presented in JorgenPedersen. 

% Let $\Omega \subset \mathbb{R}^d$ have positive and finite Lebesgue measure. Then $\Omega$ is a spectral set if and only if $\Omega$ is a tile, i.e., there exists a set $\Lambda$ so that $(\Omega, \Lambda)$ is a spectral pair if and only if there exists a set $\Lambda^{\prime}$ so that $\left(\Omega, \Lambda^{\prime}\right)$ is a tiling pair.

%with the following dual conjectures.

% Conjecture 1.3: Let $\Lambda \subset \mathbb{R}^d$. Then $\Lambda$ is a spectrum if and only if $\Lambda$ is a tiling set, i.e., there exists a set $\Omega$ so that $(\Omega, \Lambda)$ is a spectral pair if and only if there exists a set $\Omega^{\prime}$ so that $\left(\Omega^{\prime}, \Lambda\right)$ is a tiling pair.

% Conjecture 1.4: Let $\Lambda \subset \mathbb{R}^d$. Then $\left(I^d, \Lambda\right)$ is a spectral pair if and only if $\left(I^d, \Lambda\right)$ is a tiling pair.

%When  $\Omega = I^d$ the connection between tiles and spectrum is more direct than for other examples of sets $\Omega$. As shown in sigrid_note:lagarias_reeds_wang and Spectral and tiling properties of the unit cube_ALEX IOSEVICH AND STEEN PEDERSEN it is possible to classify all spectra by showing that $\Lambda$ is a spectrum for the unit cube $I^d$, if and only if $I^d$ tiles $\mathbb{R}^d$ by $\Lambda$-translates.







\end{document}