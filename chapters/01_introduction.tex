
\documentclass[../thesis.tex]{subfiles}
% Separate preamble for this subfile. This preamble is loaded last, so one can override various functions before \begin{document}

% Better comment extension for Vscode colors these comments differently
% Normal comment color
% * Important information
% ! ALERT
% ? Question
% TODO stuff to do
% // This is strikethrough


\begin{document}
%* ———————————————— !! OVERVIEW !! ————————————————
%!Vurdere å slette parantesene med navn. 
%The two main objects of study in this thesis are \emph{spectral pairs} and their connection to \emph{tiling pairs}.
This thesis's two main objects of study are \emph{spectral pairs} and their connection to \emph{tiling pairs}. While we will provide a formal definition later, we intuitively understand them to be the following; A spectral pair $\brac{\Omega,\mathcal{F}}$ represents a complementary relationship between a \emph{spectral set} $\Omega$ and a set of frequencies $\mathcal{F}$, where the exponential functions derived from $\mathcal{F}$ form a complete representation of any square-integrable function defined on $\Omega$. On the other hand, a tiling pair $\brac{\Omega,\mathcal{T}}$ represents a complementary arrangement of the set $\Omega$ (\emph{a tile}) in which all elements specified by the set of translations $\mathcal{T}$ are utilized to form a complete surface covering without any gaps or overlaps. Despite their individual importance in the fields of analysis and geometry, the relation between spectra and tiling pairs still remains an enigma and continues to be an active area of research, see \cite{levFugledeConjectureConvex2022,kissFugledeConjectureHolds2022}. The origin of this area of research is due to a now-famous paper by Fuglede \cite{fugledeCommutingSelfadjointPartial1974}, who put forward the idea that there should be a geometric way to describe spectral sets \cite{lagariasOrthonormalBasesExponentials2000,liDualityPropertiesSpectra2010}. 

\begin{conjecture}[Spectral set conjecture]
    Let $\Omega\subset \R^d$ be a measureble subset with $0<\mes{\Omega}<\infty$. Then $\Omega$ is a spectral set if and only if $\Omega$ is a tile. 
\end{conjecture}

We remark that the conjecture does not claim a connection between the \emph{spectrum} ($\mathcal{F}$) of $\Omega$ and the \emph{tiling set} ($\mathcal{T}$) of $\Omega$, meaning that there exists a set $\mathcal{F}$ such that $\brac{\Omega,\mathcal{F}}$ is a spectral pair if and only if there exists a set $\mathcal{T}$ such that $\brac{\Omega,\mathcal{T}}$ is a tiling pair. However, in certain specific cases, such as the one examined in this thesis, one apparent connection between the spectrum and tiling set arises, indicated in this specific case by the fact that $\mathcal{F}=\mathcal{T}$. Furthermore, Fuglede himself showed that the conjecture is true if one adds a \emph{lattice} assumption. Meaning, if $\brac{\Omega,\mathcal{T}}$ is a tiling pair, then $\brac{\Omega,\mathcal{T^*}}$ is a spectral pair, and conversely. Here $\mathcal{T^*}$ denotes the \emph{dual lattice} \cite{fugledeCommutingSelfadjointPartial1974}.

We refer the reader to the following sources for more in-depth coverage of the spectral set conjecture and highlight the most important results: The conjecture remains open only in dimensions one and two, and in both directions, following the discovery of counterexamples in all dimensions $d \geq 3$ where non-convex spectral sets $\Omega$ that do not tile by translation. Additionally, the conjecture is known to be true in all dimensions if one adds the assumption that $\Omega$ is a convex body. For further details, refer to \cite{levFugledeConjectureConvex2022,dutkayReductionsSpectralSet2014,liDualityPropertiesSpectra2010,farkasFugledeConjectureExistence2006,kolountzakisStudyTranslationalTiling2003,jorgensenSpectralPairsCartesian2001}. It is worth noting that the spectral set conjecture represents just one of many more intriguing aspects than what we might have eluded to earlier and that \cite{kolountzakisStudyTranslationalTiling2003} also highlights the many connections between tilings by translation and the broader field of functional analysis. 


%! LEGGE TIL OVER: One common goal of the prior efforts is to clarify the relations between spectra and tilings

%* ————————————————
INTRO TIL CUBES OG TEMAET FOR OPPGAVEN

The main theorem of study in this thesis is the following result. 
The main object of study in this thesis is the following result 





Jorgensen and Pedersen conjectured a special case of this FUGLE in \cite{jorgensenSpectralPairsCartesian2001} letting $\Omega = I^d$, the $d$-dimensional unit cube. 
\begin{conjecture}
    Let $\Lambda\subset \R^d$. Then $\brac{\Omega,\Lambda}$ is a spectral pair if and only if $\brac{\Omega,\Lambda}$ is a tiling pair. 
\end{conjecture}

The importance of this special case is also due to a more direct connection between tiles and spectra for the $d$-dimensional unit cube than for other examples of sets $\Omega$, a subject we will further elaborate on in the thesis. 
%! the same set $\Lambda\subset\R^d$ is indeed a spectrum and tiling set for $\Omega = I^d$. 
%! Highlight denne direkte linken i chapter 5 med keller. dvs: the connection between tiles and spectrum is more direct for OMEGA=I^d than for other sets due to "NULL sett tingen" and the corresponding result for tilings (Keller). 



After the conjecture of in the case where $\Omega = I^d$, two independent proofs of the conjecture was made. 

\begin{theorem}
    Let $\Lambda\subset \R^d$ and $\Omega=\bras{0,1}^d$ be the unit cube in $\R^d$. Then $\brac{\Omega,\Lambda}$ is a spectral pair if and only if $\brac{\Omega,\Lambda}$ is a tiling pair.
\end{theorem}


Etter theoremet av denne spesielle casen. 

In general there does not seem to be any simple relation between tiling sets T used with tile omega in the space domain and the set of spectra lambda for omega. se referanser. 



Theorem om cube begge veier.
This simple result is rather unexpected. It is intuitively clear when L is a periodic set, but, perhaps, surprising is that it holds without any assumptions on the set L.


A related problem is: given a specific set Ω that tiles space by translation, determine its spectra. Because of its simplicity, the cube has been studied the most. Lagarias, Reeds and Wang [9] and Iosevich and Pedersen [3] recently proved that if Q = (−1/2,1/2)d is the unit cube in ‚d, then Q+Λ is a tiling if and only if EΛ is an orthonormal basis for Q. We remark here that there exist ‘exotic’ translational tilings by the unit cube which are non-lattice; see . This had been conjectured by Jorgensen and Pedersen [4], where it was proved for dimension d T 3. The purpose of our paper is to give an alternative and, perhaps, more illuminating proof of this fact, which is based on a characterization of translational tiling by a Fourier analytic criterion.


I would say, "what has become the classical" 
The classical example of a spectral set is the unit cube Ω = I^d, for which the set Λ = Zd serves as a spectrum.


%* I&P 
Cube tilings have a long history beginning with a conjecture due to Minkowski: in every lattice tiling of Rd by translates of Q some cubes must share a complete (d − 1)-dimensional face. Minkowski’s conjecture was proved in [Haj], see [SS] for a recent exposition. Keller [Kel] while working on Minkowski’s conjecture made the stronger conjecture that one could omit the lattice assumption in Minkowski’s conjecture. Using [Sza] and [CS] it was shown in [LS] that there are cube tilings in dimensions d ≥ 10 not satisfying Keller’s conjecture.

The study of the possible spectra for the unit cube was initiated in [JP2], where Theorem 1.1 was conjectured. Theorem 1.1 was proved in [JP2] if d ≤ 3 and for any d if T is periodic. The terminology spectrum for Q originates in a problem about the existence of certain commuting self-adjoint partial differential operators. We say that two self-adjoint operators commute if their spectral measures commute, see [RS] for an introduction to the theory of unbounded self-adjoint operators. The following result was proved in [Fu] under a mild
regularity condition on the boundary, the regularity condition was removed in [Pe1].

Remark 1.2. As we shall discuss below there exists highly counter-intuitive cube-tilings in Rd for sufficiently large d. Those tilings can be much more complicated than lattice tilings. 
Theorem 1.1 is clear if T is a lattice. The point of Theorem 1.1 is that the result still holds even if the restrictive lattice assumption is dropped.

%* L&W
EGt alt det jeg allerede har markert ut. MYE BRAAA

fra 
%* JP er det eneste nyttige på første side, og dette har jeg drodlet litt ned på allered



%* Fra Kineseren om det andre beviset 
Iosevich and Pedersen [7] later and independently established the above-mentioned conjecture of Jorgensen and Pedersen by a different approach basing on the geometric argument, which is analogous to the argument used by Perron [20] in his proof of Keller’s conjecture for dimension np6: Kolountzakis [11] gave an alternative proof of this fact, which is based on a characterization of translational tiling by a Fourier analytic criterion. 


%* KINESEREN MED DUALITET
(1) Overview. It is well-known in the classical Fourier series that for the n-cube [0,1]n, the spectrum Λ = Zn gives the standard Fourier basis EΛ of L2([0, 1]n). Also [0, 1]n tiles Rn by Zn-translations.

A little generalization of these two facts is to replace Zn by a full rank lattice L and to replace [0, 1]n by a tile ΩL with the tiling set L, and conclude that (ΩL,L∗) is a spectral pair, i.e., EL∗ is an orthogonal basis of L2(ΩL), where L∗ is the dual lattice of L. Equivalently (ΩL,L) is a tiling pair 

The combination of these two features (spectrum and tiling) on the n-cube [0,1]n or on the tile ΩL with the tiling set L is used in a variety of problems in analysis and geometry. One subject in this direction is concentrated on the question of whether L and ΩL can be replaced by a certain discrete set Λ ⊂ Rn and a certain Lebesgue measurable set Ω ⊂ Rn respectively. In most cases it is more difficult to answer this question. Therefore the research on the relationship between spectra and tilings becomes a main subject of considerable interests.

The conjecture is complicated by the fact that in general a spectral set or a tile can have many different spectra and many different tiling sets. The n-cube has an enormous number of spectra and of tiling sets. The known results and examples support the view that certain specific classes of tiling sets correspond to certain corresponding classes of spectra.

%* ————————————————
Before beginning the thesis we highlight a few areas of importance where spectra and tilings are used. 
They are closely connected to wavelets where similiar connection AND COMPLEX HADAMARD MATRICIS. 

in fact, tilings are not only related to the subject 



These two connected concepts from seemingly different areas of mathematics, have applications in various fields, for example, wavelet theory (CITE) and harmonic analysis and quantum mechanics (cite EStimates on spectrum of ffractals paper).  


Wavelets are mentioned in the "structure of tilings of the line by a function."

study of gabor bases, page 10 in lit. rewiev. 
%* ————————————————

Rettelse, er det bedre å begynne med følgende

Temaet for dette er følgende theorem

det ble først conjured i blabla og bevist her. og her 


det linker opp til cube tilings og 


som en specail case av følgende conjecture av fuglede.

her er status på fuglede. 

og spectral sets. 


slutnning og la oss begynne. 




\end{document}