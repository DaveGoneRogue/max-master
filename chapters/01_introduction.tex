
\documentclass[../thesis.tex]{subfiles}
% Separate preamble for this subfile. This preamble is loaded last, so one can override various functions before \begin{document}

% Better comment extension for Vscode colors these comments differently
% Normal comment color
% * Important information
% ! ALERT
% ? Question
% TODO stuff to do
% // This is strikethrough


\begin{document}
%* ———————————————— !! OVERVIEW !! ————————————————
%The two main objects of study in this thesis are \emph{spectral pairs} and their connection to \emph{tiling pairs}.
This thesis's two main objects of study are \emph{spectral pairs} and their connection to \emph{tiling pairs}. While we will provide a formal definition later, we intuitively understand them to be the following; A spectral pair $\brac{\Omega,\mathcal{F}}$ represents a complementary relationship between a \emph{spectral set} $\Omega$ and a \emph{set of frequencies} $\mathcal{F}$, where the exponential functions derived from $\mathcal{F}$ form a complete representation of any square-integrable function defined on $\Omega$. On the other hand, a tiling pair $\brac{\Omega,\mathcal{T}}$ represents a complementary arrangement of the set, or more specifically, a \emph{tile} $\Omega$, in which all elements specified by the \emph{set of translations} $\mathcal{T}$ are utilized to form a complete covering of a space without any gaps or overlaps. Despite their individual importance in the fields of analysis and geometry, the relation between spectra and tiling pairs still remains an enigma and continues to be an active area of research, see \cite{levFugledeConjectureConvex2022,kissFugledeConjectureHolds2022}. The origin of this area of research is due to a now-famous paper by Fuglede \cite{fugledeCommutingSelfadjointPartial1974}, who put forward the idea that there should be a geometric way to describe spectral sets \cite{lagariasOrthonormalBasesExponentials2000,liDualityPropertiesSpectra2010}. 

\begin{conjecture}[Spectral set \namecref{conj:fuglede} or Fuglede's \namecref{conj:fuglede}]\label{conj:fuglede}  
    %! LEGGE TIL ET STED: One common goal of the prior efforts is to clarify the relations between spectra and tilings
    Let $\Omega\subset \R^d$ be a bounded subset with positive finite measure. Then $\Omega$ is a spectral set if and only if $\Omega$ is a tile. 
\end{conjecture}

We remark that the \namecref{conj:fuglede} does not claim a connection between the \emph{spectrum} $\mathcal{F}$ of $\Omega$ and the \emph{tiling set} $\mathcal{T}$ of $\Omega$, meaning that there exists a set $\mathcal{F}$ such that $\brac{\Omega,\mathcal{F}}$ is a spectral pair if and only if there exists a set $\mathcal{T}$ such that $\brac{\Omega,\mathcal{T}}$ is a tiling pair. However, there can be more direct connections between spectra and tiling sets in certain special cases, such as the one examined in this thesis. To give an example of what such a connection looks like, one can have $\mathcal{F}=\mathcal{T}$ for some set $\Omega$, clearly also showing that the \namecref{conj:fuglede} is true for this particular case. Additionally, Fuglede himself showed that the \namecref{conj:fuglede} is true if one adds a \emph{lattice} assumption. Meaning, if $\brac{\Omega,\mathcal{T}}$ is a tiling pair, then $\brac{\Omega,\mathcal{T^*}}$ is a spectral pair, and conversely. Here $\mathcal{T^*}$ denotes the \emph{dual lattice}, which also highlights a different kind of connection \cite{fugledeCommutingSelfadjointPartial1974}.

We refer the reader to the following sources for more in-depth coverage of the spectral set \namecref{conj:fuglede} and highlight the most important results: The \namecref{conj:fuglede} remains open only in dimensions one and two, and in both directions, following the discovery of counterexamples in all dimensions $d \geq 3$ involving \emph{non-convex} spectral sets $\Omega$ that do not tile by translation. Furthermore, the \namecref{conj:fuglede} is known to be true in all dimensions if one adds the assumption that $\Omega$ is a \emph{convex body}. For further details, refer to \cite{levFugledeConjectureConvex2022,dutkayReductionsSpectralSet2014,liDualityPropertiesSpectra2010,farkasFugledeConjectureExistence2006,kolountzakisStudyTranslationalTiling2003,jorgensenSpectralPairsCartesian2001}. It is worth noting that the spectral set \namecref{conj:fuglede} is just one of the many intriguing aspects within the broader field of functional analysis to which tilings are connected. This is also further detailed in \cite{kolountzakisStudyTranslationalTiling2003}. 

This thesis focuses on the special case when the set $\Omega$ is the \emph{$d$-dimensional unit cube}, $\bras{0,1}^d$. While the unit cube may appear simple, it has been extensively studied, not only due to its simplicity but also because of its significant role in Fourier analysis and tilings. Interestingly, it has even been used to construct the aforementioned \emph{non-convex} sets that counterexample the spectral set \namecref{conj:fuglede} by arranging the unit cubes in specific arithmetic configurations \cite{levFugledeConjectureConvex2022}. Nevertheless, it is well-known that the spectrum $\mathcal{F}=\Z^d$ corresponds to the standard Fourier basis for the space of square-integrable functions restricted to $\bras{0,1}^d$ \cite{lagariasOrthonormalBasesExponentials2000}. It is also well-known that the unit cube tiles $\R^d$ with the tiling set $\mathcal{T}=\Z^d$. Moreover, the history of cube-tilings spans well over a century, starting with Minkowski's conjecture on lattice tilings in $1907$, which later led to Keller's \namecref{conj:keller_tiling} on non-lattice tilings, the latter of which will be a topic further explored in this thesis \cite{liDualityPropertiesSpectra2010,iosevichSpectralTilingProperties1998}. It is worth mentioning that both of these conjectures have been fully resolved in \cite{hajosUeberEinfacheUnd1942,brakensiekResolutionKellerConjecture2020}, respectively.

Our goal is to show a connection between spectra and tiling sets for the unit cube. As we have alluded to above, $\mathcal{F}=\mathcal{T}=\Z^d$, there appears to be a more direct connection between tiles and spectra for $\bras{0,1}^d$ than for other examples of sets $\Omega$. We study the ensuing special case \cite{iosevichSpectralTilingProperties1998}.

\begin{theorem}\label{thrm:main_result}
    Let $\Lambda\subset \R^d$ and $\Omega=\bras{0,1}^d$ be the unit cube in $\R^d$. Then $\brac{\Omega,\Lambda}$ is a spectral pair if and only if $\brac{\Omega,\Lambda}$ is a tiling pair.
\end{theorem}

The result originates from a conjecture by Jorgensen and Pedersen, who initially proved it for $d\leq3$ and for all dimensions if $\Lambda$ was assumed to be a \emph{periodic set} \cite{jorgensenSpectralPairsCartesian2001}. What makes \cref{thrm:main_result} particularly intriguing is that it holds in \textsc{any} dimension, without \textsc{any} assumptions on $\Lambda$, and even when the restrictive lattice condition, which Fuglede proved, is removed. After the conjecture was initially put forward, two independent proofs using different approaches were given by Iosevich and Pedersen \cite{iosevichSpectralTilingProperties1998} and Lagarias, Reeds, and Wang \cite{lagariasOrthonormalBasesExponentials2000}. Later Kolountzakis showed that \cref{thrm:main_result} can be proven without the use of Keller's theorem, which both previous proofs used \cite{kolountzakisPackingTilingOrthogonality2000}. 

As we will see in the thesis, the unit cube has a huge variation of vastly different spectra and tiling sets. One of the main contributions of this thesis is that the unit cube can have aperiodic tilings. Compared to a typical spectral set, it appears likely that such sets have a \emph{unique} spectrum up to translations, making the unit cube a non-trivial object to study. By typical spectral set, we mean a typical \emph{fundamental domain} $\Omega$ of a \emph{full-rank} lattice, where one can show that the corresponding unique spectrum is the dual lattice, meaning $\Lambda = L^*$ \cite{lagariasOrthonormalBasesExponentials2000}. Furthermore, the above result, \cref{thrm:main_result}, also suggests that classes of tiling sets correspond to classes of spectra. By classes, we mean classifications of \textsc{all} spectra and tiling sets for the unit cube for a particular dimension. We will further elaborate on this subject in the thesis. 

%* Possible addition on the connection between our problem and Fugeledes conjecture
% linking our problem back to the conj. the question one wonders, BLABLA, duality paper and L & R & W paper. (see comment at the end)
%
The text is structured as follows:
\begin{itemize}
  \item We begin with a \namecref{chap:tiling} covering the preliminaries, where we lay the foundation for the subsequent discussions.
  \item The third \namecref{chap:tiling} delves into the topic of tilings. Here, we discuss the concept of tilings in general and provide a comprehensive exploration of counterintuitive cube-tilings in higher dimensions. Additionally, we undertake the classification of all tiling sets in dimension one. This \namecref{chap:tiling}'s most significant result is a new construction of an aperiodic tiling of unit cubes in all dimensions $d\geq3$. We close the \namecref{chap:tiling} with another main result, the proof of Keller's \namecref{thrm:keller_tiling}, which is central to the thesis.
  \item The fourth \namecref{chap:tiling} focuses on spectral pairs. We start by defining spectral sets and examining their properties. Additionally, we investigate the construction of spectral pairs in higher dimensions using spectra and spectral pairs from lower dimensions. Here we devote considerable effort to demonstrating and classifying all spectra in dimensions one and two.
  \item Finally, we conclude the thesis with a \namecref{chap:tiling} on the connection, or more specifically, the equivalence of spectra and tiling sets for the unit cube. We also classify all tilings sets in dimension two. While the general result of \cref{thrm:main_result} lies beyond the scope of this thesis, we provide a detailed analysis of the two independent proofs mentioned above in this final \namecref{chap:tiling}.
\end{itemize}




\mycomment{  %! Optional if time
%* ————— Applications
%* applications referred to by the Chinese guy
Before beginning the thesis, we highlight a few areas of importance where spectra and tilings are used. 
They are closely connected to wavelets where similar connection AND COMPLEX HADAMARD MATRICES. 

In fact, tilings are not only related to the subject 


These two connected concepts from seemingly different areas of mathematics have applications in various fields, for example, wavelet theory (CITE) and harmonic analysis and quantum mechanics (cite Estimates on the spectrum of fractals paper).  


Wavelets are mentioned in the "structure of tilings of the line by a function."

Study of Gabor bases, page 10 in the literature review document. 


%* The combination of these two features (spectrum and tiling) on the n-cube [0,1]n is used in a variety of problems in analysis and geometry. 
%* One subject in this direction is concentrated on the question of whether L and ΩL can be replaced by a certain discrete set Λ ⊂ Rn and a certain 
%* Lebesgue measurable set Ω ⊂ Rn, respectively. In most cases, it is more difficult to answer this question. Therefore the research on the 
%* relationship between spectra and tilings becomes a main subject of considerable interest.
}

\mycomment{  %! Paper by: L & R & W
De stater to ekvivalente theoremer. Den første er det som J&P conj. og den andre er en slags generalisering. 
Videre gjør de noen merknader på "rommene" de lever i, og hva som skiller de. 

Det som er vanskelig å skjønne der er det de sier etter fuglede, om hvordan dette linker opp til han sin conj. og tematikken generelt. 
}
\end{document}